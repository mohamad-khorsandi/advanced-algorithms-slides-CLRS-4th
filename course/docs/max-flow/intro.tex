%------------------------------------------------------------------------
\begin{itemframe}{معرفی مسئله}
\itm
در دنیای واقعی شبکه‌های بسیاری وجود دارند که با هدف انتقال چیزی ساخته شده‌اند. برای مثال شبکه‌های آب رسانی، مدارات الکتریکی(انتقال جریان الکتریکی)، خطوط راه‌آهن و غیره. به طور معمول در چنین شبکه‌هایی افزایش نرخ انتقال مطلوب است. بنابراین مسئله شار بیشینه می‌پرسد‌: چگونه می‌توان نرخ انتقال را در این شبکه‌ها بیشینه کرد؟
\itm
برای حل یک مسئله شار بیشینه باید شبکه را توسط گراف مدلسازی کرد. به این گراف «شبکه شار»
\fn{flow network}
و به هر چیزی که در این شبکه در جریان باشد به طور کلی شار
\fn{flow}
 گفته می‌شود.
\end{itemframe}

%------------------------------------------------------------------------
\begin{itemframe}{معرفی مسئله}
\itm
برای درک بهتر مسئله شار بییشیه ويژگی‌های این مسئله را همراه با مثال شبکه آب رسانی توضیح می‌دهیم؛
\itm
در مسئله شار بیشیه گراف ثابت فرض می‌شود. (مثال: خطوط آب‌رسانی مثل لوله‌ها و اتصالات از قبل ساخته شده‌ و غیر قابل تغییر اند.)
\itm
هر یال یک ظرفیت مشخص برای انتقال شار دارد و نمی‌تواند بیشتر از آن مقدار انتقال دهد. (مثال: هر لوله آب -بسته به قطر لوله‌- می‌تواند مقدار آبی مشخصی را از خود عبور دهد.)
\itm
سرعت انتقال شار در سراسر شبکه ثابت فرض می‌شود. (مثال: نمی‌توانیم تعیین کنیم آب با فشار بیشتری وارد لوله‌ها شود بنابراین سرعت حرکت آب در لوله‌ها یکسان است.)
\end{itemframe}

%------------------------------------------------------------------------
\begin{itemframe}{معرفی مسئله}

\itm
در مسئله شار بیشینه تنها متغییری که ما می‌توانیم تعیین کنیم این است که چقدر از ظرفیت هر یال برای انتقال شار استفاده کنیم. (مثال: می‌توانیم یک شیر آب سر راه هر لوله قرار دهیم و بعد از حل مسئله شار بیشینه تعیین کنیم هر شیر اجازه ورود چه حجمی از آب را بدهد.)
\itm
مجموع شار ورودی به هر رأس باید با مجموع شار خروجی برابر باشد. به عبارت دیگر هیچ شاری نمی‌تواند در رأس‌های گراف ذخیره شود یا از بین برود. به این قانون، قانون بقای شار
\fn{flow conservation}
گفته می‌شود.
(مثال: بدیهیست در اتصالاتی که لوله‌ها را به هم وصل می‌کند آب نمی‌تواند ذخیره شود یا نشت کند. ممکن هر اتصال یک یا چند لوله ورودی و خروجی داشته باشد در هر صورت مجموع آب ورودی و خروجی باید برابر باشد.)

\end{itemframe}

%------------------------------------------------------------------------
\begin{itemframe}{معرفی مسئله}

\itm
با این توضیحات شاید به نظر برسد که برای حل این مسئله کافیست به طور حریصانه از همه ظرفیت همه یال‌ها برای انتقال استفاده کنیم. (مثال: همه شیر آب‌هایی که سر راه لوله‌ها قرار گرفته اند را تا بیشترین مقدار باز کنیم تا در صورتی که مقدار کافی آب به آن لوله رسید از تمام ظرفیت آن لوله برای انتقال آب استفاده کنیم.)
\itm
نکته جالب اینجاست که برخلاف انتظار این الگوریتم حریصانه جواب بهنیه را تولید نمی‌کند. در ادامه مثال‌هایی خواهیم دید که ممکن است کاهش مقدار شار گذرنده از یک یال باعث افزایش شار کلی گذرنده از شبکه شود. (مثال: در شبکه آب رسانی ممکن است با بستن شیر، اجازه ورود حجم کمتری از آب به یک لوله را بدهیم و با این کار شار گذرنده از شبکه را بیشتر کنیم.)
\end{itemframe}

