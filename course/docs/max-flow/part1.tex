%------------------------------------------------------------------------
\begin{itemframe}{شبکه شار}
\itm
قبل ادامه بحث بهتر است با یک تعریف دقیق و رسمی از شبکه شار آشنا شویم؛
\itm
شبکه شار یک گراف جهت‌دار است. به این معنی که شار نمی‌تواند در یک یال در دو جهت حرکت کند.

شبکه شار همرا با تابع c به ورودی مسئله داده می‌شود. تابع c یال‌ها و ظرفیت‌‌ها را نگاشت می‌کند به طوری که ظرفیت یال u به v برابر است با
$c(u, v)} .
\itm
شبکه شار دارای دو رأس خاص است که ورود شار به شبکه (و یا تولید شار) و خروج از شار شبکه (و یا مصرف شار) را مدل می‌کنند. به این دو رأس منبع
\fn{source}
و مقصد
\fn{sink}
گفته می‌شود. به طور معمول به رأس منبع با s و رأس مقصد با t نشان داده می‌شوند.
\itm
رأس منبع و مقصد تنها رئوسی هستند که از قانون بقای شار پیروی نمیکنند.
\end{itemframe}

%------------------------------------------------------------------------
\begin{itemframe}{شبکه شار}
\itm
همانطور که قبلا اشاره شد، گراف جهت‌دار می‌تواند طوقه داشته‌باشد. امّا وجود طوقه در شبکه شار غیر مجاز است.
\itm
همچنین دیدیم که وجود یال‌های پادموازی نیز در گراف جهت‌دار بلامانع است. امّا در شبکه شار یال‌های پادموازی هم غیر مجاز اند.(در بخش يادآوری فصل یال‌های پادموازی بحث شده‌اند.)
\itm
حذف طوقه (دور به طول ۱) و یال‌های پادموازی (دور به طول ۲) از پیچیدگی مسئله می‌کاهد.
\itm
وجود دورهایی با طول بالا تر در شبکه شار مجاز است.
\end{itemframe}

%------------------------------------------------------------------------
\begin{itemframe}{شبکه شار}

\itm
همچنین باید یک تعریف کمی برای شار کل شبکه ارائه کنیم تا مشخص شود منظور از شار بیشینه چیست؛
\itm
خروجی این مسئله کردن تابع f است که یال‌ها را به شار گذرنده از آنها نگاشت می‌کند. به این صورت که شار گذرنده از یال u به v برابر است با
$f(u, v)} .

\itm
تابع f باید دو ويژگی داشته باشد؛ قانون بقای شار را نقض نکند و از مقدار ظرفیت هر یال تجاوز نکند. در صورتی که تابعی مانند
$f'$
این ویژگی‌ها را نقض کند یک تابع شار نیست.

\end{itemframe}

%------------------------------------------------------------------------
\begin{itemframe}{شبکه شار}
\itm
مقدار شار کل شبکه با |f| نشان داده می‌شود و به این صورت تعریف می‌شود: \\
\begin{center}
$|f| = \sum_{v \in V} f(s, v) - \sum_{v \in V} f(v, s)$
\end{center}
\itm
به زبان ساده این عبارت مجموع شار خالص خروجی از رأس منبع را مشخص می‌کند :مجموع شار خارج شونده از منبع منهای مجموع شار وارد شونده به منبع. (معمولاً مجموع شار وارد شونده به منبع صفر است.)
\itm
قرارداد می‌کنیم که اگر بین u و v یال وجود نداشته باشد
$f(u, v)$
برابر صفر است.
\end{itemframe}
%------------------------------------------------------------------------
\begin{itemframe}{شبکه شار}
\itm
امّا چطور این کمیّت می‌تواند نماینده شار گذرنده از کل شبکه باشد؟
\itm
برای درک بهتر این موضوع به شبکه شار به چشم یک بلوک بزرگ نگاه کنید که از رأس منبع شار به آن وارد و از مقصد خارج می‌شود.
\centerimg[.5]{figs/max-flow/6.png}

\itm
با توجه به قانون بقای شار، شار نمی‌تواند در این بلوک بزرگ بماند بنابراین با همان نرخی که وارد آن می‌شود باید از آن خارج شود. (فرض کنید پیکان‌ها شار خالص را نشان می‌دهند.)‌
\itm
بنابراین نرخ خالص شار خروجی از منبع نماینده شاری است که در کل شبکه جریان دارد.
\end{itemframe}


%------------------------------------------------------------------------
\begin{itemframe}{مدل‌سازی مسئله شار بیشینه}
\itm
شاید دقت کرده باشید که در تعریف شبکه شار فرض‌های ساده کننده‌ایی درنظر گرفتیم که لزوماً در یک مسئله واقعی بیشینه‌سازی شار، برقرار نیستند. دو فرض به این شکل داشتیم که در زیر آورده شده‌اند؛
\item[۱]
ممکن است در یک مسئله واقعی بیشینه‌سازی شار چند منبع و چند مقصد داشته باشیم. در حالی که در تعریف شبکه شار تنها یک منبع و مقصد برای شبکه لحاظ کردیم.
\item[۲]
ممکن است در یک مسئله واقعی بیشینه‌سازی شار یال موازی داشته باشیم. درحالی که وجود چنین یالی را در شبکه شار غیر مجاز دانستیم.
\end{itemframe}

%------------------------------------------------------------------------
\begin{itemframe}{مدل‌سازی مسئله شار بیشینه}
\itm
در ادامه نشان می‌دهیم که چطور یک مسئله واقعی بیشینه‌سازی شار را که دارای چند منبع و مقصد است و یال پادمتقارن دارد را مدل کنیم.

\end{itemframe}

%------------------------------------------------------------------------
\begin{itemframe}{مدل‌سازی مسئله شار بیشینه}
\itm
شکل زیر نشان می‌دهد چطور یک شبکه شار با چندین رأس منبع و مقصد را می‌توان به یک شبکه شار با یک منبع و مقصد تبدیل کرد.
\centerimg[.8]{figs/max-flow/7.png}

\end{itemframe}

%------------------------------------------------------------------------
\begin{itemframe}{مدل‌سازی مسئله شار بیشینه}
\itm
شکل زیر نشان می‌دهد چطور یک شبکه دارای یال‌های پادمتقارن را می‌توان به یک شبکه شار قابل‌قبول تبدیل کرد.
\centerimg[.9]{figs/max-flow/8.png}
\end{itemframe}
%------------------------------------------------------------------------
\begin{itemframe}{مثالی از مسئله شار بیشینه}
\itm
شکل زیر یک شبکه شار را به همراه شار موجود در آن نشان می‌دهد:
%todo numbering here should be fixed
\centerimg[.5]{figs/max-flow/8-1.png}

\itm
برای نشان دادن شبکه شار یک قرارداد شناخته شده وجود دارد که در شکل بالا قابل مشاهده است. به این صورت که ابتدا مقدار شار و سپس ظرفیت هر یال نوشته می‌شود و این دو عدد به وسیله خط مورب از هم جدا می‌شوند.
\end{itemframe}

%------------------------------------------------------------------------
\begin{itemframe}{مثالی از مسئله شار بیشینه}
\itm
همچنین دقت کنید که چگونه قانون پایستگی شار در این شکل رعایت شده. برای مثال ۱۱ واحد به راس
$v_4$
وارد شده و ۴ و ۷ واحد از آن خارج می‌شوند.
\itm
برای درک مسئله شار بیشینه بهتر است سعی کنیم مسئله را صورت دستی حل کنیم.
\itm
برای مثال چطور می‌توان مقدار شار شبکه بالا را افزایش داد؟ به عبارت دیگر باید مقدار شار خروجی از رأس s را بیشتر کنید بدون اینکه قانون پایستگی شار نقض شود. (مقدار شار فعلی ۱۹ واحد است.)
\itm
اگر موفق به افزایش شار شبکه شدید چطور می‌توان دریافت که این شار بیشینه است یا خیر؟

\end{itemframe}

%------------------------------------------------------------------------
\begin{itemframe}{مثالی از مسئله شار بیشینه}
\itm
شکل زیر با تغییر مقدار شار ۳ یال، مقدار شار شبکه را به ۲۳ واحد افزایش داده‌است. این سه یال عبارت اند از:
$(s, v_2), (v_3, v_2), (v_3, t)$
\centerimg[.5]{figs/max-flow/8-2.png}

\itm
نکته جالب توجه اینجاست که با صفر کردن شار یال
$(v_3, v_2)$
موفق به افزایش شار کل شبکه شدیم.
\end{itemframe}