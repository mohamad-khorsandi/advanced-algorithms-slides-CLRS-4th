\begin{itemframe}{مقدمه}
\itm
بسیاری از مسائل محاسباتی کاربردی ان‌پی کامل هستند و با این حال با توجه به اهمیت زیادی که دارند نیاز داریم جوابی برای آنها پیدا کنیم گرچه پیدا کردن جواب دقیق برای اینگونه مسائل در زمان چندجمله‌ای امکان‌پذیر نیست.
\itm
وقتی یک مسئله ان‌پی کامل است، برای حل آن سه راه پیش رو داریم : (۱) اگر ورودی نسبتاً کوچک باشد، می‌توان یک جواب بهینه در زمان نمایی به سرعت برای آن پیدا کرد. (۲) می‌توان یک حالت خاص از مسئله را در زمان چند جمله‌ای حل کرد. (۳) می‌توان یک جواب نزدیک به جواب بهینه در زمان چند جمله‌ای برای آن پیدا کرد. در بسیاری از کاربردها جواب نزدیک به جواب بهینه
\fn{near-optimal solution}
نیز کافی است. به چنین الگوریتم‌هایی که جواب نزدیک به بهینه تولید می‌کنند، الگوریتم‌های تقریبی
\fn{approximation algorithm}
می‌گوییم. برای بسیاری از مسائل ان‌پی کامل می‌توان یک الگوریتم تقریبی در زمان چندجمله‌ای پیدا کرد.
\end{itemframe}

%----------------------------------------------------
\begin{itemframe}{مقدمه}
\itm
فرض کنید بر روی مسئلهٔ بهینه‌سازی کار می‌کنید که در آن هر یک از جواب‌های بالقوه
\fn{potential solution}
دارای یک هزینه است و می‌خواهید یک جواب نزدیک به بهینه پیدا کنید. بسته به نوع مسئله، ممکن است مسئله بیشینه سازی
\fn{maximization}
یا کمینه سازی
\fn{minimization}
باشد. می‌توانید یک جواب بهینه با هزینه حداکثر یا هزینه حداقل پیدا کنید.
\end{itemframe}
%----------------------------------------------------
\begin{itemframe}{مقدمه}
\itm
می‌گوییم یک الگوریتم دارای «ضرب تقریب»
\fn{approximation ratio}
$\rho(n)$
است اگر به ازای هر ورودی با اندازهٔ n ، هزینهٔ
$C$
جواب تولید شده توسط الگوریتم نسبت به هزینهٔ
$C^*$
مربوط به جواب بهینه از مقدار
$\rho(n)$
کمتر باشد. به عبارت دیگر :
$$
\Bigl\{ \frac{C}{C^*}, \frac{C^*}{C} \Bigr\} \leqslant \rho(n)
$$
\itm
اگر یک الگوریتم دارای ضریب تقریب
$\rho(n)$
باشد، به آن الگوریتم تقریبی
$\rho(n)$
می‌گوییم.
\end{itemframe}

%----------------------------------------------------
\begin{itemframe}{مقدمه}
\itm
از الگوریتم‌های تقریبی
$\rho(n)$
هم برای مسائل کمینه سازی و هم برای مسائل بیشینه سازی استفاده می‌شود.
\itm
در یک مسئله بیشینه سازی، داریم
$0 < C \leqslant C^*$
و بنابراین مقدار
$C^*/C$
مقدار بزرگ‌تری است که در آن هزینهٔ جواب بهینه از هزینهٔ جواب تقریبی بزرگ‌تر است.
\itm
در یک مسئله کمینه سازی، داریم
$0 < C^* \leqslant C$
و بنابراین مقدار
$C/C^*$
مقدار بزرگ‌تری است که در آن هزینهٔ جواب تقریبی از هزینهٔ جواب بهینه بزرگ‌تر است.
\itm
با فرض اینکه همهٔ هزینه‌ها مقادیر مثبت هستند، ضریب تقریب در یک الگوریتم تقریبی هیچ‌گاه کمتر از ۱ نیست.
\iffalse
 زیرا اگر داشته باشیم
$C/C^* \leqslant 1$
آنگاه
$C^*/C \geqslant 1$
.
\fi
\itm
بنابراین یک الگوریتم تقریبی با ضریب ۱ جوابی بهینه تولید می‌کند و هر چه ضریب تقریب الگوریتم تقریبی بیشتر باشد، جواب به دست آمده از جواب بهینه دورتر است.
\end{itemframe}

%----------------------------------------------------
\begin{itemframe}{مقدمه}
\itm
برای بسیاری از مسائل، الگوریتم‌های تقریبی چند جمله‌ای با ضریب تقریب کوچک وجود دارد و برای برخی دیگر از مسائل، الگوریتم‌های تقریبی دارای ضریب تقریبی هستند که با مقدار n افزایش پیدا می‌کند.
\itm
در برخی از الگوریتم‌های تقریبی چندجمله‌ای، هرچه الگوریتم در زمان بیشتری اجرا شود، ضریب تقریب بهتری به دست می‌آید. در چنین مسائلی می‌توان با افزایش زمان محاسبات ضریب تقریب را بهبود داد.
\itm
این وضعیت حائز اهمّیت است و یک ناگذاری برای آن وجود دارد که در ادامه معرفی می‌کنیم.
\end{itemframe}

%todo this part is only usefull if an example of Approximation Scheme is mentioned delete it otherwise
\begin{itemframe}{مقدمه}
\itm
طرح تقریب
\fn{Approximation Scheme}
برای یک مسئلهٔ بهینه‌سازی، الگوریتمی تقریبی است که یک نمونه از مسئله را همراه با مقدار
$\varepsilon > 0$
دریافت می‌کند، به طوری که این الگوریتم یک الگوریتم تقریبی
$(1 + \varepsilon)$
 باشد.
\itm
اگر این طرح تقریب برای هر مقدار
$\epsilon > 0$
در زمانی چندجمله‌ای نسبت به اندازهٔ ورودی $n$ اجرا شود، آن را طرح تقریب چندجمله‌ای
\fn{Polynomial-Time Approximation Scheme}
 می‌نامیم.
\itm
زمان اجرای یک طرح تقریب چند جمله‌ایی ممکن است با کاهش
$\epsilon$
 به شدت افزایش یابد. برای مثال زمان اجرای آن می‌تواند چنین چیزی باشد:
$$
O(n^{2/\epsilon})
$$
\end{itemframe}

%----------------------------------------------------
\begin{itemframe}{مقدمه}
\itm
اگر یک طرح تقریب، زمان اجرای چندجمله‌ای نسبت به هر دو پارامتر
$1/\varepsilon$
و $n$ داشته‌باشد، آن را «طرح تقریب چندجمله‌ای کامل»

\fn{Fully Polynomial-Time Approximation Scheme}
می‌نامیم. زمان اجرای یک طرح تقریب چند جمله‌ایی کامل می‌تواند چنین باشد:
$$
O((1/\epsilon)^2n^3)
$$
\itm
در چنین الگوریتمی، هر کاهش با ضریب ثابت در $\epsilon$، باعث افزایش زمان اجرا به‌اندازهٔ یک ضریب ثابت می‌شود.
\end{itemframe}
