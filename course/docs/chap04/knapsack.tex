\newcommand{\itm}{\text{item}}
\begin{frame}{‌کوله‌پشتی ۱-۰}
\begin{itemize}\itemr
\item[-]
یک دزد با یک کوله‌پشتی به یک فروشگاه دستبرد می‌زند. وزنی که کوله‌پشتی او می‌تواند تحمل کند
\m{W}
است. در این فروشگاه تعداد
\m{n}
کالا وجود دارد. هر کالای
\m{\itm_i}
دارای وزن
\m{w_i}
و ارزش
\m{v_i}
است. دزد می‌خواهد از میان این کالاها تعدادی را انتخاب کرده در کوله‌پشتی خود قرار دهد به طوری‌که مجموع وزن کالاهای انتخاب شده از ظرفیت کوله‌پشتی یعنی
\m{W}
بیشتر نباشد و مجموع ارزش کالاهای دزدیده شده حداکثر باشد.
\item[-]
بنابراین دزد می‌خواهد از مجموعهٔ
\m{S = \{\itm_1 ,\itm_2 , \cdots ,\itm_n \} }
یک زیر مجموعه
\m{A}
را انتخاب کند به طور
\m{\sum_{\itm_i \in A} v_i}
بیشترین مقدار ممکن باشد و
\m{(\sum_{\itm_i \in A} w_i) \leqslant W}
باشد.
\item[-]
تعداد همهٔ حالت‌های ممکن تعداد همهٔ زیر مجموعه‌های
\m{S}
است که برابر است با
\m{2^n}
جایی که
\m{n}
تعداد کالاهاست.
\item[-]
در این مسئله دزد یا می‌تواند یک کالا را بردارد یا بگذارد و امکان شکستن کالاها به دو قسمت وجود ندارد. به همین دلیل به آن مسئله کوله پشتی ۱-۰
\fn{1}{\m{0}-\m{1} knapsack}
گفته می‌شود. در مسئلهٔ کوله‌پشتی کسری
\fn{2}{fractional knapsack}
دزد می‌تواند یک کالا را به دو قسمت تقسیم کرده، یک قسمت را در کوله‌پشتی قرار دهد و قسمت دیگر را در فروشگاه بگذارد.
\end{itemize}
\end{frame}


\begin{frame}{‌کوله‌پشتی ۱-۰}
\begin{itemize}\itemr
\item[-]
در گام اول باید اثبات کنیم این مسئله دارای زیر ساختار بهینه است یا به عبارت دیگر قانون بهینگی
\fn{1}{principle of optimality}
برای آن صادق است.
\item[-]
فرض کنید
\m{A}
زیر مجموعه بهینه از
\m{n}
کالا باشد. دو حالت وجود دارد : یا
\m{A}
شامل
\m{\itm_n}
می‌شود یا خیر.
\item[-]
اگر
\m{A}
کالای
\m{\itm_n}
را شامل نشود،
\m{A}
یک زیر مجموعه بهینه برای
\m{n-1}
کالا نیز هست.
\item[-]
اگر
\m{A}
کالای
\m{\itm_n}
را شامل شود، آنگاه مجموع ارزش‌های کالاهای
\m{A}
برابر است با
\m{v_n}
به علاوه بیشترین ارزش ممکن که از
\m{n-1}
کالا برای یک کوله‌پشتی با ظرفیت
\m{W-w_n}
به دست آمده است. این گزاره‌ها را می‌توانیم با برهان خلف اثبات کنیم.
\end{itemize}
\end{frame}


\begin{frame}{‌کوله‌پشتی ۱-۰}
\begin{itemize}\itemr
\item[-]
در گام دوم باید یک رابطه بازگشتی برای محاسبه جواب مسئله براساس جواب زیر مسئله‌ها بنویسیم.
\item[-]
%اگر
%\m{i > 0}
%و
%\m{w > 0}
%باشد آنگاه
فرض کنید
\m{P[i][w]}
بیشترین ارزش به دست آمده از
\m{i}
کالای اول است وقتی که ظرفیت کوله‌پشتی
\m{w}
باشد.
\item[-]
می‌توانیم یک رابطه بازگشتی به صورت زیر برای محاسبه
\m{P[i][w]}
بنویسیم.
\begin{align*}
\m{P[i][w]} = \left\{\begin{array}{lr}
          \m{\max (P[i-1][w] , v_i + P[i-1][w-w_i])}&\m{w_i \leqslant w}~\text{اگر}\\
          \m{P[i-1][w]}&\m{w_i > w}~\text{اگر}
\end{array}\right.
\end{align*}
\item[-]
در این مسئله به دنبال
\m{P[n][W]}
می‌گردیم.
\end{itemize}
\end{frame}


\begin{frame}{‌کوله‌پشتی ۱-۰}
\begin{itemize}\itemr
\item[-]
می‌توانیم جدولی تشکیل دهیم که هر سطر
\m{i}
در آن نشان دهنده این باشد که فقط از
\m{i}
کالای اول استفاده کرده‌ایم و ستون‌های آن همهٔ وزن‌های ممکن از
\m{0}
تا
\m{W}
باشد.
\item[-]
مقادیر
\m{P[0][w]}
و
\m{P[i][0]}
برابر با صفر هستند.
\item[-]
این جدول دارای
\m{nW}
خانه است پس محاسبه این جدول در زمان
\ath{nW}
امکان‌پذیر است.
\end{itemize}
\end{frame}


\begin{frame}{‌کوله‌پشتی ۱-۰}
\begin{itemize}\itemr
\item[-]
توجه کنید که هیچ رابطه‌ای بین
\m{n}
و
\m{W}
وجود ندارد و این الگوریتم می‌تواند از الگوریتمی که همه حالات را بررسی می‌کند بدتر باشد. برای مثال اگر
\m{W = n!}
باشد الگوریتم برنامه‌ریزی پویا از مرتبه
\m{n!}
است درحالی که بررسی همه حالات در زمان
\ath{2^n}
امکان‌پذیر است.
\item[-]
بنابراین تنها در صورتی از برنامه‌ریزی پویا استفاده می‌کنیم که
\m{nW < 2^n}
باشد.
\item[-]
پیچیدگی زمانی
\ath{nW}
گرچه شبیه به پیچیدگی زمانی چندجمله‌ای است، اما در واقع چندجمله‌ای نیست و مقدار 
\m{W}
می‌تواند یک تابع غیرچندجمله‌ای از ورودی مسئله باشد. این پیچیدگی زمانی را پیچیدگی زمانی شبه‌چندجمله‌ای
\fn{1}{pseudo-polynomial time complexity}
می‌نامیم.
\end{itemize}
\end{frame}