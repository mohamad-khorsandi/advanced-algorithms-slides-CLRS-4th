\begin{itemframe}{مسئله ازدواج پایدار}
\item[-]
در بخش قبلی، هدف یافتن یک تطابق بیشینه در یک گراف دوبخشی بدون جهت بود.
اگر بدانیم که گرافی دوبخشی کامل است (یعنی از هر رأس در $L$ به همه رئوس در $R$ یال وجود دارد) آنگاه می‌توان با یک الگوریتم حریصانهٔ ساده یک تطابق بیشینه یافت.
\item[-]
زمانی که یک گراف می‌تواند چندین تطابق مختلف داشته باشد، ممکن است بخواهیم تعیین کنیم که کدام تطابق‌ها مطلوب‌تر هستند. در بخش آینده، به یال‌ها وزن اضافه خواهیم کرد و تطابقی با بیشینه وزن را می‌یابیم.
\item[-]
اما در این بخش، به جای آن، به هر رأس در یک گراف دوبخشی کامل اطلاعاتی اضافه می‌کنیم: رتبه‌بندی رأس‌های سمت مقابل. به‌عبارت‌دیگر، هر رأس در $L$ یک فهرست مرتب از تمام رأس‌های $R$ دارد، و برعکس. برای ساده نگه داشتن بحث، فرض می‌کنیم که هر یک از مجموعه‌های $L$ و $R$ شامل $n$ رأس هستند. هدف در اینجا یافتن یک تطابق بین رأس‌های $L$ و $R$ به‌گونه‌ای «پایدار» است.
\end{itemframe}


\begin{itemframe}{مسئله ازدواج پایدار}
\item[-]
این مسئله به نام مسئله ازدواج پایدار
\fn{1}{stable-marriage problem}
شناخته می‌شود به‌طوری که $L$ مجموعه‌ای از زنان و $R$ مجموعه‌ای از مردان است. هر زن، تمام مردان را  رتبه‌بندی می‌کند و هر مرد نیز تمام زنان را رتبه‌بندی می‌کند.
\item[-]
هدف، یافتن تطابق به گونه‌ای است که اگر زن و مردی با یکدیگر زوج نشده‌اند، آنگاه حداقل یکی از آن‌ها شریک تخصیص‌یافته‌اش را ترجیح دهد.
\item[-]
اگر زنی و مردی با یکدیگر زوج نشده باشند اما هر یک، دیگری را به شریک تخصیص‌یافته‌اش ترجیح دهد، آن‌ها یک «زوج مسدودکننده»
\fn{1}{blocking pair}
را تشکیل می‌دهند. یک زوج مسدودکننده انگیزه دارد که از ازدواج تعیین شده خارج شده و با یکدیگر جفت شوند. در این صورت، این زوج پایداری تطابق را «مسدود» می‌کنند.
\item[-]
یک تطابق پایدار، تطابقی است که هیچ زوج مسدودکننده‌ای ندارد.
\end{itemframe}


\begin{itemframe}{مسئله ازدواج پایدار}
\item[-]
تعداد پاسخ‌ها به یک مسئله ازدواج پایدار الزاماً یکتا نیست. امّا آیا همیشه حداقل یک تطابق پایدار وجود دارد؟ پاسخ این پرسش، مثبت است. بنابراین همیشه یک یا چند تطابق پایدار وجود دارد.

\item[-]
یک الگوریتم ساده به نام الگوریتم گیل-شپلی
\fn{1}{Gale–Shapley}
همواره یک تطابق پایدار پیدا می‌کند. این الگوریتم دو نسخه «زن‌محور»
\fn{2}{woman-oriented}
و «مرد‌محور»
\fn{3}{man-oriented}
دارد. در اینجا نسخه‌ی زن‌محور را بررسی می‌کنیم.

\end{itemframe}


