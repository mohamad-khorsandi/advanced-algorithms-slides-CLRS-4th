
\begin{itemframe}{اثبات الگوریتم هاپکرفت-کارپ}
\item[-]
قبلا روشی بر پایه‌ی یافتن شار بیشینه برای یافتن تطابق بیشینه در گراف‌های دو‌بخشی را دیدیم.
\item[-]
این بخش یک روش کارآمدتر به نام الگوریتم هاپکرفت-کارپ
\fn{1}{Hopcroft-Karp}
 ارائه می‌دهد که در زمان
 $\mathcal{O}(\sqrt{|V|} \cdot |E|)$
 اجرا می‌شود.
\end{itemframe}


\begin{itemframe}{اثبات الگوریتم هاپکرفت-کارپ}
%todo should maximal be translated?
\item[-]
یک تطابق ماکسیمال
\fn{1}{maximal matching}
 تطابقی است که نتوان یال دیگری به آن افزود؛ یعنی برای هر یال $e \in E - M$، مجموعه‌ی $M \cup {e}$ یک تطابق نخواهد بود. یک تطابق بیشینه همواره ماکسیمال است، اما عکس آن لزوماً برقرار نیست.

\item[-]
بسیاری از الگوریتم‌ها برای یافتن تطابق بیشینه (از جمله الگوریتم هاپکرفت-کارپ) با افزایش تدریجی اندازه‌ی تطابق عمل می‌کنند.
\item[-]
فرض کنید تطابق $M$ در گراف بدون‌جهت $G = (V, E)$ داده شده باشد. یک «مسیر متناوب نسبت به $M$»
\fn{1}{M-alternating path}
مسیری ساده است که یال‌های آن به صورت یکی در میان متعلق به $M$ و $E - M$ باشند.
\item[-]
یک «مسیر افزایشی نسبت به $M$»
\fn{2}{M-augmenting path}
مسیر مسیر افزایشی نسبت به $M$، یک مسیر متناوب نسبت به $M$ است که یال‌های ابتدا و انتهای آن در $M$ نباشند.چنین مسیری همیشه دارای تعداد فردی از یال‌ها است.
\end{itemframe}


\begin{itemframe}{اثبات الگوریتم هاپکرفت-کارپ}
\item[-]
در شکل زیر قسمت a یک تطابق با اندازه ۴ را در یک گراف دو بخشی بدون جهت نشان می‌دهد و قسمت b یک مسیر افزایشی نسبت به تطابق شکل قبل که دارای ۵ یال می‌باشد را به تصویر می‌کشد.
\centerimg[.4]{figs/matching/1.png}

\end{itemframe}


\begin{itemframe}{اثبات الگوریتم هاپکرفت-کارپ}
\item[-]
فرض کنید یک تطابق $M$ داشته باشیم و همچنین یک مسیر افزایشی $P$ نسبت به آن وجود داشته باشد. جالب است بندانید اگر یال‌های مشترک $P$ و $M$ را از $M$ حذف کنیم و بقیه یال‌های P را به M اضافه کنیم، نتیجه یک تطابق می‌شود که از M یک یال بیشتر دارد.
\item[-]
این عملیات را می‌توان با عملگر تفاضل متقارن
\fn{1}{symmetric difference}
 که یکی از عملگرهای تعریف شده بر روی مجوعه‌هاست، نشان داد.
\item[-]
تفاضل متقارن دو مجموعه به این صورت تعریف می‌شود؛
$$ X \oplus Y = X - Y \cup Y - X$$
یعنی عناصری که به X یا Y تعلق دارند، اما نه به هر دو.
\end{itemframe}


\begin{itemframe}{اثبات الگوریتم هاپکرفت-کارپ}
\item[-]
پیش‌تر دیدیم که بعد از اعمال این عملیات
$M' = M \oplus P$
مجوعه $M' $ یک تطابق است که اندازه آن یک واحد از $M$ بیشتر است.
\item[-]
برای فهم بهتر این مسئله  فرض کنید مسیر افزایشی $P$ شامل ۳ یال باشد؛
$$ e_1=(v_1, v_2), e_2=(v_2, v_3), e_3(v_3, v_4)$$
می‌دانیم رأس‌های
$v_1$ و
$v_4$
تطابق‌نیافته هستند و بقیه‌ی رأس‌های آن تطابق‌یافته ‌اند.(در غیر این ‌صورت در تطابق $M$ بیش از یک یال به این رأس‌ها
متصل می‌شوند.)

\item[-]
در مسیر افزایشی یال اول و آخر در تطابق قرار ندارند بنابراین
$e_1$ و
$e_3$
در $M$ قرار ندارند، و یال
$e_2$
در $M$ قرار دارد.
\end{itemframe}


\begin{itemframe}{اثبات الگوریتم هاپکرفت-کارپ}
\item[-]
تفاضل متقارن $M' = M \oplus P$، نقش یال‌ها را برعکس می‌کند؛ یال
$e_2$
را در از $M'$ حذف می‌کند و یال‌های
$e_1$ و
$e_3$
در $M'$ قرار می‌دهد.
\item[-]
با حذف یال
$e_2$
از $M$ رأس‌های
$v_2, v_3, v_1, v_4$
هیچ یال متصلی در $M'$  ندارند بنابراین با افزودن
$e_1$ و
$e_3$
مشکلی ایجاد نمی‌شود.
\item[-]
از میان سه یال
$e_1, e_2, e_3$
یک یال در $M$ و دو یال در $M'$ وجود دارد. بنابراین، تطابق $M'$ نسبت به $M$ یک یال بیشتر دارد و هیچ رأس یا یال دیگری در $G$ تحت تأثیر تغییر $M$ به $M'$ قرار نمی‌گیرد.
\item[-]
در نتیجه، $M'$ یک تطابق در $G$ است و
$$|M'| = |M| + 1$$

\end{itemframe}


\begin{itemframe}{اثبات الگوریتم هاپکرفت-کارپ}
\item[-]
لم مهمی در بحث تطابق بیشینه وجود دارد که در آینده از آن در اثبات شرط خاتمه الگوریتم استفاده می‌کنیم.
\item[-]
فرض کنید
$M$ و
$M^*$
دو تطابق در گراف
$G = (V, E)$
 باشند، و گراف
$G' = (V, E')$
را در نظر بگیرید که یال‌های آن حاصل تفاضل متقارن این دو تطابق باشند. این لم بیان می‌کند که  اگر
$|M^*| > |M|$
باشد، آنگاه گراف $G'$ حداقل شامل $|M^*| - |M|$ مسیر متناوب نسبت به $M$ است که رأس‌های این مسیرها از یکدیگر مجزا اند.
\item[-]
\textbf{اثبات:}
هر رأس گراف $G'$ تنها می‌تواند درجه ۰ ، ۱ یا ۲ داشته باشد زیرا حداکثر یک یال متصل به هر رأس در یک تطابق وجود دارد. بنابراین اگر رأسی با درجه ۲ در  $G'$ وجود داشته باشد، قطعاً یکی از یال‌های متصل به آن متعلق به $M^*$ و دیگری متعلق به $M$ است.
%connected components should be discussed in appendix

\end{itemframe}


\begin{itemframe}{اثبات الگوریتم هاپکرفت-کارپ}
\item[-]
حال بیایید رأس‌های این گراف را بررسی کنیم؛ رأس‌هایی که درجه ۰ دارند به هیچ جای دیگر از گراف متصل نیستند. رأس‌های با درجه ۱ و ۲ تنها می‌توانند یک مسیر ساده یا یک دور ساده را تشکیل دهند. همچنین این مسیرها و دور‌ها نمی‌توانند رأس مشترکی داشته باشند زیرا در غیر این صورت حداقل یک رأس باید از درجه ۳ یا بیشتر باشد.

\item[-]
همچنین دیدیم که هر رأس با درجه ۲ به یک یال از $M^*$ و یک یال از $M$ متصل است و در دور همه رأس‌ها از درجه ۲ هستند. پس دورهای گراف $G'$ نمی‌توانند تعداد فرد یال داشته باشند (چرا؟). همچنین نیمی از یال‌های هر دور به $M^*$ و نیمی دیگر به $M$ تعلق دارند.

\item[-]
به علاوه یال‌های هر مسیر هم به طور متناوب از دو تطابق است زیرا رأس‌های درونی یک مسیر نیز درجه ۲ هستند.

\end{itemframe}


\begin{itemframe}{اثبات الگوریتم هاپکرفت-کارپ}
\item[-]
بنابراین، هر مؤلفه‌ی همبند در $G'$ یا یک رأس منفرد، یا یک دور ساده با طول زوج که یال‌های آن به‌صورت متناوب از $M$ و $M^*$ هستند، و یا یک مسیر ساده با یال‌های متناوب از $M$ و $M^*$ است.

\item[-]
فرض کردیم  $M^*$ از $M$ تعداد یال بیشتری داشته باشد.
پس مجموعه‌ی یال‌های $E'$ باید $|M^*| - |M|$ یال بیشتر از $M^*$ نسبت به $M$ داشته باشد (چرا؟).
\item[-]
دیدیم که تنها مؤلفه‌های همبندی موجود در $G'$ که دارای یال ‌هستند دور و مسیر هستند. پس تعداد یال‌های متعلق به $M^*$ باید در این مؤلفه‌ها بیشتر باشند. امّا در هر دور تعداد یال‌های برابری از هر دو تطابق استفاده شده. بنابراین، این مسیرهای ساده در $G'$ هستند که عامل این اختلاف در تعداد یال‌ها از $M^*$ نسبت به $M$ می‌باشند.

\end{itemframe}


\begin{itemframe}{اثبات الگوریتم هاپکرفت-کارپ}
\item[-]
مسیرهایی که تعداد یال‌های برابری از هر تطابق داشته باشند نیز تأثیری ندارند. امّا مسیری که تعداد متفاوتی از یال‌های $M$ و $M^*$ را دارا باشد دو نوع دارد؛
\item[۱]
یا با یال‌هایی از $M$ شروع و پایان می‌یابد (و یک یال بیشتر از $M$ نسبت به $M^*$ دارد).
\item[۲]
یا با یال‌هایی از $M^*$ شروع و پایان می‌یابد (و یک یال بیشتر از $M^*$ نسبت به $M$ دارد).
\item[-]
از آن‌جایی که $E'$ شامل $|M^*| - |M|$ یال بیشتر از $M^*$ است، باید حداقل $|M^*| - |M|$ مسیر از نوع دوم وجود داشته باشد، توجه کنید این مسیرها تمام ویژگی‌های یک مسیر افزایشی نسبت به $M$ را دارند و دیدیم تمام این مسیرها از هم مجزا هستند. بنابراین لم بالا ثابت می‌شود.
\end{itemframe}


%\iffalse
\begin{itemframe}{اثبات الگوریتم هاپکرفت-کارپ}
\item[-]
حال اگر یک الگوریتم با افزودن تدریجی یال‌ها تطابق بیشینه را بیابد، چگونه تشخیص می‌دهد که باید متوقف شود؟ ثابت خواهیم کرد «زمانی که دیگر هیچ مسیر افزایشی وجود نداشته باشد» الگوریتم باید متوقف شود. به طور دقیق‌تر:
\item[-]
$M$ یک تطابق بیشینه است (p)
$\iff$
 هیچ مسیر افزایشی نسبت به $M$ در گراف وجود نداشته باشد (q)

%todo explain contrapositive in logics perhaps in appendix
\item[-]
\textbf{اثبات:}
باید ثابت کنیم
$(p \implies q) $
 (جهت اول) و
$ (q \implies p)$
 (جهت دوم) برقرار هستند. به جای این کار می‌توانیم عکس و نقیض
\fn{1}{contrapositive}
 هر دو گزاره را اثباب کنیم.

\end{itemframe}


\begin{itemframe}{اثبات الگوریتم هاپکرفت-کارپ}
\item[-]
عکس و نقیض جهت اول
($~q\implies~p$)
: اگر یک مسیر افزایشی نسبت به تطابقی وجود داشته باشد، آنگاه آن تطابق بیشینه نیست.
\item[-]
\textbf{اثبات جهت اول:}
اگر یک مسیر افزایشی نسبت به $M$، به نام $P$ وجود داشته باشد، آنگاه تطابق $M \oplus P$ یک یال بیشتر از $M$ دارد، بنابراین $M$ نمی‌تواند تطابق بیشینه باشد.

\item[-]
عکس و نقیض جهت دوم
($‍~p \implies ~q$)
: اگر تطابقی بیشینه نباشد، آنگاه حداقل یک مسیر افزایشی نسبت به آن در گراف وجود دارد.
\item[-]
\textbf{اثبات جهت دوم:}
اگر $M^*$ یک تطابق بیشینه باشد به‌طوری که $|M^*| > |M|$. آنگاه طبق لم قبلی، گراف $G$ حداقل شامل
$|M^*| - |M| > 0$
 مسیر افزایشی با رأس‌های مجزا نسبت به $M$ است.
\end{itemframe}


\begin{itemframe}{الگوریتم هاپکرفت-کارپ}
\item[-]
تا به اینجا می‌توانیم الگوریتمی برای یافتن یک تطابق بیشینه طراحی کنیم که در زمان $O(VE)$ اجرا می‌شود.
\item[-]
به این صورت که با یک تطابق تهی $M$ آغاز می‌کنیم. سپس به‌صورت تکراری، یکی از گونه‌های جست‌وجوی اول سطح یا جست‌وجوی اول عمق را از یک رأس تطابق‌نیافته آغاز می‌کنیم تا مسیری متناوب بیابیم که به یک رأس دیگر که تطابق‌نیافته است ختم شود.
\item[-]
از مسیر افزایشی حاصل‌شده برای افزودن یک یال به $M$ استفاده می‌کنیم، تا اندازه‌ی آن یک واحد افزایش یابد.
\end{itemframe}


%\begin{itemframe}{اثبات الگوریتم هاپکرفت-کارپ}
%\fi