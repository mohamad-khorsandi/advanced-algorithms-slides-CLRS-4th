
\begin{itemframe}{مقدمه}
\item[-]
قبلا روشی بر پایه‌ی یافتن شار بیشینه برای یافتن تطابق بیشینه در گراف‌های دو‌بخشی را دیدیم.
\item[-]
این بخش یک روش کارآمدتر به نام الگوریتم هاپکرفت-کارپ
\fn{1}{Hopcroft-Karp}
 ارائه می‌دهد که در زمان
 $\mathcal{O}(\sqrt{|V|} \cdot |E|)$
 اجرا می‌شود.
\end{itemframe}


\begin{itemframe}{الگوریتم هاپکرفت-کارپ}
%todo should maximal be translated?
\item[-]
یک تطابق ماکسیمال
\fn{1}{maximal matching}
 تطابقی است که نتوان یال دیگری به آن افزود؛ یعنی برای هر یال $e \in E - M$، مجموعه‌ی $M \cup {e}$ یک تطابق نخواهد بود. یک تطابق بیشینه همواره ماکسیمال است، اما عکس آن لزوماً برقرار نیست.

\item[-]
بسیاری از الگوریتم‌ها برای یافتن تطابق بیشینه (از جمله الگوریتم هاپکرفت-کارپ) با افزایش تدریجی اندازه‌ی تطابق عمل می‌کنند.
\item[-]
فرض کنید تطابق $M$ در گراف بدون‌جهت $G = (V, E)$ داده شده باشد. یک «مسیر متناوب نسبت به $M$»
\fn{1}{M-alternating path}
مسیری ساده است که یال‌های آن به صورت یکی در میان متعلق به $M$ و $E \setminus M$ باشند.
\item[-]
یک «مسیر افزایشی نسبت به $M$»
\fn{2}{M-augmenting path}
مسیر مسیر افزایشی نسبت به $M$، یک مسیر متناوب نسبت به $M$ است که یال‌های ابتدا و انتهای آن در $M$ نباشند.چنین مسیری همیشه دارای تعداد فردی از یال‌ها است.
\end{itemframe}


\begin{itemframe}{الگوریتم هاپکرفت-کارپ}
\item[-]
در شکل زیر قسمت a یک تطابق با اندازه ۴ را در یک گراف دو بخشی بدون جهت نشان می‌دهد و قسمت b یک مسیر افزایشی نسبت به تطابق شکل قبل که دارای ۵ یال می‌باشد را به تصویر می‌کشد.
\centerimg[.4]{figs/matching/1.png}

\end{itemframe}


\begin{itemframe}{الگوریتم هاپکرفت-کارپ}


\end{itemframe}