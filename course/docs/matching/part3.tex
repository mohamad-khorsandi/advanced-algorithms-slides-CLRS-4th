\begin{itemframe}{الگوریتم مجارستانی}
\item[-]

بیایید بار دیگر اطلاعاتی را به یک گراف دو بخشی \textbf{کامل} اضافه کنیم است.
این بار به هر یال یک وزن اختصاص می‌دهیم.
دوباره فرض می‌کنیم مجموعه‌های $L$ و $R$ هر یک شامل $n$ رأس هستند، بنابراین گراف دارای
$n^2$
یال خواهد بود.
\item[-]
برای $l \in L$ و $r \in R$، وزن یال $(l, r)$ را با $w(l, r)$ نمایش می‌دهیم که نشان‌دهنده‌ی میزان سودی است که از تطبیق رأس $l$ با رأس $r$ به دست می‌آید.
\item[-]
یک تطابق کامل
\fn{1}{perfect matching}
 تطابقی است که تحت آن همه رأس‌ها تطابق‌یافته باشند.

\end{itemframe}


\begin{itemframe}{الگوریتم مجارستانی}
\item[-]
در این مسئله هدف یافتن یک تطابق کامل مانند
$M^*$
 است به‌طوری‌که مجموع وزن یال‌های آن، در میان همه تطابق‌های کامل، بیشینه باشد.
به این مسئله،‌ مسئله تخصیص
\fn{1}{assignment problem}
 گفته می‌شود.
\item[-]
بررسی تمام تطابق‌های کامل از مرتبه
$ \Omega (n!)$
است. اما الگوریتمی به نام الگوریتم مجارستانی
\fn{2}{Hungarian algorithm}
 این مسئله را بسیار سریع‌تر حل می‌کند.
\item[-]
این الگوریتم ا زمرتبه زمانی
$O(n^4)$
 است.
(البته می‌توان زمان اجرای آن را به
$O(n^3)$
 کاهش داد.)
\end{itemframe}


\begin{itemframe}{الگوریتم مجارستانی}
\item[-]
الگوریتم مجارستانی به جای کار با گراف دوبخشی کامل $G$، با زیرگرافی از آن به نام زیرگراف تساوی
\fn{1}{equality subgraph}
کار می‌کند.
این زیرگراف به‌صورت پویا تغییر می‌کند و هر تطابق کامل در زیرگراف تساوی، یک راه‌حل بهینه برای مسئله تخصیص نیز هست.
\item[-]
زیرگراف تساوی با تخصیص یک عدد به هر رأس ساخته می‌شود. به این عدد برچسب رأس گفته می‌شود و با h نشان داده می‌شود.
\item[-]
می‌گوییم $h$ یک برچسب‌گذاری مجاز
\fn{1}{feasible labeling}
است اگر برای همه $l \in L$ و $r \in R$ رابطه زیر برقرار باشد:
$$l.h + r.h \geq w(l, r)$$
\end{itemframe}


\begin{itemframe}{الگوریتم مجارستانی}
\item[-]
یک برچسب‌گذاری مجاز همواره وجود دارد. به عنوان مثال، می‌توان برچسب‌گذاری پیش‌فرض زیر را در نظر گرفت:
\begin{align*}
&l.h = \max \{ w(l, r) \mid r \in R \} &\forall l \in L\\
&r.h = 0 &\forall r \in R
\end{align*}
\item[-]
فرض کنید h یک برچسب‌گذاری مجاز باشد، زیرگراف تساوی $G_h = (V, E_h)$ دارای همان رأس‌هاست ولی فقط شامل یال‌هایی است که در آن‌ها تساوی زیر برقرار باشد:
$$
E_h = \{ (l, r) \in E \mid l.h + r.h = w(l, r) \}
$$
\end{itemframe}


\begin{itemframe}{الگوریتم مجارستانی}
\item[-]
قضیه‌ی زیر، رابطه‌ی بین تطابق کامل در زیرگراف تساوی و راه‌حل بهینه برای مسئله تخصیص را بیان می‌کند:
\item[-]
فرض کنید $h$ یک برچسب‌گذاری مجاز برای $G$ و $G_h$ زیرگراف تساوی متناظر باشد. اگر $M^*$ یک تطابق کامل روی $G_h$ باشد، آنگاه یک راه‌حل بهینه برای مسئله تخصیص روی $G$ نیز هست.
\item[-]
\textbf{اثبات:}
سودمندی $M^*$ برابر است با:
$$
w(M^\ast) = \sum_{(l, r) \in M^\ast} w(l, r)
$$

\end{itemframe}


\begin{itemframe}{الگوریتم مجارستانی}
\item[-]
گفتیم در زیرگراف تساوی وزن یال برابر است با جمع برچسب‌های دو رأس آن پس:
$$
w(M^\ast) = \sum_{(l, r) \in M^\ast} w(l, r)
$$
\item[-]
از آنجا که $G_h$ و $G$ مجموعه رأس‌های یکسانی دارند $M^\ast$ یک تطابق کامل در $G$ نیز هست. در هر تطابق کامل، هر رأس دقیقاً در یک یال قرار دارد، بنابراین:
$$
w(M^\ast) = \sum_{l \in L} l.h + \sum_{r \in R} r.h
$$
\end{itemframe}


\begin{itemframe}{الگوریتم مجارستانی}
\item[-]
حال باید ثابت کنیم سودمندی $M^\ast$ از هر تطابق کامل دلخواهی که روی گراف اصلی در نظر بگیریم بیشتر است. اگر $M$  را یک تطابق کامل دلخواه در نظر بگیریم سودمندی آن برابر است با:
$$
w(M) = \sum_{(l, r) \in M} w(l, r)
$$
\item[-]
در یک برچسب گذاری مجاز مجموع دو رأس یک یال از وزن آن بیشتر است بنابراین:
\begin{align*}
w(M) = \sum_{(l, r) \in M} w(l, r) \leq \sum_{(l, r) \in M} (l.h + r.h) = \sum_{l \in L} l.h + \sum_{r \in R} r.h
\end{align*}
\end{itemframe}


\begin{itemframe}{الگوریتم مجارستانی}
\item[-]
دیدم که سودمندی $M^\ast$ برابر با
$$
\sum_{l \in L} l.h + \sum_{r \in R} r.h
$$
و سودمندی $M$ کمتر از این مقدار است پس سودمندی $M^\ast$ از هر تطابق کامل دلخواهی در $G‌$ بیشتر است.
\item[-]
پس حالا هدف یافتن یک تطابق کامل در یک زیرگراف تساوی است. اما کدام زیرگراف تساوی؟ در اثبات بالا هیچ محدودیتی بر زیر گراف تساوی وضع نشد. بنابراین تنها کافیست یک تطابق کامل در یک زیر گراف تساوی بیابیم.

\item[-]
الگوریتم مجارستانی هم تطابق و هم برچسب‌گذاری را به‌صورت مکرر تغییر می‌دهد تا به این هدف برسد.
\item[-]
این الگوریتم با یک برچسب‌گذاری مجاز اولیه و یک تطابق دلخواه در زیرگراف تساوی شروع می‌کند. سپس، به‌صورت تکراری، یک مسیر افزایشی نسبت به  M در $G_h$ پیدا می‌کند و تطابق را به $M \oplus P$ به‌روزرسانی می‌کند. به این ترتیب اندازه تطابق افزایش می‌یابد.
\item[-]
تا زمانی که یک مسیر افزایشی نسبت به M در یک زیرگراف برابری وجود داشته باشد، اندازه تطابق می‌تواند افزایش یابد تا یک تطابق کامل حاصل شود.
\end{itemframe}


\begin{itemframe}{الگوریتم مجارستانی}
\item[-]
حال چهار سوال مطرح است:

\end{itemframe}