\begin{itemframe}{الگوریتم مجارستانی}
\item[-]

بیایید بار دیگر اطلاعاتی را به یک گراف دو بخشی \textbf{کامل} اضافه کنیم است.
این بار به هر یال یک وزن اختصاص می‌دهیم.
دوباره فرض می‌کنیم مجموعه‌های $L$ و $R$ هر یک شامل $n$ رأس هستند، بنابراین گراف دارای
$n^2$
یال خواهد بود.
\item[-]
برای $l \in L$ و $r \in R$، وزن یال $(l, r)$ را با $w(l, r)$ نمایش می‌دهیم که نشان‌دهنده‌ی میزان سودی است که از تطبیق رأس $l$ با رأس $r$ به دست می‌آید.
\item[-]
یک تطابق کامل
\fn{1}{perfect matching}
 تطابقی است که تحت آن همه رأس‌ها تطابق‌یافته باشند.

\end{itemframe}


\begin{itemframe}{الگوریتم مجارستانی}
\item[-]
در این مسئله هدف یافتن یک تطابق کامل مانند
$M^*$
 است به‌طوری‌که مجموع وزن یال‌های آن، در میان همه تطابق‌های کامل، بیشینه باشد.
به این مسئله،‌ مسئله تخصیص
\fn{1}{assignment problem}
 گفته می‌شود.
\item[-]
بررسی تمام تطابق‌های کامل از مرتبه
$ \Omega (n!)$
است. اما الگوریتمی به نام الگوریتم مجارستانی
\fn{2}{Hungarian algorithm}
 این مسئله را بسیار سریع‌تر حل می‌کند.
\item[-]
این الگوریتم ا زمرتبه زمانی
$O(n^4)$
 است.
(البته می‌توان زمان اجرای آن را به
$O(n^3)$
 کاهش داد.)
\end{itemframe}


\begin{itemframe}{الگوریتم مجارستانی}
\item[-]
الگوریتم مجارستانی به جای کار با گراف دوبخشی کامل $G$، با زیرگرافی از آن به نام زیرگراف تساوی
\fn{1}{equality subgraph}
کار می‌کند.
این زیرگراف به‌صورت پویا تغییر می‌کند و هر تطابق کامل در زیرگراف تساوی، یک راه‌حل بهینه برای مسئله تخصیص نیز هست.
\item[-]
زیرگراف تساوی با تخصیص یک عدد به هر رأس ساخته می‌شود. به این عدد برچسب رأس گفته می‌شود و با h نشان داده می‌شود.
\item[-]
می‌گوییم $h$ یک برچسب‌گذاری مجاز
\fn{1}{feasible labeling}
است اگر برای همه $l \in L$ و $r \in R$ رابطه زیر برقرار باشد:
$$l.h + r.h \geq w(l, r)$$
\end{itemframe}


\begin{itemframe}{الگوریتم مجارستانی}
\item[-]
یک برچسب‌گذاری مجاز همواره وجود دارد. به عنوان مثال، می‌توان برچسب‌گذاری پیش‌فرض زیر را در نظر گرفت:
\begin{align*}
&l.h = \max \{ w(l, r) \mid r \in R \} &\forall l \in L\\
&r.h = 0 &\forall r \in R
\end{align*}
\item[-]
فرض کنید h یک برچسب‌گذاری مجاز باشد، زیرگراف تساوی $G_h = (V, E_h)$ دارای همان رأس‌هاست ولی فقط شامل یال‌هایی است که در آن‌ها تساوی زیر برقرار باشد:
$$
E_h = \{ (l, r) \in E \mid l.h + r.h = w(l, r) \}
$$
\end{itemframe}


\begin{itemframe}{الگوریتم مجارستانی}
\item[-]
قضیه‌ی زیر، رابطه‌ی بین تطابق کامل در زیرگراف تساوی و راه‌حل بهینه برای مسئله تخصیص را بیان می‌کند:
\item[-]
فرض کنید $h$ یک برچسب‌گذاری مجاز برای $G$ و $G_h$ زیرگراف تساوی متناظر باشد. اگر $M^*$ یک تطابق کامل روی $G_h$ باشد، آنگاه یک راه‌حل بهینه برای مسئله تخصیص روی $G$ نیز هست.
\item[-]
\textbf{اثبات:}
سودمندی $M^*$ برابر است با:
$$
w(M^\ast) = \sum_{(l, r) \in M^\ast} w(l, r)
$$

\end{itemframe}


\begin{itemframe}{الگوریتم مجارستانی}
\item[-]
گفتیم در زیرگراف تساوی وزن یال برابر است با جمع برچسب‌های دو رأس آن پس:
$$
w(M^\ast) = \sum_{(l, r) \in M^\ast} w(l, r)
$$
\item[-]
از آنجا که $G_h$ و $G$ مجموعه رأس‌های یکسانی دارند $M^\ast$ یک تطابق کامل در $G$ نیز هست. در هر تطابق کامل، هر رأس دقیقاً در یک یال قرار دارد، بنابراین:
$$
w(M^\ast) = \sum_{l \in L} l.h + \sum_{r \in R} r.h
$$
\end{itemframe}


\begin{itemframe}{الگوریتم مجارستانی}
\item[-]
حال باید ثابت کنیم سودمندی $M^\ast$ از هر تطابق کامل دلخواهی که روی گراف اصلی در نظر بگیریم بیشتر است. اگر $M$  را یک تطابق کامل دلخواه در نظر بگیریم سودمندی آن برابر است با:
$$
w(M) = \sum_{(l, r) \in M} w(l, r)
$$
\item[-]
در یک برچسب گذاری مجاز مجموع دو رأس یک یال از وزن آن بیشتر است بنابراین:
\begin{align*}
w(M) = \sum_{(l, r) \in M} w(l, r) \leq \sum_{(l, r) \in M} (l.h + r.h) = \sum_{l \in L} l.h + \sum_{r \in R} r.h
\end{align*}
\end{itemframe}


\begin{itemframe}{الگوریتم مجارستانی}
\item[-]
دیدم که سودمندی $M^\ast$ برابر با
$$
\sum_{l \in L} l.h + \sum_{r \in R} r.h
$$
و سودمندی $M$ کمتر از این مقدار است پس سودمندی $M^\ast$ از هر تطابق کامل دلخواهی در $G‌$ بیشتر است.
\item[-]
پس حالا هدف یافتن یک تطابق کامل در یک زیرگراف تساوی است. اما کدام زیرگراف تساوی؟ در اثبات بالا هیچ محدودیتی بر زیر گراف تساوی وضع نشد. بنابراین تنها کافیست یک تطابق کامل در یک زیر گراف تساوی بیابیم.

\item[-]
الگوریتم مجارستانی هم تطابق و هم برچسب‌گذاری را به‌صورت مکرر تغییر می‌دهد تا به این هدف برسد.
\item[-]
این الگوریتم با یک برچسب‌گذاری مجاز اولیه و یک تطابق دلخواه در زیرگراف تساوی شروع می‌کند. سپس، به‌صورت تکراری، یک مسیر افزایشی نسبت به  M در $G_h$ پیدا می‌کند و تطابق را به $M \oplus P$ به‌روزرسانی می‌کند. به این ترتیب اندازه تطابق افزایش می‌یابد.
\item[-]
تا زمانی که یک مسیر افزایشی نسبت به M در یک زیرگراف برابری وجود داشته باشد، اندازه تطابق می‌تواند افزایش یابد تا یک تطابق کامل حاصل شود.
\end{itemframe}


\begin{itemframe}{الگوریتم مجارستانی}
\item[-]
حال چهار سوال مطرح است:
\item[-]
الگوریتم باید با کدام برچسب‌گذاری شروع کند؟
\item[-]

الگوریتم باید با چه تطابق اولیه‌ای شروع کند؟
\item[-]

اگر مسیر افزایشی در $G_h$ وجود داشته باشد، چگونه می‌توان آن را یافت؟
\item[-]
اگر جستجو برای مسیر افزایشی شکست بخورد چه باید کرد؟
\item[-]
در ادامه این بخش به ترتیب به پاسخ دادن به این سوالات می‌پردازیم.

\end{itemframe}


\begin{itemframe}{الگوریتم مجارستانی}
\decLineSpace[0mm]
\item[-]
می‌خواهیم سوالات را با استفاده از مثال زیر پاسخ بدهیم:
\centerimg{figs/matching/7.png}
\item[-]
وزن یال‌ها در ماتریسی که در بخش a نشان داده شده و برچسب‌ها در سمت چپ و بالای ماتریس مشخص شده‌اند.
\item[-]
مدخل‌های قرمز رنگ برای یال‌هایی‌اند که وزن آنها با جمع برچسب گذاری دو سر آن برابر است.
یعنی این یال‌ها در زیرگراف تساوی در بخش b وجود دارند.
\end{itemframe}


\begin{itemframe}{الگوریتم مجارستانی}
\item[-]
روش‌های مختلفی برای پیاده‌سازی یک الگوریتم حریصانه برای یافتن یک تطابق بیشینه‌ی دو بخشی وجود دارد. الگوریتم زیر یکی از این روش‌ها است. در شکل بالا قسمت b، یال‌هایی که به رنگ آبی برجسته شده‌اند، تطابق اولیه‌ی حریصانه در $G_h$ را نشان می‌دهند.
\begin{algorithm}[H]\alglr
  \caption{\textsc{GREEDY-BIPARTITE-MATCHING}$(G)$}
  \begin{algorithmic}[1]
    \State $M \gets \emptyset$
    \For{each vertex $l \in L$}
      \If{$l$ has an unmatched neighbor in $R$}
        \State choose any such unmatched neighbor $r \in R$
        \State $M \gets M \cup {(l, r)}$
      \EndIf
    \EndFor
    \State \Return $M$
  \end{algorithmic}
\end{algorithm}
\item[-]
این الگوریتم تطابقی را بازمی‌گرداند که اندازه‌ی آن حداقل نیمی از اندازه‌ی تطابق بیشینه است.
\end{itemframe}


\begin{itemframe}{الگوریتم مجارستانی}
\item[-]
برای یافتن یک مسیر افزایشی در زیرگراف تساوی، الگوریتم مجارستانی ابتدا زیرگراف تساوی جهت‌دار $G_{M,h}$ را از $G_h$ می‌سازد؛ دقیقاً همان‌گونه که الگوریتم هاپکرفت-کارپ، گراف $G_M$ را از $G$ می‌سازد.
\item[-]
مانند الگوریتم هاپکرفت-کارپ، می‌توانید مسیر افزایشی را به‌گونه‌ای تصور کنید که از یک رأس تطابق نیافته در $L$ آغاز می‌شود و به یک رأس تطابق نیافته در $R$ ختم می‌شود؛ پس یال‌های استفاده‌شده از $L$ به $R$ عضو تطابق هستند و یال‌هایی که از $R$ به $L$ طی می‌شوند، عضو تطابق نیستند.
\item[-]
از آنجا که هر مسیر افزایشی نسبت به M در $G_{M,h}$، یک مسیر افزایشی در $G_h$ نیز محسوب می‌شود، کافی است مسیرهای افزایشی را در $G_{M,h}$ بیابیم.
\item[-]
شکل قبلی قسمت $G_{M,h}$ را نشان می‌دهد که متناظر $G_h$ و تطابق $M$ در بخش b از همان شکل است.
\end{itemframe}


\begin{itemframe}{الگوریتم مجارستانی}
\decLineSpace[0mm]
\item[-]
با در اختیار داشتن $G_{M,h}$ الگوریتم مجارستانی به دنبال یک مسیر افزایشی نسبت به M از هر رأس تطابق نیافته در $L$ به هر رأس تطابق نیافته در $R$ می‌گردد.
\item[-]
در اینجا از جست‌وجوی اول سطح برای این منظور استفاده می‌کنیم، که از تمام رأس‌های تطابق نیافته در $L$ شروع می‌شود. درست مانند الگوریتم هاپکرفت-کارپ با این تفاوت که بعد از یافتن اولین رأس تطابق نیافته در $R$ متوقف می‌شود؛ شکل زیر این ایده را نشان می‌دهد.
\centerimg[.3]{figs/matching/8.png}

\end{itemframe}


\begin{itemframe}{الگوریتم مجارستانی}
\item[-]
ابتدا تمام رأس‌های تطابق در $L$ را در صف قرار می‌دهیم، نه فقط با یک رأس مبدأ.
برخلاف گراف $H$ در الگوریتم هاپکرفت-کارپ، در اینجا هر رأس فقط به یک رأس قبلی نیاز دارد، بنابراین جست‌وجوی اول سطح، یک جنگل اول سطح
\fn{1}{breadth-first forest}
به نام $F = (V_F, E_F)$ ایجاد می‌کند، به طوری که هر رأس تطابق نیافته در $L$ یک ریشه در $F$ است.
\item[-]
در شکل بالا قسمت g، جست‌وجوی اول سطح مسیر زیر را یافته است:
$$
\langle (l_4, r_2), (r_2, l_1), (l_1, r_3), (r_3, l_6), (l_6, r_5) \rangle
$$

\end{itemframe}


\begin{itemframe}{الگوریتم مجارستانی}
\item[-]
شکل زیر قسمت a تطابق جدید را نشان می‌دهد که از طریق گرفتن تفاضل متقارن بین تطابق و این مسیر افزایشی به‌دست آمده است.
\centerimg{figs/matching/9.png}
\end{itemframe}
