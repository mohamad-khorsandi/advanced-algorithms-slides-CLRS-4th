\begin{itemframe}{مقدمه}
\itm
بسیاری از مسائل دنیای واقعی را می‌توان به صورت یافتن یک تطابق در یک گراف دوبخشی بدون‌جهت مدل‌سازی کرد.
\itm
برای مثال، فرض کنید شما می‌خواهید یک نفر را استخدام کنید و چندین داوطلب برای مصاحبه وجود دارند.
هر داوطلبان تنها در بازه‌های مشخصی در دسترس است. چگونه می‌توانید برنامه‌ی مصاحبه‌ها را طوری تنظیم کنید که در هر بازه‌ی زمانی، حداکثر یک داوطلب زمان‌بندی شده باشد و در عین حال، بیشترین تعداد ممکن از داوطلبان را مصاحبه کنید؟
\itm
 این مسئله را می‌توان به صورت یک مسئله‌ی تطابق در یک گراف دو‌بخشی مدل کرد، که در آن هر رأس نماینده‌ی یک داوطلب یا یک بازه‌ی زمانی است، و اگر داوطلب در آن زمان در دسترس باشد بین داوطلب و بازه‌ی زمانی یالی وجود دارد.
\itm
اگر یالی در تطابق قرار گیرد، به این معناست که آن داوطلب در آن بازه‌ی زمانی برنامه‌ریزی شده است. هدف شما یافتن یک تطابق بیشینه است: تطابقی با بیشترین تعداد یال ممکن.
\end{itemframe}


\begin{itemframe}{مقدمه}
\itm
الگوریتم‌های این فصل تطابق را در گراف‌های دو‌بخشی می‌یابند. هر چند در بخش شار بیشینه مسئله تطابق بیشینه معرفی شد، بهترا است مجدداً مرور مختصری از مسئله و نامگذاری‌ها انجام دهیم.
\itm
ورودی یک گراف بدون‌جهت $G = (V, E)$ است که در آن $V = L \cup R$، به‌طوری‌که مجموعه‌های $L$ و $R$ از هم جدا (ناهم‌پوشان) هستند، و هر یال در $E$ به یک رأس در $L$ و یک رأس در $R$ متصل است. بنابراین، یک تطابق در این گراف، رأس‌های $L$ را به رأس‌های $R$ متصل می‌کند.
\itm
در برخی کاربردها، مجموعه‌های $L$ و $R$ دارای تعداد رأس‌های برابر هستند و در برخی دیگر، الزامی به تساوی اندازه‌ی آن‌ها نیست.
\itm
می‌گوییم یک رأس $v \in V$ توسط تطابق $M$ تطابق‌یافته
\fn{matched}
 است اگر یالی در $M$ وجود داشته باشد که به $v$ متصل باشد؛ در غیر این صورت، $v$ تطابق‌نیافته
\fn{unmatched}
 است.
\end{itemframe}


\begin{itemframe}{مقدمه}
\itm
یک گراف بدون‌جهت لزوماً نیازی به دو‌بخشی بودن ندارد تا بتوان در آن از مفهوم تطابق استفاده کرد. تطابق در گراف‌های بدون‌جهت عمومی نیز کاربرد دارد.
\itm
مسائل تطابق بیشینه و تطابق با وزن بیشینه در گراف‌های عمومی را می‌توان با الگوریتم‌های زمان چندجمله‌ای حل کرد که زمان اجرای آن‌ها مشابه تطابق در گراف‌های دو‌بخشی است، اما این الگوریتم‌ها به‌مراتب پیچیده‌ترند.
\itm
 گرچه مفهوم تطابق در گراف‌های عمومی نیز کاربرد دارد، اما ما تنها به گراف‌های دو‌بخشی می‌پردازیم.
\end{itemframe}