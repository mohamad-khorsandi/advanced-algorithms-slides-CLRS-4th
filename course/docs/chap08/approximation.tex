
\begin{frame}{‌الگوریتم‌های تقریبی}
\begin{itemize}\itemr
\item[-]
بسیاری از مسائل محاسباتی کاربردی ان‌پی کامل هستند و با این حال با توجه به اهمیت زیادی که دارند نیاز داریم جوابی برای آنها پیدا کنیم گرچه پیدا کردن جواب دقیق برای اینگونه مسائل در زمان چندجمله‌ای امکان‌پذیر نیست.
\item[-]
وقتی یک مسئله ان‌پی کامل است، برای حل آن سه راه پیش رو داریم : (۱) اگر ورودی نسبتاً کوچک باشد، می‌توان یک جواب بهینه در زمان نمایی به سرعت برای آن پیدا کرد. (۲) می‌توان یک حالت خاص از مسئله را در زمان چند جمله‌ای حل کرد. (۳) می‌توان یک جواب نزدیک به جواب بهینه در زمان چند جمله‌ای برای آن پیدا کرد. در بسیاری از کاربردها جواب نزدیک به جواب بهینه
\fn{1}{near-optimal solution}
نیز کافی است. به چنین الگوریتم‌هایی که جواب نزدیک به بهینه تولید می‌کنند، الگوریتم‌های تقریبی
\fn{2}{approximation algorithm}
می‌گوییم. برای بسیاری از مسائل ان‌پی کامل می‌توان یک الگوریتم تقریبی در زمان چندجمله‌ای پیدا کرد.
\end{itemize}
\end{frame}


\begin{frame}{‌الگوریتم‌های تقریبی}
\begin{itemize}\itemr
\item[-]
فرض کنید بر روی مسئلهٔ بهینه‌سازی کار می‌کنید که در آن هر یک از جواب‌های بالقوه
\fn{1}{potential solution}
دارای یک هزینه است و می‌خواهید یک جواب نزدیک به بهینه پیدا کنید. بسته به نوع مسئله، ممکن است مسئله بیشینه سازی
\fn{2}{maximization}
یا کمینه سازی
\fn{3}{minimization}
باشد. می‌توانید یک جواب بهینه با هزینه حداکثر یا هزینه حداقل پیدا کنید.
\end{itemize}
\end{frame}


\begin{frame}{‌الگوریتم‌های تقریبی}
\begin{itemize}\itemr
\item[-]
می‌گوییم یک الگوریتم دارای ضرب تقریب
 \fn{1}{approximation ratio}
\m{\rho(n)}
است اگر به ازای هر ورودی با اندازهٔ n ، هزینهٔ
\m{C}
جواب تولید شده توسط الگوریتم نسبت به هزینهٔ
\m{C^*}
مربوط به جواب بهینه از مقدار
\m{\rho(n)}
کمتر باشد. به عبارت دیگر :
\begin{align*}
\m{\Bigl\{ \frac{C}{C^*}, \frac{C^*}{C} \Bigr\} \leqslant \rho(n)}
\end{align*}
\item[-]
اگر یک الگوریتم دارای ضریب تقریب
\m{\rho(n)}
باشد، به آن الگوریتم تقریبی
\m{\rho(n)}
می‌گوییم.
\end{itemize}
\end{frame}


\begin{frame}{‌الگوریتم‌های تقریبی}
\begin{itemize}\itemr
\item[-]
از الگوریتم‌های تقریبی
\m{\rho(n)}
هم برای مسائل کمینه سازی و هم برای مسائل بیشینه سازی استفاده می‌شود.
\item[-]
در یک مسئله بیشینه سازی، داریم
\m{0 < C \leqslant C^*}
و بنابراین مقدار
\m{C^*/C}
مقدار بزرگ‌تری است که در آن هزینهٔ جواب بهینه از هزینهٔ جواب تقریبی بزرگ‌تر است.
\item[-]
در یک مسئله کمینه سازی، داریم
\m{0 < C^* \leqslant C}
و بنابراین مقدار
\m{C/C^*}
مقدار بزرگ‌تری است که در آن هزینهٔ جواب تقریبی از هزینهٔ جواب بهینه بزرگ‌تر است.
\item[-]
با فرض اینکه همهٔ هزینه‌ها مقادیر مثبت هستند، ضریب تقریب در یک الگوریتم تقریبی هیچ‌گاه کمتر از ۱ نیست.
\iffalse
 زیرا اگر داشته باشیم
\m{C/C^* \leqslant 1}
آنگاه
\m{C^*/C \geqslant 1}
.
\fi
\item[-]
بنابراین یک الگوریتم تقریبی با ضریب ۱ جوابی بهینه تولید می‌کند و هر چه ضریب تقریب الگوریتم تقریبی بیشتر باشد، جواب به دست آمده از جواب بهینه دورتر است.
\end{itemize}
\end{frame}


\begin{frame}{‌الگوریتم‌های تقریبی}
\begin{itemize}\itemr
\item[-]
برای بسیاری از مسائل، الگوریتم‌های تقریبی چند جمله‌ای با ضریب تقریب کوچک وجود دارد و برای برخی دیگر از مسائل، الگوریتم‌های تقریبی دارای ضریب تقریبی هستند که با مقدار n افزایش پیدا می‌کند.
\item[-]
در برخی از الگوریتم‌های تقریبی چندجمله‌ای، هرچه الگوریتم در زمان بیشتری اجرا شود، ضریب تقریب بهتری به دست می‌آید. در چنین مسائلی می‌توان با افزایش زمان محاسبات ضریب تقریب را بهبود داد.
\end{itemize}
\end{frame}