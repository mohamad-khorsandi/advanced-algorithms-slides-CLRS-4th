%------------------------------------------------------------------------
\begin{itemframe}{مسئله شار بیشینه}{آشنایی با روش فورد-فولکرسون}
\item[-]
روش فورد-فولکرسون
\fn{1}{Ford-Fulkerson method}
برای حل مسئله شار بیشینه طراحی شده‌است. بهتر است که به فورد-فولکرسون «الگوریتم» گفته نشود زیرا پیاده‌سازی دقیقی را مشخص نمی‌کند. بلکه یک چارچوب کلی برای حل مسئله شار بیشینه ارائه می‌دهد و چند پیاده‌سازی مختلف با زمان‌های اجرای متفاوت دارد.
\item[-]
در این بخش ابتدا نحوه کار روش فورد-فولکرسون را بررسی می‌کنیم و در بخش آینده درستی این روش را اثبات می‌کنیم.
\end{itemframe}

%------------------------------------------------------------------------
\begin{itemframe}{مسئله شار بیشینه}{آشنایی با روش فورد-فولکرسون}
\item[-]
مبنای روش فورد-فولکرسون استفاده از یک گراف کمکی به نام گراف متمم
\fn{1}{residual networks}
 است. گراف متمم بسیار شبیه به شبکه شار اصلی است. رأس‌های این گراف همان رأس‌های شبکه شار است امّا یال‌های آن متفاوت است. «ظرفیت یال‌های این گراف نشان می‌دهند که شار در یال های شبکه اصلی چطور می‌تواند تغییر کند.»
\item[-]
گراف متمم برخلاف شبکه شار می‌تواند یال پادمتقارن داشته باشد. امّا تفاوت‌های این دو فقط در همین مسئله است. گراف متتم نیز یک گراف جهت‌دار و دارای دو رأس s و t است. همچنین برای این گراف شار تعریف می‌شود و هر یال ظرفیت مشخصی دارد.
\end{itemframe}
%------------------------------------------------------------------------
\begin{itemframe}{مسئله شار بیشینه}{آشنایی با روش فورد-فولکرسون}
\item[-]
همانطور که گفته‌شد، ظرفیت‌های گراف متمم نشان می‌دهند شار در گراف اصلی چطور می‌تواند تغییر کند. برای مثال به شکل زیر دقت کنید. شکل a یک یال در شبکه شار و شکل b معادل همان یال در گراف متمم است.
\centerimg[.7]{figs/chap01/9.png}
\item[-]
همانطور که مشاهده می‌کنید گراف متمم می‌تواند شامل یالی باشد که در گراف اصلی وجود ندارد.
\item[-]
ظرفیت یال موافق در گراف متمم نشان می‌دهد که ۴ واحد شار می‌توان به یال متناظر در گراف اصلی اضافه کرد. همچنین ظرفیت یال مخالف در گراف متمم نشان می‌دهد که ۶ واحد شار می‌توان از یال متناظر در گراف اصلی کم کرد.
\end{itemframe}

%------------------------------------------------------------------------
\begin{itemframe}{مسئله شار بیشینه}{آشنایی با روش فورد-فولکرسون}
\item[-]
امّا شار در گراف متمم به چه معنیست؟ در روش فورد-فولکرسون شاری که برای گراف متمم قرار می‌دهیم درواقع همان تغییراتی است که می‌خواهیم بر روی شبکه اصلی ایجاد کنیم.
\item[-]
 برای مثال در شکل بالا با اعمال تغییراتی که گراف متمم پیشنهاد می‌دهد یک واحد به شار یال
\m{(u, v)}
 در شبکه اصلی اضافه می‌شود. زیرا در گراف متمم ۲ واحد شار در جهت موافق و ۱ واحد در جهت مخالف داریم.
\item[-]
در اینجا صرفاً برای درک بهتر فقط قسمتی از گراف آورده‌شده. امّا تنها عملیات تعریف شده برای اعمال تغییرات گراف متمم روی شبکه شار، اعمال \textbf{کل} شار گراف متمم بر \textbf{کل} شار اصلی است.
\item[-]
دقت کنید یال پادمتقارن در شبکه شار وجود ندارد بنابراین تبدیل به گراف متمم همیشه ممکن است. در بخش آینده نحوه ساخت گراف متمم را به طور دقیق خواهیم دید. در اینجا هدف ایجاد شهود از نحوه کار گراف متمم است.
\end{itemframe}

%------------------------------------------------------------------------
\begin{itemframe}{مسئله شار بیشینه}{آشنایی با روش فورد-فولکرسون}
\item[-]
بر روی گراف متمم به هر مسیر ساده از منبع تا مقصد «مسیر افزایشی»
\fn{1}{augmenting path}
گفته می‌شود.
\item[-]
مسیر‌های افزایشی ویژگی‌های جالب توجهی دارند:
\item[1]
اگر در گراف متمم یک مسیر افزایشی داشته باشیم و مقداری شار برای این مسیر در نظر بگیریم و شار بقیه گراف را صفر کنیم، در این صورت اعمال تغییرات گراف متمم روی شبکه اصلی باعث افزایش شار می‌شود.
\item[2]
ثابت می‌شود که اگر گراف متمم مسیر افزایشی نداشته باشد، شار گراف اصلی بیشینه است و به جواب رسیده‌ایم.
\end{itemframe}


%------------------------------------------------------------------------
\begin{itemframe}{مسئله شار بیشینه}{آشنایی با روش فورد-فولکرسون}
\item[-]
حال که با گراف متمم آشنا شدیم فهم روش فورد-فولکرسون آسان است. این روش ابتدا کل شار شبکه اصلی را برابر صفر قرار می‌دهد سپس به صورت تکراری اقدام به بهبود مقدار شار می‌کند. مراحل روش فورد-فولکرسون به این ترتیب است:
\item[۱]
شار همه یال‌های شبکه شار را برابر صفر قرار می‌دهیم.
\item[۲]
گراف متمم متناظر با شبکه شار را تشکیل می‌دهیم. درصورتی که مسیر افزایشی وجود نداشته‌باشد، شار بیشینه یافت شده بنابراین کار تمام است.
\item[۳]
درغیر این صورت یک مسیر افزایشی می‌یابیم و بیشترین شار ممکن را برای یال‌های عضو این مسیر در نظر می‌گیریم. و شار بقیه یال‌های گراف متمم را صفر می‌کنیم.
\item[۴]
شار گراف متمم را بر شبکه اصلی می‌افزاییم. و به مرحله ۲ بازمی‌گردیم.

\end{itemframe}

%------------------------------------------------------------------------
\begin{itemframe}{مسئله شار بیشینه}{آشنایی با روش فورد-فولکرسون}
\item[-]
پیش از این مثالی از مسئله شار بیشینه ارائه شد و سعی کردیم آن را به صورت دستی حل کنیم. در ادامه خواهیم دید که روش فورد-فولکرسون چگونه این مثال را حل می‌کند.
\centerimg[.8]{figs/chap01/10.png}
\item[-]
قسمت a نشان دهنده شبکه شار و b نشان دهنده گراف متمم است و مسیر آبی رنگ یک مسیر افزایشی است.
\end{itemframe}

%------------------------------------------------------------------------
\begin{itemframe}{مسئله شار بیشینه}{آشنایی با روش فورد-فولکرسون}
\item[-]
در شکل b شار یال‌ها نشان داده نشده و تنها ظرفیت‌ها نشان داده شده‌اند. امّا طبق روش فورد-فولکرسون می‌دانیم این مقدار را در کل گراف صفر و در تمام یال‌های مسیر افزایشی ۴ است که بیشترین شار ممکن در این مسیر است. (چرا؟)
\item[-]
بنابراین مقدار |f| در شبکه شار برابر ۱۹ واحد و در گراف متمم برابر ۴ واحد است. (يادآوری می‌شود که برای محاسبه این مقدار باید مجموع شار خروجی از s را درنظر بگیرید.)
\end{itemframe}

%------------------------------------------------------------------------
\begin{itemframe}{مسئله شار بیشینه}{آشنایی با روش فورد-فولکرسون}
\item[-]
بیایید شار گراف متمم را بر شبکه اعمال کنیم:
\centerimg[.5]{figs/chap01/11.png}
\item[-]
در مسیر افزایشی شکل b، یال
\m{(v_2, v_3)}
وجود داشت که برخلاف جهت یال متناظر در شکل a است. بنابراین وجود شار در این یال به باعث کاهش شار در شبکه شار می‌شود. به قرار دادن جریان در یال مخالف در گراف متمم cancellation گفته می‌شود.
\end{itemframe}

%------------------------------------------------------------------------
\begin{itemframe}{مسئله شار بیشینه}{آشنایی با روش فورد-فولکرسون}
\item[-]
مقدار |f| در شبکه شار حاصل ۲۳ واحد شد که این مقدار معادل است با مجموع مقدار |f| در شبکه شار و گراف متمم (۱۹‌+۴). این مسئله اتفاقی نیست و خواهیم دید که این قائده به طور کلی برقرار است.
\item[-]
این تنها یک تکرار از روش فورد-فولکرسون بود. حال که شبکه تغییر کرده، گراف متمم هم تغییر می‌کند. شکل زیر گراف متمم در مرحله بعد را نشان می‌دهد:
\centerimg[.5]{figs/chap01/12.png}
\end{itemframe}
