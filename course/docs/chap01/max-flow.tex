%------------------------------------------------------------------------
\begin{itemframe}{مسئله شار بیشینه}{معرفی مسئله}
\item [-]
در دنیای واقعی شبکه‌های بسیاری وجود دارند که با هدف انتقال چیزی ساخته شده‌اند. برای مثال شبکه‌های آب رسانی، مدارات الکتریکی(انتقال جریان الکتریکی)، خطوط راه‌آهن و غیره. به طور معمول در چنین شبکه‌هایی افزایش نرخ انتقال مطلوب است. بنابراین مسئله شار بیشینه می‌پرسد‌: چگونه می‌توان نرخ انتقال را در این شبکه‌ها بیشینه کرد؟
\item [-]
برای حل یک مسئله شار بیشینه باید شبکه را توسط گراف مدلسازی کرد. به این گراف «شبکه شار»
\fn{1}{flow network}
و به هر چیزی که در این شبکه در جریان باشد به طور کلی شار
\fn{2}{flow}
 گفته می‌شود.
\end{itemframe}

%------------------------------------------------------------------------
\begin{itemframe}{مسئله شار بیشینه}{معرفی مسئله}
	\item [-]
برای درک بهتر مسئله شار بییشیه ويژگی‌های این مسئله را همراه با مثال شبکه آب رسانی توضیح می‌دهیم؛
	\item
در مسئله شار بیشیه گراف ثابت فرض می‌شود. (مثال: خطوط آب‌رسانی مثل لوله‌ها و اتصالات از قبل ساخته شده‌ و غیر قابل تغییر اند.)
	\item
هر یال یک ظرفیت مشخص برای انتقال شار دارد و نمی‌تواند بیشتر از آن مقدار انتقال دهد. (مثال: هر لوله آب -بسته به قطر لوله‌- می‌تواند مقدار آبی مشخصی را از خود عبور دهد.)
	\item
سرعت انتقال شار در سراسر شبکه ثابت فرض می‌شود. (مثال: نمی‌توانیم تعیین کنیم آب با فشار بیشتری وارد لوله‌ها شود بنابراین سرعت حرکت آب در لوله‌ها یکسان است.)
\end{itemframe}

%------------------------------------------------------------------------
\begin{itemframe}{مسئله شار بیشینه}{معرفی مسئله}

	\item
در مسئله شار بیشینه تنها متغییری که ما می‌توانیم تعیین کنیم این است که چقدر از ظرفیت هر یال برای انتقال شار استفاده کنیم. (مثال: می‌توانیم یک شیر آب سر راه هر لوله قرار دهیم و بعد از حل مسئله شار بیشینه تعیین کنیم هر شیر اجازه ورود چه حجمی از آب را بدهد.)
	\item
مجموع شار ورودی به هر رأس باید با مجموع شار خروجی برابر باشد. به عبارت دیگر هیچ شاری نمی‌تواند در رأس‌های گراف ذخیره شود یا از بین برود. به این قانون، قانون بقای شار
\fn{1}{flow conservation}
گفته می‌شود.
(مثال: بدیهیست در اتصالاتی که لوله‌ها را به هم وصل می‌کند آب نمی‌تواند ذخیره شود یا نشت کند. ممکن هر اتصال یک یا چند لوله ورودی و خروجی داشته باشد در هر صورت مجموع آب ورودی و خروجی باید برابر باشد.)

\end{itemframe}

%------------------------------------------------------------------------
\begin{itemframe}{مسئله شار بیشینه}{معرفی مسئله}

	\item[-]
با این توضیحات شاید به نظر برسد که برای حل این مسئله کافیست به طور حریصانه از همه ظرفیت همه یال‌ها برای انتقال استفاده کنیم. (مثال: همه شیر آب‌هایی که سر راه لوله‌ها قرار گرفته اند را تا بیشترین مقدار باز کنیم تا در صورتی که مقدار کافی آب به آن لوله رسید از تمام ظرفیت آن لوله برای انتقال آب استفاده کنیم.)
	\item[-]
نکته جالب اینجاست که برخلاف انتظار این الگوریتم حریصانه جواب بهنیه را تولید نمی‌کند. در ادامه مثال‌هایی خواهیم دید که ممکن است کاهش مقدار شار گذرنده از یک یال باعث افزایش شار کلی گذرنده از شبکه شود. (مثال: در شبکه آب رسانی ممکن است با بستن شیر، اجازه ورود حجم کمتری از آب به یک لوله را بدهیم و با این کار شار گذرنده از شبکه را بیشتر کنیم.)
\end{itemframe}

%------------------------------------------------------------------------
\begin{itemframe}{مسئله شار بیشینه}{تعاریف رسمی}
\item[-]
قبل ادامه بحث بهتر است با یک تعریف دقیق و رسمی از شبکه شار آشنا شویم؛
\item
شبکه شار یک گراف جهت‌دار است. به این معنی که شار نمی‌تواند در یک یال در دو جهت حرکت کند.

شبکه شار همرا با تابع c به ورودی مسئله داده می‌شود. تابع c یال‌ها و ظرفیت‌‌ها را نگاشت می‌کند به طوری که ظرفیت یال u به v برابر است با
\m{c(u, v)} .
\item
شبکه شار دارای دو رأس خاص است که ورود شار به شبکه (و یا تولید شار) و خروج از شار شبکه (و یا مصرف شار) را مدل می‌کنند. به این دو رأس منبع
\fn{1}{source}
و مقصد
\fn{2}{sink}
گفته می‌شود. به طور معمول به رأس منبع با s و رأس مقصد با t نشان داده می‌شوند.
\item
رأس منبع و مقصد تنها رئوسی هستند که از قانون بقای شار پیروی نمیکنند.
\end{itemframe}

%------------------------------------------------------------------------
\begin{itemframe}{مسئله شار بیشینه}{تعاریف رسمی}
\item
همانطور که قبلا اشاره شد، گراف جهت‌دار می‌تواند طوقه داشته‌باشد. امّا وجود طوقه در شبکه شار غیر مجاز است.
\item
همچنین دیدیم که وجود یال‌های پادموازی نیز در گراف جهت‌دار بلامانع است. امّا در شبکه شار یال‌های پادموازی هم غیر مجاز اند.(در بخش يادآوری فصل یال‌های پادموازی بحث شده‌اند.)
\item
حذف طوقه (دور به طول ۱) و یال‌های پادموازی (دور به طول ۲) از پیچیدگی مسئله می‌کاهد.
\item
وجود دورهایی با طول بالا تر در شبکه شار مجاز است.
\end{itemframe}

%------------------------------------------------------------------------
\begin{itemframe}{مسئله شار بیشینه}{تعاریف رسمی}

\item[-]
همچنین باید یک تعریف کمی برای شار کل شبکه ارائه کنیم تا مشخص شود منظور از شار بیشینه چیست؛
\item
خروجی این مسئله کردن تابع f است که یال‌ها را به شار گذرنده از آنها نگاشت می‌کند. به این صورت که شار گذرنده از یال u به v برابر است با
\m{f(u, v)} .

\item
تابع f باید دو ويژگی داشته باشد؛ قانون بقای شار را نقض نکند و از مقدار ظرفیت هر یال تجاوز نکند. در صورتی که تابعی مانند
\m{f'}
این ویژگی‌ها را نقض کند یک تابع شار نیست.

\end{itemframe}

%------------------------------------------------------------------------
\begin{itemframe}{مسئله شار بیشینه}{تعاریف رسمی}
\item
مقدار شار کل شبکه با |f| نشان داده می‌شود و به این صورت تعریف می‌شود: \\
\begin{center}
\m{|f| = \sum_{v \in V} f(s, v) - \sum_{v \in V} f(v, s)}
\end{center}
\item
به زبان ساده این عبارت مجموع شار خالص خروجی از رأس منبع را مشخص می‌کند :مجموع شار خارج شونده از منبع منهای مجموع شار وارد شونده به منبع. (معمولاً مجموع شار وارد شونده به منبع صفر است.)
\item
قرارداد می‌کنیم که اگر بین u و v یال وجود نداشته باشد
\m{f(u, v)}
برابر صفر است.
\end{itemframe}
%------------------------------------------------------------------------
\begin{itemframe}{مسئله شار بیشینه}{تعاریف رسمی}
\item[-]
امّا چطور این کمیّت می‌تواند نماینده شار گذرنده از کل شبکه باشد؟
\item
برای درک بهتر این موضوع به شبکه شار به چشم یک بلوک بزرگ نگاه کنید که از رأس منبع شار به آن وارد و از مقصد خارج می‌شود.
\centerimg[.5]{figs/chap01/6.png}

\item
با توجه به قانون بقای شار، شار نمی‌تواند در این بلوک بزرگ بماند بنابراین با همان نرخی که وارد آن می‌شود باید از آن خارج شود. (فرض کنید پیکان‌ها شار خالص را نشان می‌دهند.)‌
\item
بنابراین نرخ خالص شار خروجی از منبع نماینده شاری است که در کل شبکه جریان دارد.
\end{itemframe}


%------------------------------------------------------------------------
\begin{itemframe}{مسئله شار بیشینه}{ساده‌سازی مسئله شار بیشینه}
\item[-]
شاید دقت کرده باشید که در تعریف شبکه شار فرض‌های ساده کننده‌ایی درنظر گرفتیم که لزوماً در یک مسئله واقعی بیشینه‌سازی شار، برقرار نیستند. دو فرض به این شکل داشتیم که در زیر آورده شده‌اند؛
\item[۱]
ممکن است در یک مسئله واقعی بیشینه‌سازی شار چند منبع و چند مقصد داشته باشیم. در حالی که در تعریف شبکه شار تنها یک منبع و مقصد برای شبکه لحاظ کردیم.
\item[۲]
ممکن است در یک مسئله واقعی بیشینه‌سازی شار یال موازی داشته باشیم. درحالی که وجود چنین یالی را در شبکه شار غیر مجاز دانستیم.
\end{itemframe}

%------------------------------------------------------------------------
\begin{itemframe}{مسئله شار بیشینه}{ساده‌سازی مسئله شار بیشینه}
\item[-]
در ادامه نشان می‌دهیم که چطور یک مسئله واقعی بیشینه‌سازی شار را که دارای چند منبع و مقصد است و یال پادمتقارن دارد را مدل کنیم.

\end{itemframe}

%------------------------------------------------------------------------
\begin{itemframe}{مسئله شار بیشینه}{ساده‌سازی مسئله شار بیشینه}
\item
شکل زیر نشان می‌دهد چطور یک شبکه شار با چندین رأس منبع و مقصد را می‌توان به یک شبکه شار با یک منبع و مقصد تبدیل کرد.
\centerimg[.8]{figs/chap01/7.png}

\end{itemframe}

%------------------------------------------------------------------------
\begin{itemframe}{مسئله شار بیشینه}{ساده‌سازی مسئله شار بیشینه}
\item
شکل زیر نشان می‌دهد چطور یک شبکه دارای یال‌های پادمتقارن را می‌توان به یک شبکه شار قابل‌قبول تبدیل کرد.
\centerimg[.9]{figs/chap01/8.png}
\end{itemframe}
%------------------------------------------------------------------------
\begin{itemframe}{مسئله شار بیشینه}{مثالی از مسئله شار بیشینه}
\item[-]
شکل زیر یک شبکه شار را به همراه شار موجود در آن نشان می‌دهد:
\centerimg[.5]{figs/chap01/8-1.png}

\item[-]
برای نشان دادن شبکه شار یک قرارداد شناخته شده وجود دارد که در شکل بالا قابل مشاهده است. به این صورت که ابتدا مقدار شار و سپس ظرفیت هر یال نوشته می‌شود و این دو عدد به وسیله خط مورب از هم جدا می‌شوند.
\end{itemframe}

%------------------------------------------------------------------------
\begin{itemframe}{مسئله شار بیشینه}{مثالی از مسئله شار بیشینه}
\item[-]
همچنین دقت کنید که چگونه قانون پایستگی شار در این شکل رعایت شده. برای مثال ۱۱ واحد به راس
\m{v_4}
وارد شده و ۴ و ۷ واحد از آن خارج می‌شوند.
\item[-]
برای درک مسئله شار بیشینه بهتر است سعی کنیم مسئله را صورت دستی حل کنیم.
\item[-]
برای مثال چطور می‌توان مقدار شار شبکه بالا را افزایش داد؟ به عبارت دیگر باید مقدار شار خروجی از رأس s را بیشتر کنید بدون اینکه قانون پایستگی شار نقض شود. (مقدار شار فعلی ۱۹ واحد است.)
\item[-]
اگر موفق به افزایش شار شبکه شدید چطور می‌توان دریافت که این شار بیشینه است یا خیر؟

\end{itemframe}

%------------------------------------------------------------------------
\begin{itemframe}{مسئله شار بیشینه}{مثالی از مسئله شار بیشینه}
\item[-]
شکل زیر با تغییر مقدار شار ۳ یال، مقدار شار شبکه را به ۲۳ واحد افزایش داده‌است. این سه یال عبارت اند از:
\m{(s, v_2), (v_3, v_2), (v_3, t)}
\centerimg[.5]{figs/chap01/8-2.png}

\item[-]
نکته جالب توجه اینجاست که با صفر کردن شار یال
\m{(v_3, v_2)}
موفق به افزایش شار کل شبکه شدیم.
\end{itemframe}