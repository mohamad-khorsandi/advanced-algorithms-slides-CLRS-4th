
\begin{frame}{‌پسگرد}
\begin{itemize}\itemr
\item[-]
روش پسگرد
\fn{1}{backtrack}
روشی است که با استفاده از آن همهٔ فضای حالت را جستجو می‌کنیم. در این روش با یکی از حالت‌ها در فضای حالت شروع می‌کنیم و برای انتخاب حالت بعد یکی از پارامترهای فضای حالت را تغییر می‌دهیم. ممکن است در انتخاب حالت بعد، چند پارامتر قابل تغییر باشند. در اینصورت یکی از پارامترها را تغییر می‌دهیم و به حالت بعد می‌رویم و سپس پسگرد می‌کنیم تا یکی دیگر از پارامترها را تغییر دهیم. بدین صورت همهٔ حالت‌ها در فضای حالت بررسی می‌شوند.
\end{itemize}
\end{frame}

\iffalse
\begin{frame}{‌پسگرد}
\begin{itemize}\itemr
\item[-]
برای مثال فرض کنید می‌خواهیم همهٔ رشته‌ها با طول ۲ که از سه حرف a و b و c تشکیل شده‌اند را بشماریم. یا به عبارت دیگر همهٔ فضای حالت را تولید کنیم.
\item[-]
با استفاده از روش پسگرد در ابتدا ۳ انتخاب برای حرف اول داریم حرف a را انتخاب می‌کنیم و سپس از بین ۳ انتخاب برای حرف دوم، حرف a را انتخاب می‌کنیم پس رشتهٔ aa را به دست می‌آوریم، سپس پسگرد می‌کنیم و حرف b را برای حرف دوم رشته انتخاب می‌کنیم و رشته ab را بدست می‌آوریم. بار دیگر با یک پسگرد رشتهٔ ac را بدست می‌آوریم. در پسگرد بعدی هیچ انتخابی برای حرف دوم وجود نخواهد داشت پس دوباره پسگرد می‌کنیم و حرف b را به عنوان حرف اول انتخاب می‌کنیم. با استفاده از این روش به ترتیب رشته‌های
ba
،
bb
،
bc
،
ca
،
cb
و
cc
به دست می‌آیند.
\end{itemize}
\end{frame}
\fi

\begin{frame}{‌پسگرد}
\begin{itemize}\itemr
\item[-]
فرض کنید وارد یک باغ پر پیچ و خم یا باغ هزارتو
\fn{1}{maze}
شده‌اید و می‌خواهید راه خروجی را پیدا کند. راه را در پیش می‌گیرید و به هر چند راهی که می‌رسید راه اول از سمت چپ را انتخاب می‌کنید. در پایان یا راه خروجی را پیدا می‌کنید و یا به بن‌بست بر می‌خورید. در صورتی که به بن‌بست رسیدید، بازمی‌گردید تا به اولین چندراهی قبل از بن‌بست برسید. به جای راه اول از سمت چپ، دومین راه از سمت چپ را امتحان می‌کنید و در صورت برخورد با بن‌بست راه را باز می‌گردید و در چند راهی راه سوم را امتحان می‌کنید. فرض کنید که همهٔ راه‌ها در چند راهی آخر را امتحان کردید و به بن‌بست خوردید. در این صورت باید مسیر را بازگردید تا به دومین چند راهی قبل از بن‌بست برسید و این بار در دومین چندراهی قبل از بن‌بست، دومین راه را انتخاب کنید. این روند را ادامه می‌دهید تا راه خروجی را پیدا کنید. به این روش حل مسئله روش پسگرد گفته می‌شود.
\item[-]
هر یک از چندراهی‌ها یکی از پارامترهای مسئله است که مقادیر مختلف آن را امتحان می‌کنید.
\end{itemize}
\end{frame}


\begin{frame}{‌پسگرد}
\begin{itemize}\itemr
\item[-]
روش پسگرد وقتی استفاده می‌شود که می‌خواهیم مسئله‌ای را حل کنیم که در آن عناصر یک دنباله
\fn{1}{sequence}
باید از اشیائی از یک مجموعهٔ
\fn{2}{set}
معین انتخاب شوند، به طوری‌که دنباله ویژگی مشخصی داشته باشد و معیار
\fn{3}{criterion}
معینی را برآورده کند.
\end{itemize}
\end{frame}