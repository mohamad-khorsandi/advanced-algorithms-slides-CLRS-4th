\begin{itemframe}{اثبات‌های نظریه ان‌پی کامل}
\itm
در بخش قبل ثابت کردیم همه زبان‌های ان‌پی به زبان CIRCUIT-SAT کاهش‌پذیر اند.
\itm
در این بخش نشان می‌دهمی که چگونه می‌توان ان‌پی-کامل بودن زبان‌ها را بدون کاهش مستقیم هر زبان در NP به زبان مورد نظر، اثبات کرد.
\itm
لم روبه‌رو مبنایی برای اثبات ان‌پی-کامل بودن یک زبان فراهم می‌سازد:\\
اگر از یک زبان ان‌پی-کامل به زبانی مانند $L$ یک کاهش وجود داشته باشد، $L$ ان‌پی سخت است. حال اگر $L$ ان‌پی هم باشد نتیجه می‌شود ان‌پی کامل است.
\end{itemframe}

%----------------------------------------------------
\begin{itemframe}{اثبات‌های نظریه ان‌پی کامل}
\itm
به عبارت دیگر برای اثبات ان‌پی کامل بودن یک زبان دو کار باید انجام شود؛‌ الف) ثابت شود زبان ان‌پی است. ب) یک کاهش از یک زبان ان‌پی کامل به آن پیدا شود.
\itm
برای فهم بهتر این لم به شکل زیر دقت کنید:
\pic{9}
\end{itemframe}

%----------------------------------------------------
\begin{itemframe}{اثبات‌های نظریه ان‌پی کامل}
\itm
برای مثال حال می‌توانیم به سادگی اثبات کنیم مسئله ارضاء‌پذیری فرمول
\fn{formula satisfiability}
ان‌پی کامل است.
\itm
در مسئله ارضاءپذیری فرمول یک عبارت منطقی داده میشود و‌ باید بررسی کنیم که آیا این عبارت ارضاءپذیر است یا خیر. به عبارت دیگر آیا یک مقدار دهی به متغییرهای آن وجود دارد که کل عبارت را برابر با ۱ کند یا خیر.
\itm
این مسئله از نظر تاریخی اولین مسئله‌ایی است که ثابت شد ان‌پی کامل است.
\end{itemframe}

%----------------------------------------------------
\begin{itemframe}{اثبات‌های نظریه ان‌پی کامل}
\itm
مسئله ارضاءپذیری فرمول به بیان زبان‌های صوری چنین تعریف می‌شود:
%todo encoding should have been mentioned by now
$$
SAT = \{\langle  \phi \rangle: \phi \text{ is a satisfiable boolean formula}\}
$$

\itm
برای اثبات ان‌پی کامل بودن SAT باید نشان دهیم؛
\item[الف]
اگر یک مقداردهی به متغییر‌های یک عبارت منطقی داشته باشیم ارزیابی عبارت در زمان چند جمله‌ایی امکان پذیر است.
\item[ب]
مسئله CIRCUIT-SAT در زمان چند جمله‌ایی قابل کاهش به مسئله SAT است.
\itm
برای دو حالت بالا الگوریتم چند جمله‌ایی وجود دارد. بنابراین به مسئله ارضاءپذیری فرمول، ان‌پی کامل است.
\end{itemframe}

%----------------------------------------------------
\begin{itemframe}{اثبات‌های نظریه ان‌پی کامل}
\itm
هر چه برای زبان‌های بیشتری ثابت کنیم ان‌پی کامل هستند برای اثبات‌های بعدی ابزارهای بیشتری در اختیار داریم. اما هنوز هم کار زیاد ساده نیست.
\itm
تصور کنید که بخواهید همه نمونه‌های یک مسئله مانند ارضاءپذیری فرمول را به نمونه‌هایی از یک مسئله گراف تبدیل کنید. این کار ممکن است نیازمند بررسی تعداد زیادی از حالات باشد.
\itm
بنابراین بهتر است از یک مسئله با ورودی محدودتر استفاده کنیم. مسئله ارضاءپذیری ۳-CNF برای این هدف مناسب است.
\itm
فرم CNF یک فرم استاندارد برای عبارات منطقی است. در درس مدار منطقی این فرم با نام POS نیز شناخته می‌شود.
\end{itemframe}

%----------------------------------------------------
\begin{itemframe}{اثبات‌های نظریه ان‌پی کامل}
\itm
در اینجا ۳-CNF را بررسی می‌کنیم که در آن عبارت AND تعدادی عبارت کوچک‌تر است و هر عبارت کوچک تر دقیقاً شامل ۳ متغییر یا نقیض متغییر است. نمونه‌ایی از عبارت در فرم ۳-CNF به این صورت است:
$$
(\lnot x_1 \lor \lnot x_1 \lor x_2) \land (x_3 \lor x_2 \lor x_4) \land (\lnot x_1 \lor \lnot x_3 \lor \lnot x_4)
$$
\itm
تعریف می‌کنیم زبان ۳-CNF-SAT شامل کدگذاری عبارات منطقی ارضاءپذیر به فرم ۳-CNF‌ است. می‌خواهیم نشان دهیم این زبان ان‌پی کامل است.
\end{itemframe}

%----------------------------------------------------
\begin{itemframe}{اثبات‌های نظریه ان‌پی کامل}

 \itm
زبان ۳-CNF-SAT حالت ساد‌ه‌شده‌ایی از SAT است، پس هر الگوریتمی که SAT را حل کند ۳-CNF-SAT را هم حل می‌کند. از آنجا که SAT ان‌پی است پس ۳-CNF-SAT هم ان‌پی است.
 \itm
اثبات می‌شود که مسئله SAT در زمان چند جمله‌ایی به ۳-CNF-SAT کاهش می‌یابد و در نتیجه ان‌پی سخت است.
در اینجا به بررسی این اثبات نمی‌پردازیم.
 \itm
بنابراین ۳-CNF-SAT نیز عضو ان‌پی کامل است.
\end{itemframe}

%----------------------------------------------------
\begin{itemframe}{اثبات‌های نظریه ان‌پی کامل}
 \itm
مسئله ارضاءپذیری ۳-CNF‌ به ما در اثبات ان‌پی کامل بودن بسیاری از مسائل کمک می‌کند. در شکل زیر می‌توانید ترتیب کاهش مسائل به یکدیگر را ببینید.
\pic[.3]{10}
 \itm
به این ترتیب می‌توان ثابت کرد همه این مسائل ان‌پی کامل هستند.
\end{itemframe}