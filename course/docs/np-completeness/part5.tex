\begin{itemframe}{مسائل ان‌پی کامل}
\itm
مسائل ان‌پی کامل در حوزه‌های متنوعی مطرح می‌شوند مانند: نظریه گراف، طراحی شبکه، مجموعه‌ها، ذخیره‌سازی و بازیابی، توالی و زمان‌بندی، جبر و نظریه اعداد، بازی‌ها و معماها، بهینه‌سازیا، زیست‌شناسی، شیمی، فیزیک، و حوزه‌های دیگر.
 \itm
ان‌پی کامل بودن مسئله ارضاءپذیری ۳-CNF‌ در بخش قبل اثبات شد.
\end{itemframe}


\begin{itemframe}{مسائل ان‌پی کامل}
\itm
این مسئله به ما در اثبات ان‌پی کامل بودن بسیاری از مسائل کمک می‌کند. در شکل زیر می‌توانید ترتیب کاهش مسائل به یکدیگر را ببینید.
\pic[.3]{10}
\itm
ان‌‌پی کامل بودن هر زبان موجود در این شکل، از طریق کاهش از زبانی که به آن اشاره دارد، اثبات می‌شود.
\end{itemframe}


\begin{itemframe-s}{مسائل ان‌پی کامل}{مسئله‌ی کلیک}
\itm
یک کلیک
\fn{clique}
 در یک گراف بدون جهت $G = (V, E)$، زیرمجموعه‌ای مثل
$V'$
از رأس‌هاست که هر زوج از آن‌ها توسط یالی در $E$ به یکدیگر متصل هستند.
به‌عبارت دیگر، کلیک یک زیرگراف کامل از $G$ است.
اندازه‌ی کلیک برابر با تعداد رأس‌های آن است.
\itm
هدف این مسئله‌ی بهینه‌سازی یافتن کلیکی با بیشترین اندازه در گراف است.
مسئله‌ی تصمیم‌گیری متناظر می‌پرسد: آیا کلیکی با اندازه‌ی داده‌شده‌ی $k$ در گراف وجود دارد یا نه.
\itm
تعریف رسمی آن به‌صورت زیر است:
$$
\text{CLIQUE} = \{ \langle G, k \rangle : G \text{G is a graph containing a clique of size k } \}
$$
\end{itemframe-s}


\begin{itemframe-s}{مسائل ان‌پی کامل}{مسئله‌ی کلیک}
\itm
می‌خواهیم ثابت کنیم مسئله‌ی کلیک ان‌پی کامل است.
\itm
\textbf{اثبات:}
ابتدا نشان می‌دهیم که کلیک ان‌پی است. فرض کنید یک گراف و یک زیرمجموعه
$V'$
به عنوان گواه داده‌شده.
\itm
برای بررسی اینکه آیا $V'$ یک کلیک است، در زمان چندجمله‌ای می‌توان بررسی کرد که آیا برای هر زوج
$u, v \in V'$
، یال
$(u, v)$
در $E$ وجود دارد یا نه.

\end{itemframe-s}


\begin{itemframe-s}{مسائل ان‌پی کامل}{مسئله‌ی کلیک}
\itm
حال باید نشان دهیم
$$
\text{3-CNF-SAT} \leq_P \text{CLIQUE}
$$
\itm
ممکن است تعجب کنید که چگونه یک نمونه از 3-CNF-SAT به یک نمونه از $CLIQUE$ کاهش می‌یابد، چراکه در ظاهر، فرمول‌های منطقی ارتباطی با گراف‌ها ندارند.
\itm
الگوریتم کاهش با یک نمونه از 3-CNF-SAT آغاز می‌شود.
فرض کنید:
$$
\phi = C_1 \wedge C_2 \wedge \cdots \wedge C_k
$$
یک فرمول بولی در فرم 3-CNF با $k$ جزء
\fn{clause}
باشد.
هر جزء $C_r$ دقیقاً سه لیترال متمایز
$l^r_1$, $l^r_2$, $l^r_3$
دارد.
\end{itemframe-s}


\begin{itemframe-s}{مسائل ان‌پی کامل}{مسئله‌ی کلیک}
\itm
می‌خواهیم گرافی $G$ بسازیم به‌طوری‌که $\phi$ ارضاء‌پذیر باشد اگر و تنها اگر $G$ دارای کلیکی با اندازه‌ی $k$ باشد.
\itm
ساخت گراف
$G= (V, E)$
: برای هر جزء
$C_r = (l^r_1 \vee l^r_2 \vee l^r_3)$
در
$\phi$
،سه رأس
$v^r_1$, $v^r_2$, $v^r_3$
به $V$ اضافه می‌کنیم.
\itm
سپس یال بین
$(v^r_i, v^s_j)$
را به $E$ اضافه می‌کنیم اگر و تنها اگر:
$v^r_i$
و
$v^s_j$
متعلق به جزءهای مختلف باشند ($r \neq s$)، و
لیترال‌های متناظر آن‌ها سازگار باشند (یعنی $l^r_i$ نقیض $l^s_j$ نباشد).
\itm
این گراف را می‌توان در زمان چندجمله‌ای از $\phi$ ساخت.
\end{itemframe-s}


\begin{itemframe-s}{مسائل ان‌پی کامل}{مسئله‌ی کلیک}
\itm
\textbf{مثال:}
اگر
$$
\phi = (x_1 \vee \lnot x_2 \vee \lnot x_3) \wedge (\lnot x_1 \vee x_2 \vee x_3) \wedge (x_1 \vee x_2 \vee x_3)
$$
باشد، گراف $G$ حاصل مطابق شکل زیر ساخته می‌شود.
\pic{11}

\end{itemframe-s}


\begin{itemframe-s}{مسائل ان‌پی کامل}{مسئله‌ی کلیک}
\itm
اکنون باید نشان دهیم که این تبدیل یک کاهش معتبر است.
\itm
\textbf{جهت اول:}
فرض کنید $\phi$ یک مقداردهی ارضاء‌پذیر دارد.
در این صورت، هر بند $C_r$ حداقل یک لیترال
$l^r_i$
دارد که مقدار ۱ گرفته است.
هر چنین لیترالی متناظر با یک رأس $v^r_i$ است.
\itm
با انتخاب یکی از این لیترال‌های با مقدار ۱ از هر جزء، مجموعه‌ای
$V' \subseteq V$
 با $k$ رأس به‌دست می‌آید.
ادعا می‌کنیم $V'$ یک کلیک است.
\itm
برای هر دو رأس
$v^r_i, v^s_j \in V'$
 با
$r \neq s$،
لیترال‌های $l^r_i$ و $l^s_j$ هر دو مقدار ۱ گرفته‌اند،
پس نمی‌توانند نقیض یکدیگر باشند.
در نتیجه، طبق ساخت گراف، یال
$(v^r_i, v^s_j) \in E$
است.
پس $V'$ یک کلیک است.
\end{itemframe-s}


\begin{itemframe-s}{مسائل ان‌پی کامل}{مسئله‌ی کلیک}
\itm
\textbf{جهت دوم:}
فرض کنید گراف $G$ کلیکی به اندازه $k$ داشته باشد؛ یعنی زیرمجموعه‌ای مانند $V'$ از رأس‌ها با اندازه k که یک کلیک است.
\itm
از آنجا که هیچ یالی بین رأس‌های یک جزء وجود ندارد،
مجموعه $V'$ دقیقاً یک رأس از هر بند شامل می‌شود.
اگر
$v^r_i \in V'$
، آنگاه مقدار ۱ را به لیترال متناظر $l^r_i$ نسبت می‌دهیم.
\itm
از آنجا که گراف هیچ یالی بین لیترال‌های ناسازگار ندارد،
هیچ لیترال و نقیض آن به‌طور هم‌زمان مقدار ۱ نمی‌گیرند.
در نتیجه، هر جزء ارضاءپذیر است، و بنابراین $\phi$ نیز ارضاءپذیر خواهد بود.
\itm
برای متغیرهایی که به هیچ رأسی در کلیک مربوط نمی‌شوند، می‌توان مقداردهی دلخواه انجام داد.
\end{itemframe-s}


\begin{itemframe-s}{مسائل ان‌پی کامل}{مسئله‌ی کلیک}
\itm
در مثال شکل زیر، یک تخصیص ارضاءپذیر برای
$\phi$
مقدار
$x_2 = 0$ و $x_3 = 1$
دارد. یک کلیک متناظر با اندازه
$k = 3$
شامل رأس‌هایی است که به ترتیب متناظر با
$\lnot x_2$
از جزء اول،
$x_3$
از جزء دوم، و
$x_3$
از جزء سوم هستند.
\pic{11}

از آنجا که این کلیک هیچ رأسی متناظر با
$x_1$
 یا
$\lnot x_1$
ندارد، این تخصیص می‌تواند مقدار
$x_1$
را به طور دلخواه 0 یا 1 قرار دهد.
\end{itemframe-s}


\begin{itemframe-s}{مسائل ان‌پی کامل}{مسئله‌ی کلیک}
\itm
ممکن است این‌طور تصور شود که ما تنها نشان داده‌ایم که CLIQUE در گراف‌هایی که رأس‌ها فقط در سه‌تایی‌ها ظاهر می‌شوند و هیچ یالی بین رأس‌های یک سه‌تایی وجود ندارد، ان‌پی سخت است.
\itm
در واقع، ما تنها ان‌پی سخت بودن CLIQUE را در این حالت محدود نشان داده‌ایم، اما همین اثبات برای نشان دادن ان‌پی سخت بودن CLIQUE در گراف‌های کلی کافی است. چرا؟
\itm
اگر الگوریتمی با زمان چندجمله‌ای وجود داشته باشد که مسئله‌ی CLIQUE را در گراف‌های کلی حل کند، آن الگوریتم می‌تواند همان مسئله را در گراف‌های محدود نیز حل کند.
\end{itemframe-s}


\begin{itemframe-s}{مسائل ان‌پی کامل}{مسئله‌ی کلیک}

\itm
اما جهت مخالف، یعنی کاهش دادن نمونه‌های 3-CNF-SAT با ساختار خاص به نمونه‌های کلی از CLIQUE، کافی نیست. چرا؟ شاید نمونه‌هایی از 3-CNF-SAT که ما برای کاهش انتخاب کرده‌ایم، «ساده» باشند، و در نتیجه، ما عملاً یک مسئله‌ی ان‌پی سخت را به CLIQUE کاهش نداده باشیم.
\itm
افزون بر این، کاهش استفاده‌شده تنها از خود نمونه‌ی 3-CNF-SAT استفاده می‌کند، نه از پاسخ آن. اگر کاهش چندجمله‌ای به دانستن این‌که آیا فرمول $\phi$ ارضاء‌پذیر است متکی بود، مرتکب خطا شده بودیم، زیرا ما نمی‌دانیم که چگونه می‌توان در زمان چندجمله‌ای تعیین کرد که آیا $\phi$ ارضاءپذیر است یا نه.
\end{itemframe-s}