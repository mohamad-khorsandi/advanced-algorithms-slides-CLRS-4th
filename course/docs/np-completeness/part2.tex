%----------------------------------------------------
\begin{itemframe}{کلاس ان‌پی}
\itm
اکنون بیایید به الگوریتم‌هایی نگاه کنیم که «عضویت در زبان‌ها» را بررسی می‌کنند.
برای مثال، فرض کنید برای یک نمونه‌ی
$\langle G, u, v, k \rangle$
از مسئله‌ی تصمیم‌گیری PATH،
یک مسیر p از u بهv نیز در اختیار دارید.
می‌توانید بررسی کنید که آیا p واقعاً یک مسیر در گراف G است یا نه، و همچنین اینکه آیا طول مسیر p حداکثر k است یا خیر.
\itm
اگر این دو شرط برقرار باشند، می‌توان p را به عنوان یک گواه
\fn{certificate}
در نظر گرفت که نشان می‌دهد این نمونه واقعاً متعلق به زبان PATH است.
\itm
هرچند برای مسئله‌ای که عضو پی باشد گواه سود چندانی ندارد.
در عوض، بیایید به مسئله‌ای بپردازیم که الگوریتم تصمیم‌گیرنده‌ایی با زمان چندجمله‌ای برای آن نمی‌شناسیم. اما اگر یک گواه در اختیار داشته باشیم، بررسی درستی آن آسان است.
\end{itemframe}


\begin{itemframe}{کلاس ان‌پی}
\itm
دسته‌ایی از مسائلی که پاسخ آنها در زمان چند جمله‌ایی قابل ارزیابی است ان‌پی نامیده می‌شوند. نمونه‌ایی از این مسائل دور همیلتونی است. به گرافی که دور همیلتونی داشته باشد گراف همیلتونی
\fn{hamiltonian}
 گفته می‌شود.
\itm
برای حل این مسئله هیچ الگوریتم با زمان
$O(n^k)$
یافت نشده.
\itm
حالا فرض کنید که کسی به شما بگوید که گراف داده شده همیلتونی است و رأس‌های دور همیلتونی را به ترتیب به شما بدهد کافیست بررسی کنید که ترتیب داده شده یک دور ساده است و همه رأس‌ها را دارد این کار در زمان
$O(n^2)$
امکان پذیر است. بنابراین دور همیلتونی عضو کلاس ان‌پی است.
\end{itemframe}


\begin{itemframe}{کلاس ان‌پی}
\itm
این طور به تفاوت این دو کلاس نگاه کنید:‌ کلاس پی شامل مسائلی است که سریع حل می‌شوند و ان‌پی شامل مسائلی است که ارزیابی درستی آنها آسان است.
\itm
شاید شما طبق تجربه عنوان کنید که ارزیابی درستی پاسخ بسیار از به دست آوردن آن ساده‌تر است. بسیاری از دانشمندان علوم کامپیوتر نیز بر این باوراند.
\end{itemframe}


\begin{itemframe}{کلاس ان‌پی}
\itm
نکته قابل توجه اینجاست که تا به حال این مسئله ثابت نشده است. یعنی هیچ مسئله‌ای نمی‌شناسیم که در زمان چند جمله‌ایی قابل ارزیابی باشد امّا ثابت شود در زمان چند جمله‌ایی قابل حل نیست.
\centerimg{figs/np-completeness/4.png}

\itm
در صورتی که چنین مسئله‌ایی پیدا شود در فضای
$NP - P$
قرار می‌گیرد و ثابت می‌شود
$P \neq NP$.
به هر حال هر دو حالت شکل بالا در حال حاضر ممکن است صحیح باشد.
\end{itemframe}

%----------------------------------------------------
\begin{itemframe}{کلاس ان‌پی}
\itm
مسئله
$P \neq NP$
تنها مسئله حل نشده در این مبحث نیست. برخلاف کار‌ها زیادی که در این زمینه انجام شده هنوز مشخص نیست که آیا کلاس ان‌پی تحت عملگر مکمل بسته است یا خیر. به این معنی که اگر
$L$
عضو پی باشد،
$\overline{L}$
نیز عضو ان‌پی است.
\itm
دسته مسائل مکمل کلاس ان‌پی به اختصار co-NP گفته میشود که co مخفف complement است.
\itm
کلاس کو-ان‌پی به این صورت تعریف می‌شود: همه زبان‌هایی مانند $L$ به طوری که
$\overline{L} \in NP$ .
\end{itemframe}

%----------------------------------------------------
\begin{itemframe}{کلاس ان‌پی}
\itm
به علاوه می‌دانیم کلاس پی نیز تحت مکمل‌گیری بسته است به این معنی که اگر $ L $ در کلاس پی باشد $ \overbar{L} $ نیز در این کلاس است. بنابراین پی زیر مجوعه ان‌پی نیز هست زیرا مکمل همه اعضای آن در ان‌پی حضور دارند.
\itm
دریافتیم که کلاس پی زیر مجموعه اشتراک ان‌پی و کو-ان‌پی است: $ P \subset (NP \cup co-NP)$ . امّا آیا مجموعه پی کل فضای اشتراک بین ان‌پی و کو-ان‌پی را در بر می‌گیرد؟ به عبارت دیگر آیا زبانی در $(NP \cup co-NP) -P $ وجود دارد؟
\itm
سوال دیگر یکی دیگر از سوالات پاسخ داده نشده است. می‌توانید حدس بزنید با وجود چندین سوال که پاسخ آنها مشخص نباشد می‌بایستی چندین نمایش مجموعه‌ایی (نمودار ون) برای هر یک از نظریه‌های محتمل رسم کرد. در شکل زیر چهار حالت محتمل رسم شده است.
\end{itemframe}

%----------------------------------------------------
\begin{itemframe}{کلاس ان‌پی}
\itm
چهار حالت ممکن برای روابط بین کلاس‌های پیچیدگی زمانی بدین شرح است:
\centerimg[.7]{figs/np-completeness/5.png}
\end{itemframe}

%----------------------------------------------------
\begin{itemframe}{کلاس ان‌پی}
\item[شکل a]
کلاس P زیرمجموعهٔ محض NP و NP زیرمجموعهٔ محض co-NP است. بیشتر پژوهشگران این احتمال را بسیار بعید می‌دانند.
\item[شکل b]
اگر NP تحت مکمل بسته باشد، آنگاه NP برابر با co-NP است، اما لزوماً P با NP برابر نیست.
\item[شکل c]
کلاس P زیرمجموعهٔ خاص اشتراک NP و co-NP است، اما NP تحت متمم بسته نیست.
\item[شکل d]
کلاس NP برابر با co-NP نیست و همچنین P زیر مجموعه محض اشتراک NP و co-NP است. بیشتر پژوهشگران این احتمال را محتمل‌ترین حالت می‌دانند.
\end{itemframe}

%----------------------------------------------------
\begin{itemframe}{کلاس ان‌پی}
\itm
بنابراین، درک ما از رابطهٔ دقیق میان پی و ان‌پی به‌طور ناامیدکننده‌ایی ناقص است. اگر یک مسئله در $ NP - P $ قرار گیرد، ممکن است نتوانیم ثابت کنیم این مسئله سخت است یا در زمان چند جمله‌ایی قابل حل است.
\itm
با این حال اگر بتوانیم ثابت کنیم که آن مسئله ان‌پی کامل است، آنگاه اطلاعات ارزشمندی دربارهٔ آن به‌دست آورده‌ایم. ان‌پی کامل بودن یک مسئله به ما می‌گوید که این مسئله به اندازه همه مسائل موجود در کلاس ان‌پی کامل سخت است.
\itm
بخش آینده در مورد  ان‌پی کامل بودن بیشتر صحبت خواهیم کرد.
\end{itemframe}

