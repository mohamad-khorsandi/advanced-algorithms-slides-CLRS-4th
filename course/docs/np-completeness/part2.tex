\begin{itemframe}{کلاس ان‌پی}
\itm
اکنون بیایید به الگوریتم‌هایی نگاه کنیم که «عضویت در زبان‌ها» را بررسی می‌کنند.
برای مثال، فرض کنید برای یک نمونه‌ی
$\langle G, u, v, k \rangle$
از مسئله‌ی تصمیم‌گیری PATH،
یک مسیر p از u بهv نیز در اختیار دارید.
می‌توانید بررسی کنید که آیا p واقعاً یک مسیر در گراف G است یا نه، و همچنین اینکه آیا طول مسیر p حداکثر k است یا خیر.
\itm
اگر این دو شرط برقرار باشند، می‌توان p را به عنوان یک گواه
\fn{certificate}
در نظر گرفت که نشان می‌دهد این نمونه واقعاً متعلق به زبان PATH است.
\itm
هرچند برای مسئله‌ای که عضو پی باشد گواه سود چندانی ندارد.
در عوض، بیایید به مسئله‌ای بپردازیم که الگوریتم تصمیم‌گیرنده‌ایی با زمان چندجمله‌ای برای آن نمی‌شناسیم. اما اگر یک گواه در اختیار داشته باشیم، کار آسان‌تر است.
\end{itemframe}


\begin{itemframe-s}{کلاس ان‌پی}{دور همیلتونی}
\itm
یک دور همیلتونی در یک گراف بدون‌جهت دوری ساده است که هر رأس از مجموعهٔ $V$ را دقیقاً یک بار طی می‌کند.
گرافی که دارای یک دور همیلتونی باشد، گراف همیلتونی
\fn{hamiltonian}
 نامیده می‌شود و در غیر این صورت، غیرهمیلتونی
\fn{nonhamiltonian}
 است.
\itm
مسئله دور همیلتونی ذاتاً یک مسئله تصمیم‌گیری است که می‌پرسد: «آیا گراف داده شده دارای دور همیلتونی است؟». تعریف مسئلهٔ دور همیلتونی به‌صورت یک زبان رسمی صوری به این صورت است:
$$
\text{HAM-CYCLE} = \{ \langle G \rangle : \text{ G is a hamiltonian graph } \}
$$
\end{itemframe-s}


\begin{itemframe-s}{کلاس ان‌پی}{الگوریتم‌های تصدیق}
\itm
حال به مسئله‌ای کمی ساده‌تر توجه کنید. فرض کنید دوستی به شما می‌گوید که گرافی داده‌شده همیلتونی است، و سپس برای اثبات این ادعا، ترتیب رأس‌ها در یک دور همیلتونی را ارائه می‌دهد.
\itm
قطعاً بررسی این اثبات آسان است: کافی‌ست بررسی کنید آیا این دور یک جایگشت از رأس‌های گراف است و آیا هر یک از یال‌های متوالی در دور واقعاً در گراف وجود دارند یا نه.
\itm
می‌توان این الگوریتم بررسی را به گونه‌ای پیاده‌سازی کرد که در زمان
$O(n^2)$
 اجرا شود.
\end{itemframe-s}


\begin{itemframe-s}{کلاس ان‌پی}{الگوریتم‌های تصدیق}
\itm
ما یک الگوریتم تصدیق
\fn{verification algorithm}
 را به‌صورت یک الگوریتم دوپارامتری $A$ تعریف می‌کنیم، که یکی از پارامترهای آن همان رشته‌ای ورودی است که به آن $x$ می‌گوییم و پارامتر دیگر رشته‌ای دودویی به نام گواه
\fn{certificate}
 است که با $y$ نشان داده می‌شود.
\itm
در صورتی گواه صحیح باشد الگوریتم $A$ می‌تواند ثابت کند که x عضو آن زبان است و مقدار ۱ را باز می‌گرداند:
$A(x, y) = 1$
\itm
زبانی که توسط یک الگوریتم بررسی $A$ تصدیق می‌شود به‌صورت زیر تعریف می‌شود:
$$
L = \{ x \in \{0, 1\}^* :\text{ exists there }  y \in \{0, 1\}^* \text{ such that } A(x, y) = 1 \}
$$
\end{itemframe-s}


\begin{itemframe}{کلاس ان‌پی}
\itm
کلاس پیچیدگی ان‌پی شامل زبان‌هایی است که می‌توان آن‌ها را توسط یک الگوریتم با زمان چندجمله‌ای تصدیق کرد.
به‌عبارت دقیق‌تر، یک زبان $L$ متعلق به کلاس NP است اگر و تنها اگر الگوریتمی دوورودی به نام $A$ با زمان اجرای چندجمله‌ای و عدد ثابتی $c$ وجود داشته باشد، به‌طوری‌که:
$$
L =
\{ x \in \{0,1\}^* :
\text{ there exists a certificate y with } |y| = O(|x|^c)
\text{ such tha } A(x, y) = 1
\}
$$
\itm
در این صورت می‌گوییم که الگوریتم $A$ زبان $L$ را در زمان چندجمله‌ای «تصدیق» می‌کند.
\end{itemframe}


\begin{itemframe}{کلاس ان‌پی}
\itm
از بحث قبلی ما درباره‌ی مسئله‌ی دور همیلتونی، می‌توان دریافت که:
$$
HAM-CYCLE \in NP
$$
علاوه بر این، اگر زبانی عضو پی باشد، آنگاه عضو ان‌پی نیز خواهد بود.
\itm
زیرا اگر الگوریتمی با زمان چندجمله‌ای برای تصمیم‌گیری زبان $L$ وجود داشته باشد، می‌توان آن را به یک الگوریتم تصدیق دوورودی تبدیل کرد که گواهی را نادیده می‌گیرد.
سپس محاسبات خودش را انجام می‌دهد تا مشخص کند ورودی متعلق به $L$ است یا خیر. سپس دقیقاً رشته‌هایی را می‌پذیرد که متعلق به زبان مدنظر باشند.
\end{itemframe}


\begin{itemframe}{کلاس ان‌پی}
\itm
بنابراین دریافتیم
$$
P \subseteq NP
$$
اما آیا
$$
P \subset NP
$$
برقرار است یا؟
$$
P = NP
$$
\itm
این طور به تفاوت این دو کلاس نگاه کنید:‌ کلاس پی شامل مسائلی است که در زمان کوتاه قابل حل و ان‌پی شامل مسائلی است که در زمان کوتاه قابل تصدیق هستند.
\itm
شاید شما طبق تجربه عنوان کنید که تصدیق گواه بسیار از به دست آوردن پاسخ ساده‌تر است. بسیاری از دانشمندان علوم کامپیوتر نیز بر این باوراند.
\end{itemframe}


\begin{itemframe}{کلاس ان‌پی}
\itm
نکته قابل توجه اینجاست که تا به حال این مسئله ثابت نشده است. یعنی هیچ مسئله‌ای نمی‌شناسیم که در زمان چند جمله‌ایی قابل ارزیابی باشد امّا ثابت شود در زمان چند جمله‌ایی قابل حل نیست.
\centerimg{figs/np-completeness/4.png}

\itm
در صورتی که چنین مسئله‌ایی پیدا شود در فضای
$NP - P$
قرار می‌گیرد و ثابت می‌شود
$P \neq NP$.
به هر حال هر دو حالت شکل بالا در حال حاضر ممکن است صحیح باشد.
\end{itemframe}


\begin{itemframe}{کلاس ان‌پی}
\itm
مسئله
$P \neq NP$
تنها مسئله حل نشده در این مبحث نیست. برخلاف کار‌ها زیادی که در این زمینه انجام شده هنوز مشخص نیست که آیا کلاس ان‌پی تحت عملگر مکمل بسته است یا خیر. به این معنی که اگر
$L$
عضو پی باشد،
$\overline{L}$
نیز عضو ان‌پی است.
\itm
دسته مسائل مکمل کلاس ان‌پی به اختصار co-NP گفته میشود که co مخفف complement است.
\itm
کلاس کو-ان‌پی به این صورت تعریف می‌شود: همه زبان‌هایی مانند $L$ به طوری که
$\overline{L} \in NP$ .
\end{itemframe}


\begin{itemframe}{کلاس ان‌پی}
\itm
به علاوه می‌دانیم کلاس پی نیز تحت مکمل‌گیری بسته است به این معنی که اگر $ L $ در کلاس پی باشد $ \overbar{L} $ نیز در این کلاس است. بنابراین پی زیر مجوعه ان‌پی نیز هست زیرا مکمل همه اعضای آن در ان‌پی حضور دارند.
\itm
دریافتیم که کلاس پی زیر مجموعه اشتراک ان‌پی و کو-ان‌پی است:
$ \text{P} \subset (\text{NP} \cup \text{co-NP})$
ی. امّا آیا مجموعه پی کل فضای اشتراک بین ان‌پی و کو-ان‌پی را در بر می‌گیرد؟ به عبارت دیگر آیا زبانی در
$(\text{NP} \cup \text{co-NP}) - \text{P} $
 وجود دارد؟
\itm
سوال دیگر یکی دیگر از سوالات پاسخ داده نشده است. می‌توانید حدس بزنید با وجود چندین سوال که پاسخ آنها مشخص نباشد می‌بایستی چندین نمایش مجموعه‌ایی (نمودار ون) برای هر یک از نظریه‌های محتمل رسم کرد. در شکل زیر چهار حالت محتمل رسم شده است.
\end{itemframe}


\begin{itemframe}{کلاس ان‌پی}
\itm
چهار حالت ممکن برای روابط بین کلاس‌های پیچیدگی زمانی بدین شرح است:
\centerimg[.7]{figs/np-completeness/5.png}
\end{itemframe}


\begin{itemframe}{کلاس ان‌پی}
\item[شکل a]
کلاس P زیرمجموعهٔ محض NP و NP زیرمجموعهٔ محض co-NP است. بیشتر پژوهشگران این احتمال را بسیار بعید می‌دانند.
\item[شکل b]
اگر NP تحت مکمل بسته باشد، آنگاه NP برابر با co-NP است، اما لزوماً P با NP برابر نیست.
\item[شکل c]
کلاس P زیرمجموعهٔ خاص اشتراک NP و co-NP است، اما NP تحت متمم بسته نیست.
\item[شکل d]
کلاس NP برابر با co-NP نیست و همچنین P زیر مجموعه محض اشتراک NP و co-NP است. بیشتر پژوهشگران این احتمال را محتمل‌ترین حالت می‌دانند.
\end{itemframe}


\begin{itemframe}{کلاس ان‌پی}
\itm
بنابراین، درک ما از رابطهٔ دقیق میان پی و ان‌پی به‌طور ناامیدکننده‌ایی ناقص است. اگر یک مسئله در $ NP - P $ قرار گیرد، ممکن است نتوانیم ثابت کنیم این مسئله سخت است یا در زمان چند جمله‌ایی قابل حل است.
\itm
با این حال اگر بتوانیم ثابت کنیم که آن مسئله ان‌پی کامل است، آنگاه اطلاعات ارزشمندی دربارهٔ آن به‌دست آورده‌ایم. ان‌پی کامل بودن یک مسئله به ما می‌گوید که این مسئله به اندازه همه مسائل موجود در کلاس ان‌پی کامل سخت است.
\itm
بخش آینده در مورد  ان‌پی کامل بودن بیشتر صحبت خواهیم کرد.
\end{itemframe}
