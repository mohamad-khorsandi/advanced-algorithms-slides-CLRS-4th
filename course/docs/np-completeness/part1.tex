\begin{itemframe}{زمان چند جمله‌ایی}
\decLineSpace
\itm
در طی این بحث رده‌های مختلفی از مسائل سروکار داریم که ساده‌ترین این رده P است. P مجموعه مسائلی است که در زمان چند جمله‌ایی قابل حل اند. در این بخش می‌خواهیم این رده مسائل را دقیق‌تر بررسی کنیم.
\itm
ابتدا بیایید مفهوم قابل حل بودن در زمان چند جمله‌ایی را بررسی کنیم. به‌طور کلی، مسائلی که دارای راه‌حل‌هایی با زمان چندجمله‌ای هستند را آسان در نظر می‌گیریم.
\itm
اگرچه مسئله‌ای را که زمان حل آن از مرتبه
$\Omega(n^{100}))$
 باشد، نمی‌توان آسان برشمرد اما در عمل اکثر مسائلی که با زمان چندجمله‌ای قابل‌حل‌اند، زمان بسیار کمتری می‌طلبند.
\end{itemframe}
%----------------------------------------------------
\begin{itemframe}{زمان چند جمله‌ایی}
\itm
به علاوه تجربه نشان داده که زمانی که اولین الگوریتم چندجمله‌ای برای یک مسئله پیدا می‌شود، معمولاً الگوریتم‌های کارآمدتری نیز در پی آن کشف می‌گردند. به علاوه اگر مسئله‌ایی در زمان چند جمله‌ایی در یکی از مدل‌های محاسباتی حل شود توسط اکثر مدل‌های محاسباتی دیگر نیز در همین زمان حل می‌شود.
\itm
%todo add a short review for AT course in appendix including computational models
برای مثال کلاس مسائل قابل حل در زمان چند جمله‌ایی توسط ماشین تورینگ و ماشین دارای حافظه دسترسی مستقیم یکسان است.
‌یادآوری می‌شود که برای دسترسی حافظه در ماشین تورینگ باید همه خانه‌های قبلی پیمایش شوند.

\end{itemframe}

%----------------------------------------------------
\begin{itemframe-s}{زمان چند جمله‌ایی}{تعریف مسئله}
\itm
برای فهم کلاس P باید ابتدا یک تعریف رسمی برای «مسئله» ارائه کنیم.
\itm
یک مسئله یک رابطه بین دو مجموعه I (نمونه‌های مسئله) و S (پاسخ نمونه‌ها)

همانطور که گفته‌شد در این مبحث تنها با مسائل تصمیم گیری سروکار داریم.
در یک مسئله تصمیم‌گیری مجموعه S تنها شامل ۰ و ۱ است.
\itm
برای مثال شکل تصمیم‌گیری مسئله کوتاه ترین مسیر به این صورت است:
آیا از مبداء به مقصد در گراف داده‌شد مسیری با طول کوتاه‌تر از k وجود دارد یا خیر.

\end{itemframe-s}
%----------------------------------------------------
\begin{itemframe-s}{زمان چند جمله‌ایی}{کد کردن مسئله}
\itm
برای اینکه یک نمونه مسئله به عنوان ورودی به کامپیوتر داده شود اول باید آن را تبدیل به رشته کنیم که کامپیوتر متوجه آن شود.
\itm
به این کار کد کردن
\fn{encoding}
مسئله گفته می‌شود. به طور دقیق‌تر کدگذاری یک نگاشت نمونه‌های یک مسئله
 (مجموعه S )
به مجموعه‌ایی از رشته‌های باینری است.
\itm
به مسئله‌ایی که اعضای مجموعه S آن مسائل کداگذاری نشده باشند «مسئله‌ی انتزاعی»
\fn{abstract problem}
و زمانی که S را مجموعه‌ایی از رشته‌های باینری قرار دهیم به آن «مسئله‌ی غیر انتزاعی»
\fn{abstract problem}
گفته می‌شود.
\itm
برای مثال تبدیل اعداد با سیستم دودویی و کارکترها با سیستم ASCII هر کدام از انواع کدگذاری‌اند.
\end{itemframe-s}

%----------------------------------------------------
\begin{itemframe-s}{زمان چند جمله‌ایی}{کد کردن مسئله}
\itm
اندازه یک نمونه از یک مسئله غیر انتزاعی طول آن رشته دودویی است.
حل کردن یک مسئله به صورت رسمی چنین تعریف می‌شود: می‌گوییم یک الگوریتم یک مسئله غیر انتزاعی را را در زمان
$O(f(n))$
حل می‌کند اگر الگوریتم با دادن یک نمونه از مسئله با اندازه n در زمان
$O(f(n))$
پاسخ درست را به دست آورد.
\itm
یک مسئله غیر انتزاعی قابل حل در زمان چند جمله‌ایی است اگر الگوریتمی وجود داشته باشد که این مسئله را در زمان
$O(n^k)$
«حل» کند.
%TODO defining solve in poly for abstract problems
% TODO add encoding notation (<G>)
\end{itemframe}

%----------------------------------------------------
\begin{itemframe-s}{زمان چند جمله‌ایی}{مروری بر تئوری زبان‌ها صوری}
\itm
گفتیم که مسائلی که در این مبحث روی آنها کار می‌کنیم را به مسائل تصمیم گیری محدود می‌کنیم. با این فرض می‌توانیم از مفاهیم نظریه زبان‌های صوری
\fn{formal-language theory}
استفاده کنیم. این نظریه تحت عنوان نظریه زبان‌ها و ماشین‌ها نیز شناخته می‌شود.
\itm
در صورتی که با این نظریه آشنایی ندارید برای مطالعه ادامه این درس بهتر است پیوست یک که مربوط به زبان‌های صوری است را مطالعه کنید.
\end{itemframe}

