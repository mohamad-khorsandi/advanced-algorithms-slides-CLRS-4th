\begin{itemframe}{زمان چند جمله‌ایی}
\decLineSpace
\itm
در طی این بحث رده‌های مختلفی از مسائل سروکار داریم که ساده‌ترین این رده P است. P مجموعه مسائلی است که در زمان چند جمله‌ایی قابل حل اند. در این بخش می‌خواهیم این رده مسائل را دقیق‌تر بررسی کنیم.
\itm
ابتدا بیایید مفهوم قابل حل بودن در زمان چند جمله‌ایی را بررسی کنیم. به‌طور کلی، مسائلی که دارای راه‌حل‌هایی با زمان چندجمله‌ای هستند را آسان در نظر می‌گیریم.
\itm
اگرچه مسئله‌ای را که زمان حل آن از مرتبه
$\Omega(n^{100}))$
 باشد، نمی‌توان آسان برشمرد اما در عمل اکثر مسائلی که با زمان چندجمله‌ای قابل‌حل‌اند، زمان بسیار کمتری می‌طلبند.
\end{itemframe}
%----------------------------------------------------
\begin{itemframe}{زمان چند جمله‌ایی}
\itm
به علاوه تجربه نشان داده که زمانی که اولین الگوریتم چندجمله‌ای برای یک مسئله پیدا می‌شود، معمولاً الگوریتم‌های کارآمدتری نیز در پی آن کشف می‌گردند. به علاوه اگر مسئله‌ایی در زمان چند جمله‌ایی در یکی از مدل‌های محاسباتی حل شود توسط اکثر مدل‌های محاسباتی دیگر نیز در همین زمان حل می‌شود.
\itm
برای مثال کلاس مسائل قابل حل در زمان چند جمله‌ایی توسط ماشین تورینگ و ماشین دارای حافظه دسترسی مستقیم یکسان است.
‌یادآوری می‌شود که برای دسترسی حافظه در ماشین تورینگ باید همه خانه‌های قبلی پیمایش شوند.
\itm
در این بخش هدف بررسی دقیق‌تر کلاس P است. برای اینکار ابتدا باید تعدادی مفاهیم مرتبط را معرفی کنیم.
\end{itemframe}

%----------------------------------------------------
\begin{itemframe-s}{زمان چند جمله‌ایی}{تعریف مسئله انتزاعی}
\itm
یک «مسئله انتزاعی»
\fn{abstract problem}
 یک رابطه بین دو مجموعه I (نمونه‌های مسئله) و S (پاسخ نمونه‌ها) است.
همانطور که گفته‌شد در این مبحث تنها با مسائل تصمیم گیری سروکار داریم.
در یک مسئله تصمیم‌گیری مجموعه S تنها شامل دو عضو است؛ ۰ و ۱.
\itm
مسئله انتزاعی در واقع همان چیزیست که ما به عنوان مسئله معمولی می‌شناسیم. از این جهت به آنها انتزاعی می‌گوییم که به خودی خود  برای کامپیوتر قابل فهم نیستند. به بیان دیگر هنوز تعریف نکرده‌ایم که اجزای مجموعه I دقیقاً چیستند.
\end{itemframe-s}
%----------------------------------------------------
\begin{itemframe-s}{زمان چند جمله‌ایی}{کد گذاری مسئله}
\itm
برای اینکه یک نمونه مسئله به عنوان ورودی به یک برنامه کامپیوتری داده شود اول باید آن را تبدیل به رشته کنیم که کامپیوتر متوجه آن شود.
\itm
به این کار کد گذاری
\fn{encoding}
مسئله گفته می‌شود. به طور دقیق‌تر کدگذاری یک نگاشت نمونه‌های یک مسئله (مجموعه S )
به مجموعه‌ایی از رشته‌های باینری است.
\itm
به مسئله‌ایی که اعضای مجموعه S آن مسائل کداگذاری نشده باشند «مسئله‌ی انتزاعی»
و زمانی که S را مجموعه‌ایی از رشته‌های باینری قرار دهیم به آن «مسئله‌ی عینی»
\fn{concrete problem}
گفته می‌شود.
\itm
برای مثال تبدیل اعداد با سیستم دودویی و کارکترها با سیستم ASCII هر کدام از انواع کدگذاری‌اند.
\end{itemframe-s}

%----------------------------------------------------
\begin{itemframe-s}{زمان چند جمله‌ایی}{کد گذاری مسئله}
\itm
یک مسئلهٔ عینی در زمان چند جمله ای قابل حل
\fn{polynomial-time solvable}
 است، اگر الگوریتمی وجود داشته باشد که در زمان
$O(n^k)$
پاسخ درست مسئله را تولید کند. به طوری که k مقداری ثابت و n طول رشته کد گذاری شده است.
\itm
اکنون می‌توانیم کلاس پیچیدگی P رسماً تعریف کنیم؛ مجموعه‌ای از مسائل تصمیم‌گیری عینی که در زمان چندجمله‌ای قابل حل هستند.

\itm
اما برای ما کار کردن با مسائل انتزاعی ساده‌تر است بنابراین می‌خواهیم حل پذیری در زمان چند جمله‌ایی را روی مسائل انتزاعی نیز تعریف کنیم.
\end{itemframe}

%----------------------------------------------------
\begin{itemframe-s}{زمان چند جمله‌ایی}{کد گذاری مسئله}

\itm
بهتر است این تعریف مستقل از نحوه کدگذاری مسئله باشد. همانطور که می‌دانید ما معمولاً در مورد اینکه یک مسئله را می‌توان در چه زمانی حل کرد اظهار نظر می‌کنیم بدون اینکه کدگذاری مدنظر خود را معرفی کنیم.
\itm
متأسفانه این کار چندان هم ممکن نیست. برای مثال فرض کنید یک الگوریتم تنها یک عدد w ورودی می‌گیرد. می‌توانیم این عدد را به صورت یگانی
\fn{unary}
یا به صورت دودویی کدگذاری کنیم. (در کدگذاری یگانی w تا 1 در رشته قرار می‌گیرد.)
\itm
فرض کنید برای هر دو الگوریتمی از مرتبه خطی نسبت به اندازه ورودی موجود باشد بنابراین پیچیدگی زمانی هر دو را با
$\Theta(n)$
نشان می‌دهیم. اما این دو پیچیدگی زمانی در واقع برابر نیستند. زیرا اندازه n در کدگذاری یگانی نسبت به کد گذاری دودویی نمایی است.
\end{itemframe}

%----------------------------------------------------
\begin{itemframe-s}{زمان چند جمله‌ایی}{کد گذاری مسئله}
\itm
 دیدیم که تاثیر کدگذاری بر تعریف ما از زمان چند جمله‌ایی اجتناب ناپذیر است. اما در عمل اصلاً کدگذاری‌های بیش از حد طولانی مانند یگانی را در نظر نمی‌گیریم.
\itm
تفاوت بین بقیه کدگذاری‌های معقول هم تأثیر کوچکی بر اینکه آیا مسئله در زمان چند جمله‌ایی حل می‌شود یا خیر می‌گذارد.
\itm
برای مثال نمایش در مبنای ۲ و یا مبنای ۳ تفاوتی در چند جمله‌ایی بودن مسئله نمی‌گذارد زیرا می‌توانیم در زمان چندجمله‌ایی مبنای ۳ را به مبنای ۲ تبدیل کنیم.
\end{itemframe}

%----------------------------------------------------
\begin{itemframe-s}{زمان چند جمله‌ایی}{کد گذاری مسئله}
\itm
برای یک مجموعه از نمونه‌های مسئله به نام، می‌گوییم دو کدگذاری با هم ارتباط چندجمله‌ای
\fn{polynomially related}
 دارند، اگر دو تابع وجود داشته باشند که در زمان چندجمله‌ای این دو کدگذاری را به یکدیگر تبدیل کنند.
\itm
اگر دو کدگذاری از یک مسئلهٔ انتزاعی با یکدیگر ارتباط چندجمله‌ای داشته باشند، آنگاه این که آیا آن مسئله در زمان چندجمله‌ای قابل حل است یا نه، مستقل از انتخاب هریک از این دو کدگذاری خواهد بود.
\itm
به طور کلی فرض میکنیم کدگذاری مسئله «استاندارد» است
یعنی کدگذاری اعداد صحیح ارتباط چندجمله‌ایی به کدگذاری باینری داشته باشد و
کد گذاری لیست‌ها ارتباط مستقیم با کدگذاری به صورت
$\{item1, item2, item3, item4\}$
داشته باشد.
\end{itemframe}
%todo ask: isn't unary related to binary encoding?

%----------------------------------------------------
\begin{itemframe-s}{زمان چند جمله‌ایی}{کد گذاری مسئله}
\itm
با داشتن این کدگذاری استاندارد می‌توانیم یک کدگذاری معقول برای بقیه تعاریف ریاضی مانند گراف، تاپل، درخت و غیره نیز در نظر گرفت.
\itm
نماد کدگذاری استاندارد
$\langle \rangle$
 است. برای مثال اگر
$G$
 یک گراف باشد
$\langle G \rangle $
کدگذاری استاندارد G را نشان می‌دهد.
\itm
وقتی فرض کنیم که کدگذاری استفاده شده ارتباط چند جمله‌ایی با کدگذاری استاندارد دارد، می‌توانیم مستقیماً در مورد مسائل انتزاعی بحث کنیم. از این به بعد فرض می‌کنم که همه مسائل به صورت استاندارد کدگذاری شده‌اند.
همچنین معمولاً از تفاوت بین مسائل انتزاعی و عینی صرف نظر می‌کنیم.
\end{itemframe}

%----------------------------------------------------
\begin{itemframe-s}{زمان چند جمله‌ایی}{مروری بر نظریه زبان‌ها صوری}
\itm
گفتیم که مسائلی که در این مبحث روی آنها کار می‌کنیم را به مسائل تصمیم گیری محدود می‌کنیم. با این فرض می‌توانیم از مفاهیم نظریه زبان‌های صوری
\fn{formal-language theory}
استفاده کنیم. این نظریه تحت عنوان نظریه زبان‌ها و ماشین‌ها
\fn{formal language and automata theory}
 نیز شناخته می‌شود.
\itm
دیدگاه این نظریه نسبت به مسئله متفاوت است. در واقع در این نظریه هر مسئله یک زبان صوری
\fn{formal language}
 است و زبان ها نیز مجوعه‌ایی از رشته‌ها روی یک الفبای مشخص هستند.
\itm
از این دیدگاه در آینده بهره خواهیم برد. بنابراین اگر با این نظریه آشنایی ندارید بهتر است پیوست شماره یک را مطالعه کنید.
\end{itemframe}
