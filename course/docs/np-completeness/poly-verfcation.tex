\begin{itemframe}{ارزیابی در زمان چند جمله‌ایی}
\item[-]
دسته‌ایی از مسائلی که پاسخ آنها در زمان چند جمله‌ایی قابل ارزیابی است ان‌پی نامیده می‌شوند. نمونه‌ایی از این مسائل دور همیلتونی است. به گرافی که دور همیلتونی داشته باشد گراف همیلتونی
\fn{1}{hamiltonian}
 گفته می‌شود.
\item[-]
برای حل این مسئله هیچ الگوریتم با زمان
$O(n^k)$
وجود ندارد.
\item[-]
حالا فرض کنید که کسی به شما بگوید که گراف داده شده همیلتونی است و رأس‌های دور همیلتونی را به ترتیب به شما بدهد کافیست بررسی کنید که ترتیب داده شده یک دور ساده است و همه رأس‌ها را دارد این کار در زمان
$O(n^2)$
امکان پذیر است. بنابراین دور همیلتونی عضو کلاس ان‌پی است.

\end{itemframe}

\begin{itemframe}{ارزیابی در زمان چند جمله‌ایی}
\item [-]
این طور به تفاوت این دو کلاس نگاه کنید:‌ کلاس پی شامل مسائلی است که سریع حل می‌شوند و ان‌پی شامل مسائلی است که ارزیابی درستی آنها آسان است.
\item [-]
شاید شما طبق تجربه عنوان کنید که ارزیابی درستی پاسخ بسیار از به دست آوردن آن ساده‌تر است. بسیاری از دانشمندان علوم کامپیوتر نیز بر این باوراند


\end{itemframe}

\begin{itemframe}{ارزیابی در زمان چند جمله‌ایی}
\item [-]
نکته قابل توجه اینجاست که تا به حال این مسئله ثابت نشده است. یعنی هیچ مسئله‌ای نمی‌شناسیم که در زمان چند جمله‌ایی قابل ارزیابی باشد امّا ثابت شود در زمان چند جمله‌ایی قابل حل نیست.

\centerimg{figs/np-completeness/4.png}

\item [-]
در صورتی که چنین مسئله‌ایی پیدا شود در فضای
$NP - P$
قرار می‌گیرد و ثابت می‌شود
$P \neq NP$.
به هر حال هر دو حالت شکل بالا در حال حاضر ممکن است صحیح باشد.
\end{itemframe}

\begin{itemframe}{ارزیابی در زمان چند جمله‌ایی}
\item [-]
مسئله
$P \neq NP$
تنها مسئله حل نشده در این مبحث نیست. برخلاف کار‌ها زیادی که در این زمینه انجام شده هنوز مشخص نیست که آیا کلاس ان‌پی تحت عملگر مکمل بسته است یا خیر. به این معنی که اگر
$L$
عضو پی باشد،
$\overline{L}$
نیز عضو ان‌پی است.
%todo this part requires knowlage from AT course.
\item [-]
دسته مسائل مکمل کلاس ان‌پی به اختصار co-NP گفته میشود که co مخفف complement است.
\item [-]
کلاس کو-ان‌پی به این صورت تعریف می‌شود: همه زبان‌هایی مانند
$ L $
 به طوری که
$\overline{L} \in NP$ .

\end{itemframe}

\begin{itemframe}{ارزیابی در زمان چند جمله‌ایی}
\item [-]
می‌خواهیم نمایش مجموعه‌ایی (نمودار ون) این حالت‌ها را بررسی کنیم. برای اینکار باید چند سوال دیگر پاسخ دهیم:
\item [-]
آیا کلاس پی تحت مکمل‌گیری بسته است؟ فرض کنید الگوریتمی دارید که

\end{itemframe}