%----------------------------------------------------
\begin{itemframe-s}{زمان چند جمله‌ایی}{مروری بر نظریه زبان‌ها صوری}

\itm
در ادامه می‌خواهیم برخی تعاریف این نظریه را مرور کنیم.
\end{itemframe}

%todo ask: move this to np-complete chapter?
%----------------------------------------------------
\begin{itemframe-s}{زمان چند جمله‌ایی}{مروری بر نظریه زبان‌ها صوری}
\itm
یک الفبا $\Sigma$ یک مجموعهٔ متناهی از نمادهاست.
یک زبان $L$ روی $\Sigma$، هر مجموعه‌ای از رشته‌هاست که از نمادهای موجود در $\Sigma$ ساخته شده باشند.
\itm
برای مثال، اگر
$\Sigma = {0, 1}$
، آنگاه مجموعهٔ
$$
L = \{10,\ 11,\ 101,\ 111,\ 1011,\ 1101,\ 10001,\ \dots\}
$$
یک زبان است که حاوی نمایش دودویی اعداد اول است.
\itm
رشتهٔ تهی را با
$\varepsilon$
نمایش می‌دهیم و زبان تهی را با
$\emptyset$
و
$\Sigma^*$
یک زبان شامل همهٔ رشته‌ها روی الفبای
$\Sigma$
است.
\end{itemframe}

%----------------------------------------------------
\begin{itemframe-s}{زمان چند جمله‌ایی}{مروری بر نظریه زبان‌ها صوری}
\itm
هر زبان $L$ روی
$\Sigma$
 زیرمجموعه‌ای از
$\Sigma^*$
 است.
\itm
زبان‌ها از عملیات مختلفی پشتیبانی می‌کنند. از انجا که زبان‌ها مجموعه هستند، عملیات‌های نظریهٔ مجموعه‌ها مانند اجتماع و اشتراک روی آنها تعریف می‌شود که مستقیماً از تعریف‌های مجموعه‌ای پیروی می‌کنند.
\itm
همچنین متمم
\fn{complement}
یک زبان $L$ به صورت زیر تعریف می‌شود:
$$
\overline{L} = \Sigma^* - L
$$
\end{itemframe}

%----------------------------------------------------
\begin{itemframe-s}{زمان چند جمله‌ایی}{مروری بر نظریه زبان‌ها صوری}
\itm
الحاق
\fn{concatenation}
 دو زبان $L_1$ و $L_2$ زبانی است به صورت:
$$
L = \{x_1x_2 : x_1 \in L_1 \text{ and } x_2 \in L_2\}
$$
\itm
بستار
\fn{closure}
یا ستارهٔ کلینی
\fn{Kleene star}
یک زبان $L$، زبانی است به صورت:
$$
L^* = \{\varepsilon\} \cup L \cup L^2 \cup L^3 \cup \cdots
$$
همچنین $L^k$ زبانی است که از الحاق $k$ مرتبه‌ای زبان $L$ با خودش به دست می‌آید.
\end{itemframe}

%----------------------------------------------------
\begin{itemframe-s}{زمان چند جمله‌ایی}{مروری بر نظریه زبان‌ها صوری}
\itm
در دیدگاه نظریهٔ زبان‌ها، همه نمونه‌ها برای هر مسئلهٔ تصمیم‌گیری $Q$، مجموعهٔ
$\Sigma^*$
 است که در آن
$\Sigma = {0, 1}$.
\itm
از آن‌جا که $Q$ به‌طور کامل با آن دسته از نمونه‌های مسئله که خروجی آن‌ها برابر ۱ (یعنی پاسخ «بله») است مشخص می‌شود، می‌توان $Q$ را به صورت زبان $L$ روی
$\Sigma = {0, 1}$
 در نظر گرفت، به طوری که:
$$
L = \{x \in \Sigma^* : Q(x) = 1\}
$$
\end{itemframe}
