\section{پیوست شماره یک: مروری بر نظریه زبان‌ها صوری}

%todo ask: move this to np-complete chapter?
%----------------------------------------------------
\begin{itemframe}{مروری بر نظریه زبان‌ها صوری}
\itm
یک الفبا $\Sigma$ یک مجموعهٔ متناهی از نمادهاست.
یک زبان $L$ روی $\Sigma$، هر مجموعه‌ای از رشته‌هاست که از نمادهای موجود در $\Sigma$ ساخته شده باشند.
\itm
برای مثال، اگر
$\Sigma = {0, 1}$
، آنگاه مجموعهٔ
$$
L = \{10,\ 11,\ 101,\ 111,\ 1011,\ 1101,\ 10001,\ \dots\}
$$
یک زبان است که حاوی نمایش دودویی اعداد اول است.
\itm
رشتهٔ تهی را با
$\varepsilon$
نمایش می‌دهیم و زبان تهی را با
$\emptyset$
و
$\Sigma^*$
یک زبان شامل همهٔ رشته‌ها روی الفبای
$\Sigma$
است.
\end{itemframe}

%----------------------------------------------------
\begin{itemframe}{مروری بر نظریه زبان‌ها صوری}
\itm
هر زبان $L$ روی
$\Sigma$
 زیرمجموعه‌ای از
$\Sigma^*$
 است.
\itm
زبان‌ها از عملیات مختلفی پشتیبانی می‌کنند. از انجا که زبان‌ها مجموعه هستند، عملیات‌های نظریهٔ مجموعه‌ها مانند اجتماع و اشتراک روی آنها تعریف می‌شود که مستقیماً از تعریف‌های مجموعه‌ای پیروی می‌کنند.
\itm
همچنین متمم
\fn{complement}
یک زبان $L$ به صورت زیر تعریف می‌شود:
$$
\overline{L} = \Sigma^* - L
$$
\end{itemframe}

%----------------------------------------------------
\begin{itemframe}{مروری بر نظریه زبان‌ها صوری}
\itm
الحاق
\fn{concatenation}
 دو زبان $L_1$ و $L_2$ زبانی است به صورت:
$$
L = \{x_1x_2 : x_1 \in L_1 \text{ and } x_2 \in L_2\}
$$
\itm
بستار
\fn{closure}
یا ستارهٔ کلینی
\fn{Kleene star}
یک زبان $L$، زبانی است به صورت:
$$
L^* = \{\varepsilon\} \cup L \cup L^2 \cup L^3 \cup \cdots
$$
همچنین $L^k$ زبانی است که از الحاق $k$ مرتبه‌ای زبان $L$ با خودش به دست می‌آید.
\end{itemframe}

%----------------------------------------------------
\begin{itemframe}{مروری بر نظریه زبان‌ها صوری}
\itm
در دیدگاه نظریهٔ زبان‌ها مجموعهٔ
$\Sigma^*$

 است که در آن
$\Sigma = {0, 1}$
، مجموعهٔ همه نمونه‌های یک مسئله تصمیم گیری است.
\itm
از آن‌جا که $Q$ به‌طور کامل با آن دسته از نمونه‌های مسئله که خروجی آن‌ها برابر ۱ (یعنی پاسخ «بله») است مشخص می‌شود، می‌توان $Q$ را به صورت زبان $L$ روی
$\Sigma = {0, 1}$
 در نظر گرفت، به طوری که:
$$
L = \{x \in \Sigma^* : Q(x) = 1\}
$$
\end{itemframe}

%----------------------------------------------------
\begin{itemframe}{مروری بر نظریه زبان‌ها صوری}
\itm
برای مثال، مسئله‌ی تصمیم‌گیری PATH دارای زبان متناظر زیر است:

\begin{align*}
\text{PATH} = \{ \langle G, u, v, k \rangle :  & G = (V, E) \text{ is an undirected graph } \\
				& u, v \in V، \\
				& k \geq 0 \text{ is an integer, and } \\
				& G \text{ G contains a path from u to v with at most k edges }
\}
\end{align*}
\itm
گاهی از نام یکسان برای اشاره به هر دو مفهوم «مسئله‌ی تصمیم‌گیری» و «زبان متناظر آن» استفاده می‌کنیم. برای مثال PATH می‌تواند به مسئله تصمیم گیری کوتاه‌ترین مسیر یا زبان بالا اشاره داشته باشد.
\end{itemframe}

%----------------------------------------------------
\begin{itemframe}{مروری بر نظریه زبان‌ها صوری}
\itm
چارچوب زبان‌های صوری به ما امکان می‌دهد که رابطه‌ی بین مسائل تصمیم‌گیری و الگوریتم‌هایی که آن‌ها را حل می‌کنند را به‌طور فشرده بیان کنیم.
\itm
می‌گوییم که یک الگوریتم A رشته‌ای
$x \in \{0,1\}^*$
 را «می‌پذیرد»
\fn{accepts}
اگر خروجی الگوریتم با ورودی x برابر با ۱ باشد، یعنی
$A(x) = 1$
زبان پذیرفته‌شده توسط الگوریتم A مجموعه‌ای از رشته‌هاست که الگوریتم آن‌ها را می‌پذیرد:
$$
L = \{ x \in \{0,1\}^* : A(x) = 1 \}
$$
\end{itemframe}

%----------------------------------------------------
\begin{itemframe}{مروری بر نظریه زبان‌ها صوری}
\itm
اگر
$A(x) = 0$
 باشد، می‌گوییم الگوریتم A رشته‌ی x را «رد می‌کند»
\fn{rejects} .
\itm
حتی اگر الگوریتمی مثل A  زبانی مانند L را بپذیرد، این الگوریتم لزوماً رشته‌ی
$x \notin L$
را که به آن داده شده است رد نمی‌کند.
به جای آن، ممکن است الگوریتم برای همیشه در حلقه بماند.
\end{itemframe}

%----------------------------------------------------
\begin{itemframe}{مروری بر نظریه زبان‌ها صوری}
\itm
زبان L توسط الگوریتم A «تصمیم‌گیری»
\fn{decided}
 می‌شود اگر:
\item[1]
هر رشته‌ی دودویی در L توسط A پذیرفته شود، و
\item[2]
هر رشته‌ی دودویی که در L نیست، توسط A رد شود.
\itm
به یک زبان «قابل پذیرش در زمان چندجمله‌ای»
\fn{accepted in polynomial time}
گفته می‌شود اگر:
\item[1]
توسط الگوریتم A پذیرفته شود، و
\item[2]
یک ثابت k وجود داشته باشد به‌طوری‌که برای هر رشته‌ی $x \in L$ با طول n، الگوریتم A رشته‌ی x را در زمان $O(n^k)$ بپذیرد.

\end{itemframe}

%----------------------------------------------------
\begin{itemframe}{مروری بر نظریه زبان‌ها صوری}
\itm
به‌طور مشابه، به یک زبان «قابل تصمیم‌گیری در زمان چندجمله‌ای»
\fn{decided in polynomial}
 گفته می‌شود اگر یک ثابت k وجود داشته باشد به‌طوری‌که برای هر رشته‌ی
$x \in {0,1}^*$
با طول n، الگوریتم در زمان $O(n^k)$ به‌درستی تشخیص دهد که آیا $x \in L$ است یا نه.
\itm
در نتیجه، برای «پذیرفتن» یک زبان، یک الگوریتم کافیست تنها برای رشته‌هایی که در زبان موجود اند پاسخ تولید کند.
اما برای «تصمیم‌گیری»، باید برای تمام رشته‌های دودویی در
${0,1}^*$
به‌درستی تشخیص دهد که آیا عضو L هستند یا نه.
\itm
برای مثال، زبان PATH قابل پذیرش و قابل تصمیم گیری در زمان چند جمله‌ایی است. اما برای برخی مسائل دیگر، مانند مسئله توقف تورینگ، الگوریتم پذیرنده وجود دارد، اما هیچ الگوریتم تصمیم‌گیرنده‌ای وجود ندارد.
\end{itemframe}
