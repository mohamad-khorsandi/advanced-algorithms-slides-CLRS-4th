
\begin{itemframe-s}{یادآوری}{تعریف گراف}
\itm
گراف‌ها به دو دسته جهت‌دار و بدون جهت تقسیم می‌شوند. این دسته بندی از این جهت مهم است که گراف‌های جهت‌دار و بدون جهت به شکل \textbf{مجزا} تعریف می‌شوند و تفاوت‌هایی در تعریف آنها وجود دارد؛
\item[۱]
گراف بدون جهت: طبق تعریف این گراف نمی‌تواند طوقه یا یال موازی داشته باشد.
\item[۲]
گراف جهت‌دار: طبق تعریف این گراف می‌تواند طوقه داشته باشد اما نمی‌تواند یال موازی داشته باشد.

\end{itemframe-s}

\begin{itemframe-s}{یادآوری}{تعریف گراف}
\itm
البته دو یال بین دو رأس یکسان در صورتی که در جهت مخالف یکدیگر باشند موازی محسوب \textbf{نمی‌شوند}. برای مثال شکل زیر یک گراف فاقد یال موازی است.
\centerimg[.2]{figs/shortest-path/1.png}
\itm
 به این یال‌ها پادموازی
\fn{antiparallel}
 گفته می‌شود. بنابراین گراف جهت دار \textbf{می‌تواند} یال پادموازی داشته باشد.

\itm
در هر یک از مسائل بسته به ذات مسئله نوع خاصی از گراف به عنوان ورودی در نظر گرفته می‌شود. برای مثال ورودی مسئله کوتاه ترین مسیر در حالت کلی یک گراف جهت دار و وزن دار است.
\end{itemframe-s}


\begin{itemframe-s}{یادآوری}{تحلیل الگوریتم‌های گراف}
\itm
الگوریتم‌های گراف برخلاف بیشتر الگوریتم‌هایدارای دو متغییر تاثیر گذار در اندازه ورودی‌اند: تعداد یال‌ها (|E|) و تعداد رئوس (|V|) .
\itm
بر اساس یک قرارداد شناخته‌شده می‌توان در نمادهای مجانبی از قرار دادن نماد اندازه در اطراف V و E صرف‌نظر ‌کرد.

\itm
الگوریتم‌های گراف به طور معمول در دو حالت بررسی می‌شوند:
\item[الف]
زمانی که گراف متراکم باشد: در این حالت فرض میکنیم همه رئوس به هم متصل هستند بنابراین تعداد یال ها از مرتبه
\ath{V^2}
است.
\item[ب]
زمانی که گراف خلوت باشد:‌ در این حالت به طور معمول فرض می‌شود که تعداد یال‌ها از مرتبه
\ath{V}
است.
\end{itemframe-s}


\begin{itemframe-s}{یادآوری}{تحلیل الگوریتم‌های گراف}
\itm
پیچیدگی زمانی ارائه شده برای یک الگوریتم گراف را می‌توان در دو حالت بالا تحلیل کرد. برای مثال تمرین زیر را در نظر بگیرید:
\centerimg[1]{figs/shortest-path/2.png}

\end{itemframe-s}


\begin{itemframe-s}{یادآوری}{تحلیل الگوریتم‌های گراف}
\itm
پیچیدگی زمانی الگوریتم پریم با استفاده از هرم دودویی از مرتبه
$O(E logV+V lgV)$
و با استفاده از هرم فیبوناچی از مرتبه
$O(E+V logV)$
است. با جایگذاری V به جای E در این دو تابع درمی‌یابیم که در گراف خلوت هر دو پیاده‌سازی‌ از لحاظ مجانبی سرعت یکسانی دارند و از مرتبه
$O(Vlg V)$
اند. اما در گراف متراکم پیاده‌سازی با هرم فیبوناچی از لحاظ مجانبی سریع تر و از مرتبه
$O(lgV^2)$
 است.
\itm
چنین تحلیلی در دیگر الگوریتم‌‌های گراف هم کاربرد دارد. برای مثال الگوریتم فلوید-وارشال از مرتبه زمانی
$O(V^3)$
 است. از تحلیل این تابع می‌توان نتیجه گرفت خلوت یا متراکم بودن گراف از نظر مجانبی تاثیری بر سرعت این الگوریتم ندارد.
\end{itemframe-s}



