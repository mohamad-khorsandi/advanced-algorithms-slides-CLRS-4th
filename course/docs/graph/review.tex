%------------------------------------------------------------------------
\begin{itemframe}{یادآوری}{تعریف گراف}
\item[-]
گراف‌ها به دو دسته جهت‌دار و بدون جهت تقسیم می‌شوند. این دسته بندی از این جهت مهم است که گراف‌های جهت‌دار و بدون جهت به شکل \textbf{مجزا} تعریف می‌شوند و تفاوت‌هایی در تعریف آنها وجود دارد؛
\item[۱]
گراف بدون جهت: طبق تعریف این گراف نمی‌تواند طوقه یا یال موازی داشته باشد.
\item[۲]
گراف جهت‌دار: طبق تعریف این گراف می‌تواند طوقه داشته باشد اما نمی‌تواند یال موازی داشته باشد.

\end{itemframe}
%------------------------------------------------------------------------
\begin{itemframe}{یادآوری}{تعریف گراف}
\item[-]
البته دو یال بین دو رأس یکسان در صورتی که در جهت مخالف یکدیگر باشند موازی محسوب \textbf{نمی‌شوند}. برای مثال شکل زیر یک گراف فاقد یال موازی است.
\centerimg[.2]{figs/chap01/1.png}
\item[-]
 به این یال‌ها پادموازی
\fn{1}{antiparallel}
 گفته می‌شود. بنابراین گراف جهت دار \textbf{می‌تواند} یال پادموازی داشته باشد.

\item[-]
در هر یک از مسائل بسته به ذات مسئله نوع خاصی از گراف به عنوان ورودی در نظر گرفته می‌شود. برای مثال ورودی مسئله کوتاه ترین مسیر در حالت کلی یک گراف جهت دار و وزن دار است.
\end{itemframe}

%------------------------------------------------------------------------
\begin{itemframe}{یادآوری}{تحلیل الگوریتم‌های گراف}
\item[-]
الگوریتم‌های گراف برخلاف بیشتر الگوریتم‌هایدارای دو متغییر تاثیر گذار در اندازه ورودی‌اند: تعداد یال‌ها (|E|) و تعداد رئوس (|V|) .
\item[-]
بر اساس یک قرارداد شناخته‌شده می‌توان در نمادهای مجانبی از قرار دادن نماد اندازه در اطراف V و E صرف‌نظر ‌کرد.

\item[-]
الگوریتم‌های گراف به طور معمول در دو حالت بررسی می‌شوند:
\item[الف]
زمانی که گراف متراکم باشد: در این حالت فرض میکنیم همه رئوس به هم متصل هستند بنابراین تعداد یال ها از مرتبه
\ath{V^2}
است.
\item[ب]
زمانی که گراف خلوت باشد:‌ در این حالت به طور معمول فرض می‌شود که تعداد یال‌ها از مرتبه
\ath{V}
است.
\end{itemframe}

%------------------------------------------------------------------------
\begin{itemframe}{یادآوری}{تحلیل الگوریتم‌های گراف}
\item[-]
پیچیدگی زمانی ارائه شده برای یک الگوریتم گراف را می‌توان در دو حالت بالا تحلیل کرد. برای مثال تمرین زیر را در نظر بگیرید:
\centerimg[1]{figs/chap01/2.png}

\end{itemframe}

%------------------------------------------------------------------------
\begin{itemframe}{یادآوری}{تحلیل الگوریتم‌های گراف}
\item[-]
پیچیدگی زمانی الگوریتم پریم با استفاده از هرم دودویی از مرتبه
\m{O(E logV+V lgV)}
و با استفاده از هرم فیبوناچی از مرتبه
\m{O(E+V logV)}
است. با جایگذاری V به جای E در این دو تابع درمی‌یابیم که در گراف خلوت هر دو پیاده‌سازی‌ از لحاظ مجانبی سرعت یکسانی دارند و از مرتبه
\m{O(Vlg V)}
اند. اما در گراف متراکم پیاده‌سازی با هرم فیبوناچی از لحاظ مجانبی سریع تر و از مرتبه
\m{O(lgV^2)}
 است.
\item[-]
چنین تحلیلی در دیگر الگوریتم‌‌های گراف هم کاربرد دارد. برای مثال الگوریتم فلوید-وارشال از مرتبه زمانی
\m{O(V^3)}
 است. از تحلیل این تابع می‌توان نتیجه گرفت خلوت یا متراکم بودن گراف از نظر مجانبی تاثیری بر سرعت این الگوریتم ندارد.
\end{itemframe}



