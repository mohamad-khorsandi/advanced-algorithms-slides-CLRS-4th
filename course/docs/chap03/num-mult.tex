
\begin{frame}{‌ضرب اعداد}
\begin{itemize}\itemr
\item[-]
می‌خواهیم حاصلضرب دو عدد
\m{u}
و
\m{v}
را محاسبه کنیم.
\item[-]
با استفاده از روش تقسیم و حل، هریک از اعداد را به دو قسمت تقسیم کرده و با استفاده از حاصل‌ضرب قسمت‌های کوچک‌تر، ضرب دو عدد را محاسبه می‌کنیم.
\item[-]
اگر عدد
\m{u}
یک عدد
\m{n}
رقمی باشد می‌توانیم بنویسیم :
\begin{align*}
\m{u = x \times 10^m + y}
\end{align*}
به طوری که
\m{m = \lfloor \frac{n}{2} \rfloor}
است و
\m{x}
یک عدد
\m{\lceil \frac{n}{2} \rceil}
رقمی و
\m{y}
یک عدد
\m{\lfloor \frac{n}{2} \rfloor}
رقمی است.
\end{itemize}
\end{frame}


\begin{frame}{‌ضرب اعداد}
\begin{itemize}\itemr
\item[-]
اگر دو عدد
\m{n}
رقمی
\m{u}
و
\m{v}
داشته باشیم، می‌توانیم بنویسیم :
\begin{align*}
& \m{u = x \times 10^m + y}\\
& \m{v = w \times 10^m + z}
\end{align*}
\item[-]
ضرب این دو عدد برابر است با :
\begin{align*}
& \m{uv = (x \times 10^m + y)(w \times 10^m + z)}\\
& \m{~~~~= xw \times 10^{2m} + (xz + yw) \times 10^m + yz}
\end{align*}
\item[-]
حال برای ضرب دو عدد با اندازه
\m{n}
باید ۴ ضرب برروی اعدادی با اندازه
\m{\frac{n}{2}}
انجام دهیم.
\end{itemize}
\end{frame}


\begin{frame}{‌ضرب اعداد}
\begin{itemize}\itemr
\item[-]
عملیات جمع در زمان خطی انجام می‌شود، بنابراین پیچیدگی زمانی ضرب دو عدد با استفاده از تقسیم و حل برابر است با :
\begin{align*}
\m{T(n) = 4 T(\frac{n}{2}) + O(n)}
\end{align*}
\item[-]
با حل این رابطه به دست می‌آوریم
\m{T(n) = \ath{n^2}}.
\end{itemize}
\end{frame}


\begin{frame}{‌ضرب اعداد}
\begin{itemize}\itemr
\item[-]
پیچیدگی زمانی الگوریتم ضرب دو عدد
\m{n}
رقمی
\m{O(n^2)}
است زیرا تعداد
\m{n^2}
عملیات ضرب باید انجام شود.
\item[-]
بنابراین با روش تقسیم و حل هیچ بهبودی در پیچیدگی زمانی الگوریتم ضرب حاصل نشده است.
\item[-]
در روش تقسیم و حل، ضرب دو عدد
\m{n}
رقمی با استفاده از ۴ ضرب اعداد
\m{\frac{n}{2}}
رقمی انجام شد. اگر بتوانیم ضرب‌ها را کاهش دهیم، می‌توانیم پیچیدگی زمانی الگوریتم را بهبود دهیم.
\end{itemize}
\end{frame}


\begin{frame}{‌ضرب اعداد}
\begin{itemize}\itemr
\item[-]
توجه کنید که برای محاسبه ضرب دو عدد نیاز به محاسبه
\m{xw}
،
\m{(xz + yw)}
و
\m{yz}
داشتیم که برای محاسبه آن چهار عملیات ضرب نیاز بود.
\item[-]
به جای انجام چهار ضرب می‌توانیم با استفاده از یک عمل ضرب مقدار
\m{r}
را به صورت زیر محاسبه کنیم.
\begin{align*}
\m{r = (x + y)(w + z) = xw + (xz + yw) + yz}
\end{align*}
\item[-]
بنابراین مقدار
\m{xz + yw}
را می‌توانیم به صورت زیر محاسبه کنیم.
\begin{align*}
\m{xz + yw = r - xw - yz}
\end{align*}
\end{itemize}
\end{frame}


\begin{frame}{‌ضرب اعداد}
\begin{itemize}\itemr
\item[-]
پس برای محاسبه سه مقدار
\m{xw}
،
\m{(xz + yw)}
و
\m{yz}
نیاز داریم سه عملیات ضرب
\m{(x+y)(w+z)}
و
\m{xw}
و
\m{yz}
را انجام دهیم.
\item[-]
پیچیدگی زمانی این الگوریتم برابر است با :
\begin{align*}
\m{T(n) = 3T(\frac{n}{2}) + O(n)}
\end{align*}
\item[-]
با محاسبه این رابطه بازگشتی به دست می‌آوریم :
\begin{align*}
\m{T(n) = O(n^{\lg 3}) = O(n^{1.58})}
\end{align*}
\end{itemize}
\end{frame}


\begin{frame}{‌ضرب اعداد}
\begin{itemize}\itemr
\item[-]
این الگوریتم در سال ۱۹۶۰ توسط آناتولی کاراستوبا
\fn{1}{Anatoly Karatsuba}
ریاضی‌دان روسی ابداع شد و الگوریتم کاراتسوبا نامیده می‌شود.
\item[-]
در سال ۲۰۱۹ یک الگوریتم از مرتبه
\m{O(n \lg n)}
برای ضرب اعداد معرفی شده است که بر پایه الگوریتم شونهاج-استراسن
\fn{2}{Schonhage-Strassen algerithm}
و تبدیل‌های فوریه سریع بنا نهاده شده است.
\end{itemize}
\end{frame}