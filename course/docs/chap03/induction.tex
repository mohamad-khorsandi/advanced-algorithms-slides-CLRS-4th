
\begin{frame}{‌طراحی الگوریتم با استقرا}
\begin{itemize}\itemr
\item[-]
استقرای ریاضی
\fn{1}{induction}
روشی است برای اثبات درستی گزاره
\m{P(n)}
برای همه اعداد طبیعی
\m{n}
. به عبارت دیگر هنگامی که می‌خواهیم درستی گزاره‌های
\m{P(1)}
،
\m{P(2)}
،
\m{\cdots}
،
\m{P(n)}
را ثابت کنیم، می‌توانیم از استقرا استفاده کنیم.
\item[-]
به زبان استعاری با استفاده از استقرا ثابت می‌کنیم که می‌توانیم هر نردبانی را با طول دلخواه یا بینهایت بالا برویم اگر ثابت کنیم که می‌‌توانیم برروی پله اول برویم (پایهٔ استقرا
\fn{2}{base case}
) و همچنین ثابت کنیم اگر برروی پلهٔ
\m{n}
بودیم می‌توانیم برروی پله
\m{n+1}
نیز گام بگذاریم (گام استقرا
\fn{3}{induction step}
).
\item[-]
بنابراین در روش استقرایی برای اثبات درستی
\m{P(n)}
باید ثابت کنیم
\m{P(1)}
درست است (پایهٔ استقرا) و همچنین اگر
\m{P(n)}
درست باشد، آنگاه
\m{P(n+1)}
نیر درست است (گام استقرا).
\end{itemize}
\end{frame}


\begin{frame}{‌طراحی الگوریتم با استقرا}
\begin{itemize}\itemr
\item[-]
استقرای ریاضی براساس اصل دومینو
\fn{1}{domino principle}
است. فرض کنید تعداد زیادی دومینو به صورت ایستاده در کنار یکدیگر قرار گرفته‌اند و می‌خواهیم همهٔ دومینوهای ایستاده را بیاندازیم. برای اینکه همهٔ دومینوها بر زمین بیفتند کافی است دومینوها به گونه‌ای قرار داده شوند که با افتادن اولین دومینو، دومین دومینو برزمین بیافتد و با افتادن دومی، سومی و به همین ترتیب با افتادن
\m{n}
امین دومینو،
\m{n+1}
امین دومینو بر زمین بیافتد. سپس کافی است به اولین دومینو ضربه‌ای بزنیم تا همه دومینوهای ایستاده بیافتند و نیازی به انداختن تک‌تک آنها نداریم.
\end{itemize}
\end{frame}

\iffalse
\begin{frame}{‌طراحی الگوریتم با استقرا}
\begin{itemize}\itemr
\item[-]
در طراحی یک مسئله به روش استقرایی، باید برای پاسخ مسئله یک رابطه پیدا کنیم و پاسخ مسئله را به روش استقرایی اثبات کنیم. و سپس می‌توانیم از رابطهٔ به دست آمده برای حل مسئله استفاده کنیم.
\item[-]
برای مثال فرض کنید می‌خواهیم جمع
\m{n}
عدد اول صحیح را به دست آوریم. برای این کار
\m{n}
عدد را با یکدیگر جمع کنیم. پس الگوریتم در واقع
\m{O(n)}
است.
\item[-]
برای حل این مسئله به روش استقرایی باید رابطه‌ای برای جواب مسئله پیدا کنیم. به عبارت دیگر آیا عبارتی وجود دارد که توسط آن بتوان جمع
\m{n}
عدد اول اعداد صحیح را به دست آورد؟
\end{itemize}
\end{frame}
\fi

\begin{frame}{‌طراحی الگوریتم با استقرا}
\begin{itemize}\itemr
\item[-]
برای مثال با استفاده از استقرا می‌توان اثبات کرد:
\begin{align*}
\m{P(n) = 1 + 2 + 3 + \cdots + n = \frac{n(n+1)}{2}}
\end{align*}
\item[-]
باید اثبات کنیم
\m{P(1) = \frac{1(2)}{2}}
درست است (پایهٔ استقرا) و همچنین اگر
\m{P(n) = \frac{n(n+1)}{2}}
باشد آنگاه
\m{P(n+1) = \frac{(n+1)(n+2)}{2}}
نیز درست است (گام استقرا).
\end{itemize}
\end{frame}


\begin{frame}{‌طراحی الگوریتم با استقرا}
\begin{itemize}\itemr
\item[-]
اثبات : 
\item[-]
پایه استقرا درست است زیرا 
\m{P(1) = 1 = \frac{1(2)}{2} = 1}
\item[-]
می‌دانیم
\m{P(n+1) = P(n) + (n+1)}
بنابراین
\m{P(n+1) = \frac{n(n+1)}{2} + (n+1)} .
با بسط این رابطه به دست می‌آوریم
\m{P(n+1) = \frac{(n+1)(n+2)}{2}} .
بنابراین گام استقرا نیز درست است.
\item[-]
با استفاده از این رابطه برای محاسبهٔ
\m{n}
عدد کافی است از رابطه
\m{P(n)}
استفاده کنیم. این الگوریتم در زمان
\m{O(1)}
انجام می‌شود، در حالی که جمع 
\m{n}
عدد با استفاده از یک حلقه در زمان
\m{O(n)}
انجام می‌شود.
\end{itemize}
\end{frame}



\begin{frame}{‌طراحی الگوریتم با استقرا}
\begin{itemize}\itemr
\item[-]
استقرای ریاضی در طراحی الگوریتم‌ها بسیار پر استفاده است.
\item[-]
برای طراحی یک الگوریتم برای حل یک مسئله با استفاده از استقرا کافی است :
\item[۱.]
مسئله را در حالت پایه یعنی حالتی که اندازه ورودی کوچک است حل کنیم.
\item[۲.]
نشان‌دهیم چگونه می‌توان یک مسئله را با استفاده از یک زیر مسئله (یعنی مسئله‌ای با اندازهٔ کوچک‌تر) حل کرد.
\end{itemize}
\end{frame}


\begin{frame}{‌طراحی الگوریتم با استقرا}
\begin{itemize}\itemr
\item[-]
فرض کنید می‌خواهیم به ازای دنباله‌ای از اعداد حقیقی
\m{a_0}
،
\m{a_1}
،
\m{a_2}
،
\m{\cdots}
،
\m{a_n}
و عدد داده شده
\m{x}،
مقدار چند جمله‌ای زیر را محاسبه کنیم.\\
\begin{align*}
\m{P_n(x) = a_nx^n + a_{n-1}x^{n-1} + \cdots + a_1x + a_0}
\end{align*}
\end{itemize}
\end{frame}


\begin{frame}{‌طراحی الگوریتم با استقرا}
\begin{itemize}\itemr
\item[-]
یک الگوریتم بدیهی برای حل این مسئله با جایگذاری اعداد
\m{a_i}
و
\m{x}
در چند جمله
\m{P_n(x)}
مقدار آن را محاسبه می‌کند.
\begin{algorithm}[H]\alglr
\caption{Compute Polynomial}
  \begin{algorithmic}[1]
   \Func{ComputePolynomial}{a[], x}
    \State P = a[0]
    \For{i = 1 to n}
      \State X = 1
      \For{j = 1 to i}
          \State X = X * x
       \EndFor
       \State P = P + a[i] * X
     \EndFor
     \State \Return P                          
  \end{algorithmic}
  \label{alg:merge}
\end{algorithm}
\item[-]
پیچیدگی زمانی این الگوریتم
\m{O(n^2)}
است.
\item[-]
حال می‌خواهیم با استفاده از استقرا این مسئله را در زمان کمتری حل کنیم.
\end{itemize}
\end{frame}


\begin{frame}{‌طراحی الگوریتم با استقرا}
\begin{itemize}\itemr
\item[-]
برای حل مسئله با استفاده از استقرا باید بتوانیم مسئله را بر اساس یک زیر مسئله بیان کنیم.
\item[-]
یک زیر مسئله از مسئلهٔ محاسبه چند جمله‌ای را به صورت زیر در نظر بگیرید.
\begin{align*}
\m{P_{n-1}(x) = a_nx^{n-1} + a_{n-1}x^{n-2} + \cdots + a_1}
\end{align*}
\item[-]
فرض کنید جواب
\m{P_{n-1}(x)}
داده شده است. چگونه می‌توانیم
\m{P_n(x)}
را محاسبه کنیم؟
\end{itemize}
\end{frame}


\begin{frame}{‌طراحی الگوریتم با استقرا}
\begin{itemize}\itemr
\item[-]
برای محاسبه
\m{P_n(x)}
می‌توانیم رابطه‌ای به صورت زیر بنویسیم.
\begin{align*}
\m{P_n(x) = x \cdot P_{n-1}(x) + a_0}
\end{align*}
\item[-]
همچنین می‌توانیم
\m{P_{n-1}(x)}
را بر اساس
\m{P_{n-2}(x)}
محاسبه کنیم.
\item[-]
داریم :
\begin{align*}
\m{P_{n-2}(x) = a_nx^{n-2} + a_{n-1}x^{n-3} + \cdots + a_2}
\end{align*}
\item[-]
بنابراین خواهیم داشت :
\begin{align*}
\m{P_{n-1}(x) = x \cdot P_{n-2}(x) + a_1}
\end{align*}
\end{itemize}
\end{frame}


\begin{frame}{‌طراحی الگوریتم با استقرا}
\begin{itemize}\itemr
\item[-]
در حالت کلی برای محاسبه
\m{P_{n-j}(x)}
با استفاده از یک زیرمسئله می‌توانیم رابطهٔ زیر را ارائه کنیم:
\begin{align*}
\m{P_{n-j}(x) = x \cdot P_{n-(j+1)}(x) + a_j}
\end{align*}
\item[-]
در حالت پایه داریم:
\begin{align*}
\m{P_0(x) = a_n}
\end{align*}
\end{itemize}
\end{frame}


\begin{frame}{‌طراحی الگوریتم با استقرا}
\begin{itemize}\itemr
\item[-]
فرض کنیم
\m{n - j = i}
، در اینصورت خواهیم داشت :
\begin{align*}
\left\{ \begin{array}{lcl}
\m{P_i(x) = x \cdot P_{i-1}(x) + a_{n-i}} & \m{i > 0} & \text{اگر} \\
\m{P_0(x) = a_n} & \m{i = 0} & \text{اگر}
\end{array}\right.
\end{align*}
\end{itemize}
\end{frame}


\begin{frame}{‌طراحی الگوریتم با استقرا}
\begin{itemize}\itemr
\item[-]
بنابراین با استفاده از رابطه بازگشتی به دست آمده می‌توانیم الگوریتمی به صورت زیر بنویسیم.
\begin{algorithm}[H]\alglr
\caption{Compute Polynomial}
  \begin{algorithmic}[1]
   \Func{ComputePolynomial}{a[], x}
    \State P = a[n]
    \For{i = 1 to n}
       \State P = x * P + a[n-i]
     \EndFor  
     \State \Return P                           
  \end{algorithmic}
  \label{alg:merge}
\end{algorithm}
\item[-]
پیچیدگی زمانی این الگوریتم
\m{O(n)}
است که از الگوریتم بدیهی که در زمان
\m{O(n^2)}
چند جمله‌ای را محاسبه می‌کند سریع‌تر است.
\end{itemize}
\end{frame}


\begin{frame}{‌طراحی الگوریتم با استقرا}
\begin{itemize}\itemr
\item[-]
این الگوریتم به روش هورنر
\fn{1}{Horner's method}
معروف است که توسط ریاضی‌دان انگلیسی ویلیام هورنر
\fn{2}{William Horner}
ابداع شده است، گرچه خود هورنر آن را به ریاضی‌دان فرانسوی-ایتالیایی ژوزف لاگرانژ
\fn{3}{Joseph-Louis Lagrange}
نسبت داده است. گفته می‌شود این الگوریتم قبل از لاگرانژ احتمالاً توسط ریاضی‌دانان ایرانی و چینی ابداع شده است.
\begin{align*}
&\m{a_0 + a_1x + a_2x^2 + \cdots + a_nx^n =} \\
&~~~~~~~~~~~~~~~~~~~~~~~~~~~~~~~\m{a_0 + x ( a_1 + x (a_2 + x ( a_3 + \cdots + x (a_{n-1} + xa_n) \cdots)))}
\end{align*}
\end{itemize}
\end{frame}