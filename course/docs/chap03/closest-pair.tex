
\begin{frame}{‌نزدیک‌ترین زوج نقطه}
\begin{itemize}\itemr
\item[-]
مسئلهٔ یافتن نزدیک‌ترین زوج نقطه
\fn{1}{closest pair of points}
در بین تعدادی نقاط در صفحه یکی دیگر از مسائلی است که با استفاده از روش تقسیم و حل قابل حل است.
\item[-]
این مسئله در حوزه‌های متعددی کاربرد دارد. به طور کلی هر موجودی را می‌توان با استفاده از تعدادی ویژگی مدلسازی کرد. این ویژگی‌ها یک نقطه در یک فضای چند بعدی ایجاد می‌کنند. حال برای پیدا کردن دو موجود با ویژگی‌های مشابه باید در فضای چند بعدی تشکیل شده، به دنبال نزدیک‌ترین زوج نقطه‌ها بگردیم.
\item[-]
مسئلهٔ یافتن نزدیک‌ترین زوج نقطه را ابتدا در فضای یک بعدی و سپس در فضای دو بعدی بررسی می‌کنیم.
\end{itemize}
\end{frame}


\begin{frame}{‌نزدیک‌ترین زوج نقطه}
\begin{itemize}\itemr
\item[-]
فرض کنید مجموعهٔ S حاوی n نقطه در فضای یک بعدی بر روی یک محور داده شده است. می‌خواهیم از بین این n نقطه دو نقطه پیدا کنیم که فاصلهٔ آن دو به یکدیگر از فاصلهٔ هر زوج نقطهٔ دیگری کمتر باشد. به عبارت دیگر می‌خواهیم نزدیک‌ترین زوج نقطه را در مجموعهٔ S پیدا کنیم.
\item[-]
در یک الگوریتم ابتدایی باید فاصلهٔ بین همهٔ زوج نقطه‌ها را محاسبه کنیم و از بین فاصله‌های محاسبه شده، کوتاهترین فاصله و نقاط متناظر با آن فاصله‌ را پیدا کنیم. از آنجایی که تعداد
\m{n^2}
زوج نقطه وجود دارد، زمان مورد نیاز برای محاسبهٔ همهٔ فاصله‌ها و حل مسئله
\m{O(n^2)}
است.
\end{itemize}
\end{frame}


\begin{frame}{‌نزدیک‌ترین زوج نقطه}
\begin{itemize}\itemr
\item[-]
با استفاده از روش تقسیم و حل، این مسئله را در زمان کمتری حل می‌کنیم.
\item[-]
تقسیم : نقاط مجموعه S را بر روی محور مختصات x مرتب می‌کنیم. سپس مجموعهٔ S حاوی n نقطه را به دو مجموعه
\m{S_1}
و
\m{S_2}
هر کدام شامل
\m{n/2}
نقطه تقسیم می‌کنیم.
\item[-]
حل : الگوریتم را به طور بازگشتی برای مجموعهٔ
\m{S_1}
و
\m{S_2}
فراخوانی می‌کنیم.
\item[-]
ترکیب : نزدیک‌ترین زوج نقطه در بین نقاط مجموعهٔ
\m{S}
یا
در مجموعهٔ
\m{S_1}
است و یا
\m{S_2}
و یا زوج نقطه‌ای که از اولین نقطه
\m{S_2}
و آخرین نقطهٔ
\m{S_1}
تشکیل شده است. پس در بین این سه حالت باید نزدیک‌ترین زوج نقطه محاسبه و بازگردانده شود.
\item[-]
حالت پایه : در حالت پایه مجموعهٔ S شامل ۲ نقطه است که فاصلهٔ بین این دو نقطه بازگردانده می‌شود.
\end{itemize}
\end{frame}


\begin{frame}{‌نزدیک‌ترین زوج نقطه}
\begin{itemize}\itemr
\item[-]
این الگوریتم را می‌توانیم به صورت زیر بنویسیم.
\begin{algorithm}[H]\alglr
  \caption{Closest Pair of One-Dimensional Points} 
  \begin{algorithmic}[1]
   \Func{Closest-Pair-1D}{S} \LeftComment{ S is already sorted}
    \If {|S| == 2}
       \State \Return d = p2 - p1
     \EndIf
     \State Divide S from the mid-point and conxtruct S1 and S2
     \State d1 = Closest-Pair-1D(S1)
     \State d2 = Closest-Pair-1D(S2)
     \State p1 = max(S1)
     \State p2 = min(S2)
     \State d = min \{d1, d2, p2-p1\}   
     \State \Return d                 
  \end{algorithmic}
  \label{alg:merge}
\end{algorithm}
\end{itemize}
\end{frame}


\begin{frame}{‌نزدیک‌ترین زوج نقطه}
\begin{itemize}\itemr
\item[-]
برای محاسبهٔ زمان اجرای این الگوریتم، زمان اجرای مرحله تقسیم و ترکیب را محاسبه می‌کنیم که هر دو در زمان ثابت انجام می‌شوند.
برای محاسبه زمان اجرا از رابطهٔ بازگشتی
\m{T(n) = 2T(n/2) + O(1)}
استفاده می‌کنیم، بنابراین به دست می‌آوریم
\m{T(n) = O(nlgn)} .
\end{itemize}
\end{frame}


\begin{frame}{‌نزدیک‌ترین زوج نقطه}
\begin{itemize}\itemr
\item[-]
الگوریتم یافتن نزدیک‌ترین زوج نقطه در فضای دو بعدی به طور مشابه با الگوریتم یافتن نزدیک‌ترین زوج نقطه در فضای یک بعدی عمل می‌کند.
\item[-]
الگوریتم تقسیم و حل به صورت زیر عمل می‌کند.
\item[-]
تقسیم : در مرحله تقسیم یک خط عمودی در فضای دو بعدی رسم می‌کنیم که مجموعهٔ S را از وسط به دو زیر مجموعهٔ
\m{S_1}
و
\m{S_2}
تقسیم می‌کند.
\item[-]
حل : نزدیک‌ترین زوج نقطه را در بین نقاط مجموعهٔ
\m{S_1}
و
\m{S_2}
به طور بازگشتی پیدا می‌کنیم. فرض کنیم فاصلهٔ نزدیک‌ترین زوج نقطه در مجموعهٔ
\m{S_1}
برابر با
\m{d_1}
و در مجموعهٔ
\m{S_2}
برابر با
\m{d_2}
باشد.
\item[-]
ترکیب : فاصلهٔ نزدیک‌ترین زوج نقطه در مجموعهٔ S یا برابر است با
\m{d_1}
یا
\m{d_2}
و یا برابر است با فاصله یکی از زوج نقطه‌ها که یکی از آنها در
\m{S_1}
و دیگری در
\m{S_2}
است.
\end{itemize}
\end{frame}


\begin{frame}{‌نزدیک‌ترین زوج نقطه}
\begin{itemize}\itemr
\item[-]
برای پیدا کردن فاصلهٔ زوج نقطه‌هایی که یکی از آنها در
\m{S_1}
و دیگری در
\m{S_2}
است، سه خط عمودی را در نظر بگیرید. خط عمودی اول طوری رسم شده است که دقیقا وسط مجموعهٔ نقاط قرار گرفته است یعنی فاصلهٔ آن از نقطه‌ای که مقدار x آن کمترین است برابر است با فاصله آن خط از نقطه‌ای که مقدار x آن بیشترین است. سپس دو خط موازی دیگر با این خط وسط در نظر بگیرید که فاصله هر کدام از آن‌ها از خط وسط برابر با 
\m{d=min(d_1, d_2)}
 باشد. بدین ترتیب یک نوار با عرض ۲d تشکیل داده‌ایم.
\item[-]
توجه کنید هیچ زوج نقطه‌ای که خارج از نوار و یکی از آنها در 
\m{S_1}
و دیگری در 
\m{S_2}
است،
وجود نخواهد داشت که فاصلهٔ آنها از d کمتر باشد.
بنابراین باید بررسی کنیم آیا در این نوار زوج نقطه‌ای وجود دارد که فاصله‌اش از d کمتر است یا خیر.
\item[-]
می‌توان به طور هندسی اثبات کرد که برای این کار کافی است از بالاترین نقطه شروع ‌کنیم و فاصلهٔ هر نقطه را با ۷ نقطهٔ بعدی مقایسه کنیم که این کار در زمان
\m{O(n)}
امکان پذیر است.
\item[-]
بنابراین زمان اجرای این الگوریتم را می‌توانیم از رابطه
\m{T(n) = 2T(n/2) + O(n)}
محاسبه کنیم که به دست می‌آوریم
\m{T(n) = O(nlgn)}.
\end{itemize}
\end{frame}



\begin{frame}{‌نزدیک‌ترین زوج نقطه}
\begin{itemize}\itemr
\item[-]
برای اینکه نشان دهیم هر نقطه را در نوار میانی حداکثر با ۷ نقطه باید مقایسه کنیم، ابتدا نوار میانی را به مربع‌هایی با ابعاد
\m{n/2 \times n/2}
تقسیم می‌کنیم.
\item[-]
سپس نشان می‌دهیم که در یک مربع با ابعاد 
\m{n/2 \times n/2}
نمی‌تواند بیش از یک نقطه وجود داشته باشد، چرا که وجود بیش از دو نقطه در مربعی با این ابعاد نشان دهندهٔ این است که دو نقطه در یک طرف مرز میانی فاصله ای کمتر از d دارند و این با توجه به مفروضات غیرممکن است.
\item[-]
سپس نشان می‌دهیم در بدترین حالت تعداد نقاطی که در هر یک از مربع‌های کوچک قرار دارند و فاصلهٔ آنها از یک نقطه معین می‌تواند کمتر از d باشد، ۷ است.
\end{itemize}
\end{frame}

