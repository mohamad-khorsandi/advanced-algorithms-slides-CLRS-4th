
\begin{frame}{‌روش جایگذاری}
\begin{itemize}\itemr
\item[-]
روش جایگذاری برای حل روابط بازگشتی از دو گام تشکیل شده است. در گام اول جواب رابطهٔ بازگشتی یا عبارت فرم بسته
\fn{1}{closed-form expression}
که در رابطهٔ بازگشتی صدق می‌کند حدس زده می‌شود. در گام دوم توسط استقرای ریاضی
\fn{2}{mathematical induction}
اثبات می‌شود که جوابی که حدس زده شده است درست است و در رابطهٔ بازگشتی صدق می‌کند.
\item[-]
برای اثبات توسط استقرای ریاضی، ابتدا باید ثابت کرد که جواب حدس زده شده برای مقادیر کوچک n درست است. سپس باید اثبات کرد که اگر جواب حدس زده شده برای n درست باشد، برای
n+1
نیز درست است. در این روش از جایگذاری جواب حدس زده شده در رابطهٔ اصلی برای اثبات استفاده می‌شود و به همین دلیل روش جایگذاری نامیده می‌شود.
\item[-]
متاسفانه هیچ قاعدهٔ کلی برای حدس زدن جواب رابطهٔ بازگشتی وجود ندارد و یک حدس خوب به کمی تجربه و خلاقیت نیاز دارد.
\end{itemize}
\end{frame}



\begin{frame}{‌روش جایگذاری}
\begin{itemize}\itemr
\item[-]
برای مثال فرض کنید می‌خواهیم رابطهٔ
\m{T(n) = 2 T(n-1)}
و
\m{T(0) = 1}
را حل کنیم.
\item[-]
این رابطه را برای n های کوچک می‌نویسیم و حدس می‌زنیم 
\m{T(n) = 2^n}
باشد.
\item[-]
سپس رابطه را با استفاده از استقرا اثبات می‌کنیم.
\end{itemize}
\end{frame}


\begin{frame}{‌روش جایگذاری}
\begin{itemize}\itemr
\item[-]
در برخی مواقع یک رابطهٔ بازگشتی شبیه رابطه‌هایی است که جواب آنها را می‌دانیم و در چنین مواقعی می‌توانیم از حدس استفاده کنیم.
\item[-]
برای مثال رابطه
\m{T(n) = 2 T(n/2 + 17) + \ath{n}}
را در نظر بگیرید. شبیه این رابطه را بدون عدد ۱۷ قبلا دیده‌ایم اما می‌توانیم حدس بزنیم که این عدد برای n های بزرگ تأثیر زیادی ندارد. پس حدس میزنیم که جواب این رابطه
\m{T(n) = O(n \lg n)}
باشد.
\item[-]
یک روش دیگر برای حدس زدن این است که ابتدا یک کران پایین حدس زده و سپس کران پایین را افزایش دهیم تا به جواب واقعی نزدیک شویم.
\end{itemize}
\end{frame}