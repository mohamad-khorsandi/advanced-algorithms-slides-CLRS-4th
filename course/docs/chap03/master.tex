
\begin{frame}{‌روش قضیه اصلی}
\begin{itemize}\itemr
\item[-]
روش قضیه اصلی
\fn{1}{master theorem method}
برای حل مسائل بازگشتی استفاده می‌شود که به صورت
\m{T(n) = aT(n/b) + f(n)}
هستند به طوری که
\m{a > 0}
و
\m{b > 1}
دو ثابت هستند.
\item[-]
تابع
\m{f(n)}
در اینجا تابع محرک
\fn{2}{driving function}
 نامیده می‌شود و یک رابطهٔ بازگشتی که به شکل مذکور است، رابطهٔ بازگشتی اصلی
\fn{3}{master recurrence}
  نامیده می‌شود.
\item[-]
در واقع رابطهٔ بازگشتی اصلی زمان اجرای الگوریتم‌های تقسیم و حل را توصیف می‌کند که مسئله‌ای به اندازهٔ n را به a زیر مسئله هر کدام با اندازهٔ
\m{n/b}
تقسیم می‌کنند. تابع
\m{f(n)}
هزینه تقسیم مسئله به زیر مسئله‌ها به علاوه هزینه ترکیب زیر مسئله‌ها را نشان می‌دهد.
\item[-]
اگر یک رابطهٔ بازگشتی شبیه رابطه قضیه اصلی باشد و علاوه بر آن چند عملگر کف و سقف در آن وجود داشته باشد، همچنان می‌توان از رابطهٔ قضیه اصلی استفاده کرد.
\end{itemize}
\end{frame}


\begin{frame}{‌روش قضیه اصلی}
\begin{itemize}\itemr
\item[-]
قضیه اصلی : فرض کنید
\m{a > 0}
و
\m{b > 1}
دو ثابت باشند و
\m{f(n)}
یک تابع باشد که برای اعداد بسیار بزرگ تعریف شده باشد.
\item[-]
رابطهٔ بازگشتی
\m{T(n)}
که بر روی اعداد طبیعی
\m{n \in \NN}
تعریف شده است را به صورت زیر در نظر بگیرید.
\begin{center}
\m{T(n) = aT(n/b) + f(n)}
\end{center}
\iffalse
به طوری که
\m{aT(n/b)}
برابر است با
\m{a'T( \lfloor n/b \rfloor ) + a"T(\lceil n/b \rceil)}
به ازای ثابت‌های
\m{a' \geqslant 0}
و
\m{a" \geqslant 0}
که در رابطهٔ
\m{a = a' + a"}
صدق می‌کنند.
\fi
\end{itemize}
\end{frame}


\begin{frame}{‌روش قضیه اصلی}
\begin{itemize}\itemr
\item[-]
 رفتار مجانبی
\m{T(n) = aT(n/b) + f(n)}
به صورت زیر است :
\item[۱-]
اگر ثابت
\m{\epsilon > 0}
وجود داشته باشد به طوری‌که
\m{f(n) = O(n^{\log_b^a - \epsilon})}
آنگاه
\m{T(n) = \ath{n^{\log_b^a}}}.
\item[۲-]
اگر ثابت
\m{k \geqslant 0}
وجود داشته باشد به طوری‌که
\m{f(n) = \ath{n^{\log_b^a} \lg^k n}}
آنگاه
\m{T(n) = \ath{n^{\log_b^a} \lg^{k+1}n}}.
\item[۳-]
اگر ثابت
\m{\epsilon > 0}
وجود داشته باشد به طوری‌که
\m{f(n) = \Omega(n^{\log_b^a + \epsilon})}
 آنگاه
\m{T(n) = \ath{f(n)}}.

برای برخی از توابع
\m{f(n)}
 نیاز داریم بررسی کنیم
\m{f(n)}
در رابطهٔ
\m{af(n/b) \leqslant cf(n)}
 به ازای
\m{c < 1}
و 
\m{n}
 های به اندازهٔ کافی بزرگ
صدق کند، اما برای توابعی که در تحلیل الگوریتم‌ها به آنها برمی‌خوریم این شرط معمولا برقرار است.
\end{itemize}
\end{frame}


\begin{frame}{‌روش قضیه اصلی}
\begin{itemize}\itemr
\item[۱-]
در حالت اول رشد جزء بازگشتی از رشد تابع محرک بیشتر است.
به عنوان مثال در
\m{T(n) = 2T(n/2) + \lg n}
رشد جزء بازگشتی
\ath{n}
و رشد تابع محرک
\ath{\lg n}
است.
بنابراین
\m{T(n) = \ath{n}} .
\item[۲-]
در حالت دوم رشد جزء بازگشتی و تابع محرک برابر است و یا رشد تابع محرک با یک ضریب
\ath{\lg^k n}
از جزء بازگشتی سریع‌تر است. به عنوان مثال در
\m{T(n) = 2T(n/2) + n \lg n}
رشد جزء بازگشتی
\ath{n}
و رشد تابع محرک
\ath{n \lg n}
است.
در این حالت تعداد سطوح درخت بازگشت 
\m{\lg n}
و مجموع هزینه‌های هر سطح
\ath{n \lg n}
است. بنابراین 
\m{T(n) = \ath{n \lg^2 n}} .
\item[۳-]
در حالت سوم رشد جزء بازگشتی از رشد تابع محرک کمتر است.
به عنوان مثال در
\m{T(n) = 2T(n/2) + n^2}
رشد جزء بازگشتی
\ath{n}
و رشد تابع محرک
\ath{n^2}
است.
بنابراین
\m{T(n) = \ath{n^2}} .
\end{itemize}
\end{frame}


\begin{frame}{‌روش قضیه اصلی}
\begin{itemize}\itemr
\item[-]
در یک حالت خاص اگر داشته باشیم، 
\m{T(n) = aT(n/b) + cn^k}
آنگاه می‌توانیم اثبات کنیم:
\begin{align*}
\m{T(n)} = \left\{\begin{array}{lr}
          \m{\Theta(n^{\log_b a})}& \m{a > b^k}~\text{اگر}\\
          \m{\Theta(n^k \lg n)}& \m{a = b^k}~\text{اگر}\\
          \m{\Theta(n^k)}& \m{a < b^k}~\text{اگر}
\end{array}\right.
\end{align*}

\end{itemize}
\end{frame}

\begin{frame}{‌روش قضیه اصلی}
\begin{itemize}\itemr
\item[-]
رابطهٔ بازگشتی
\m{T(n) = 9T(n/3) + n}
را در نظر بگیرید. در این رابطه داریم
\m{a = 9}
و
\m{b = 3}
بنابراین به دست می‌آوریم
\m{n^{\log_b^a} = n^{\log_3^9} = \ath{n^2}}.
از آنجایی که
\m{f(n) = n = O(n^{2- \epsilon})}
به ازای هر ثابت
\m{\epsilon < 1}
بنابراین می‌توانیم حالت اول در قضیه اصلی را در نظر بگیریم و نتیجه بگیریم
\m{T(n) = \ath{n^2}}.
\end{itemize}
\end{frame}


\begin{frame}{‌روش قضیه اصلی}
\begin{itemize}\itemr
\item[-]
رابطهٔ بازگشتی
\m{T(n) = T(2n/3) + 1}
را در نظر بگیرید. در این رابطه داریم
\m{a = 1}
و
\m{b = 3/2}
بنابراین
\m{n^{\log_b^a} = n^{\log_{3/2}^1} = n^0 = 1}.
در اینجا حالت دوم در قضیه اصلی را داریم یعنی
\m{f(n) = 1 = \ath{n^{\log_b^a} \lg^0 n} = \ath{1}}
بنابراین جواب رابطهٔ بازگشتی برابر است با
\m{T(n) = \ath{\lg n}}.
\end{itemize}
\end{frame}


\begin{frame}{‌روش قضیه اصلی}
\begin{itemize}\itemr
\item[-]
در رابطهٔ بازگشتی
\m{T(n) = 3T(n/4) + n\lg n}
داریم
\m{a = 3}
و
\m{b = 4}
که بدین معنی است که
\m{n^{\log_b^a} = n^{\log_4^3} = \ath{n^{0.793}}}.
از آنجایی که
\m{f(n) = n \lg n = \Omega (n^{\log_4^3 + \epsilon})}
جایی که
\m{\epsilon}
حدود
\m{0.2}
است، بنابراین حالت سوم در قضیه اصلی را می‌توانیم در نظر بگیریم اگر شرط
\m{af(n/b) \leqslant cf(n)}
برقرار باشد.
\begin{flushleft}
\m{af(n/b) = 3(n/4) \lg(n/4) \leqslant (3/4) n \lg n = 3/4f(n)}
\end{flushleft}
بنابراین با استفاده از حالت سوم جواب رابطهٔ بازگشتی برابراست با
\m{T(n) = \ath{n \lg n}}.
\end{itemize}
\end{frame}


\begin{frame}{‌روش قضیه اصلی}
\begin{itemize}\itemr
\item[-]
رابطهٔ بازگشتی
\m{T(n) = 2T(n/2) + \ath{n}}
رابطه‌ای بود که برای مرتب‌سازی ادغامی به دست آوردیم. از آنجایی که
\m{a = 2}
و
\m{b = 2}
داریم
\m{n^{\log_2^2} = n}.
حالت دوم در اینجا برقراراست زیرا به ازای
\m{k=0}
داریم
\m{f(n) = \ath{n}}
و بنابراین جواب رابطهٔ بازگشتی برابر است با
\m{T(n) = \ath{n \lg n}}.
\end{itemize}
\end{frame}


\begin{frame}{‌روش قضیه اصلی}
\begin{itemize}\itemr
\item[-]
رابطهٔ
\m{T(n) = 8T(n/2) + \ath{1}}
زمان اجرای الگوریتم ضرب ماتریسی را توصیف می‌کند. در اینجا داریم
\m{a = 8}
و
\m{b = 2}
بنابراین
\m{n^{\log_2^8} = n^3}.
تابع محرک
\m{f(n) = \ath{1}}
 است و بنابراین به ازای هر
\m{ \epsilon < 3}
داریم
\m{f(n) = O(n^{3 - \epsilon})} .
بنابراین حالت اول قضیه اصلی برقرار است. نتیجه می‌گیریم
\m{T(n) = \ath{n^3}}.
\end{itemize}
\end{frame}


\begin{frame}{‌روش قضیه اصلی}
\begin{itemize}\itemr
\item[-]
در تحلیل زمان اجرای الگوریتم استراسن رابطهٔ
\m{T(n) = 7T(n/2) + \ath{n^2}}
را به دست آوردیم. در این رابطهٔ بازگشتی
\m{a = 7}
و
\m{b = 2}
بنابراین
\m{n^{\log_2^7} = n^{\lg 7}}.
از آنجایی که
\m{\lg 7 = 2.8073...}
، می‌توانیم قرار دهیم
\m{\epsilon = 0.8}
و برای تابع محرک خواهیم داشت
\m{f(n) = \ath{n^2} = O \bigl( n^{\lg 7 - \epsilon} \bigr)}
، پس حالت اول در قضیه اصلی برقرار است و بنابراین جواب رابطهٔ بازگشتی برابر است با
\m{T(n) = \ath{n^{\lg 7}}}.
\end{itemize}
\end{frame}
