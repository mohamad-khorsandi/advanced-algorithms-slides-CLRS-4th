
\begin{frame}{‌ضرب ماتریس‌ها}
\begin{itemize}\itemr
\item[-]
از روش تقسیم و حل می‌توانیم برای ضرب ماتریس‌های مربعی استفاده کنیم.
\item[-]
فرض کنید
\m{A = (a_{ij})}
و
\m{B = (b_{ij})}
دو ماتریس
\m{n \times n}
باشند. ماتریس
\m{C = A \cdot B}
نیز یک ماتریس
\m{n \times n}
است که درایه‌های آن به صورت زیر محاسبه می‌شوند.
\begin{flushleft}
\m{c_{ij} = \sum_{k=1}^{n} a_{ik} \cdot b_{kj}}
\end{flushleft}
\end{itemize}
\end{frame}


\begin{frame}{‌ضرب ماتریس‌ها}
\begin{itemize}\itemr
\item[-]
الگوریتم ضرب دو ماتریس در زیر نوشته شده‌است.
\begin{algorithm}[H]\alglr
  \caption{Matrix} 
  \begin{algorithmic}[1]
    \Func{Matrix-Multiply}{A, B, C, n}
       \For{i = 1 \To n} \LeftComment{compute entries in each of n rows}
           \For{j = 1 \To n} \LeftComment{compute n entries in row i}
			   \For{k = 1 \To n}
					\State c[i,j] = c[i,j] + a[i,k] * b[k,j] \LeftComment{compute c[i,j]}
			    \EndFor
			\EndFor
	   \EndFor 
     \end{algorithmic}
  \label{alg:merge}
\end{algorithm}
\item[-]
از آنجایی که خط ۴ باید
\m{n^3}
بار تکرار شود، بنابراین زمان مورد نیاز برای اجرای این الگوریتم برابر است با
\ath{n^3}.
\end{itemize}
\end{frame}

\begin{frame}{‌ضرب ماتریس‌ها}
\begin{itemize}\itemr
\item[-]
حال می‌خواهیم ضرب دو ماتریس را توسط روش تقسیم و حل محاسبه کنیم.
\item[-]
در مرحله تقسیم، یک ماتریس
\m{n \times n}
را به چهار ماتریس
\m{n/2 \times n/2}
تقسیم می‌کنیم. برای سادگی فرض می‌کنیم n توانی از ۲ باشد و امکان تقسیم کردن آن به ۲ در فرایند الگوریتم تقسیم و حل وجود داشته باشد.
\end{itemize}
\end{frame}


\begin{frame}{‌ضرب ماتریس‌ها}
\begin{itemize}\itemr
\item[-]
با فرض اینکه هر یک از ماتریس‌های
A
,
B
و
C
را به چهار قسمت تقسیم کنیم، محاسبات به صورت زیر انجام می‌شود.
\begin{align*}
\m{C = A \cdot B \Rightarrow} \left( \begin{array}{cc} \m{C_{11}} & \m{C_{12}} \\ \m{C_{21}} & \m{C_{22}} \end{array} \right)
= \left( \begin{array}{cc} \m{A_{11}} & \m{A_{12}} \\ \m{A_{21}} & \m{A_{22}} \end{array} \right)
\left( \begin{array}{cc} \m{B_{11}} & \m{B_{12}} \\ \m{B_{21}} & \m{B_{22}} \end{array} \right)
\end{align*}
\item[-]
بنابراین خواهیم داشت :
\begin{align*}
\m{C_{11} = A_{11} \cdot B_{11} + A_{12} \cdot B_{21}}\\
\m{C_{12}= A_{11} \cdot B_{12} + A_{12} \cdot B_{22}}\\
\m{C_{21}= A_{21} \cdot B_{11} + A_{22} \cdot B_{21}}\\
\m{C_{22}= A_{21} \cdot B_{12} + A_{22} \cdot B_{22}}
\end{align*}
\end{itemize}
\end{frame}



\begin{frame}{‌ضرب ماتریس‌ها}
\begin{itemize}\itemr
\item[-]
بنابراین ضرب یک جفت ماتریس 
\m{n \times n}
را به ضرب هشت جفت ماتریس 
\m{n/2 \times n/2}
تبدیل کردیم.
\item[-]
توجه کنید که در این محاسبات نتیجهٔ ضرب
\m{A_{11} \cdot B_{11}}
و همچنین
\m{A_{12} \cdot B_{21}}
باید در 
\m{C_{11}}
ذخیره شود.
\begin{align*}
\m{C_{11} = A_{11} \cdot B_{11} + A_{12} \cdot B_{21}}\\
\m{C_{12}= A_{11} \cdot B_{12} + A_{12} \cdot B_{22}}\\
\m{C_{21}= A_{21} \cdot B_{11} + A_{22} \cdot B_{21}}\\
\m{C_{22}= A_{21} \cdot B_{12} + A_{22} \cdot B_{22}}
\end{align*}
\item[-]
حالت پایه در این الگوریتم وقتی است که می خواهیم دو ماتریس
\m{1 \times 1}
را در هم ضرب کنیم که در این حالت در واقع دو عدد را در هم ضرب می‌کنیم.
\end{itemize}
\end{frame}


\begin{frame}{‌ضرب ماتریس‌ها}
\begin{itemize}\itemr
\item[-]
الگوریتم تقسیم و حل برای ضرب دو ماتریس را می‌توانیم به صورت زیر بنویسیم.
\begin{algorithm}[H]\alglr
  \caption{Matrix} 
  \begin{algorithmic}[1]
    \Func{Matrix-Multiply-Recursive}{A, B, C, n}
   \Statex \LeftComment{Base case.}    
    \If{n==1}
    	\State \mc{c_{11}} = \mc{c_{11}} + \mc{a_{11}} * \mc{b_{11}}
    	\State \Return
    \EndIf
   \Statex \LeftComment{Divide.}
   \State Partition A, B, and C into n/2 \mc{\times} n/2 submatrices 
    	\Statex \mc{A_{11}, A_{12}, A_{21}, A_{22}}; 
    	\Statex \mc{B_{11}, B_{12}, B_{21}, B_{22}};
    	\Statex \mc{C_{11}, C_{12}, C_{21}, C_{22}}; respectively    
  \end{algorithmic}
  \label{alg:merge}
\end{algorithm}
\end{itemize}
\end{frame}


\begin{frame}{‌ضرب ماتریس‌ها}
\begin{itemize}\itemr
\item[-]
\begin{algorithm}[H]\alglr
  \caption{Matrix} 
  \begin{algorithmic}[1]
    \setcounter{ALG@line}{4}
   \Statex \LeftComment{Conquer.}   
    	\State Matrix-Multiply-Recursive (\mc{A_{11}, B_{11}, C_{11}, n/2})
    	\State Matrix-Multiply-Recursive (\mc{A_{11}, B_{12}, C_{12}, n/2})
    	\State Matrix-Multiply-Recursive (\mc{A_{21}, B_{11}, C_{21}, n/2})
    	\State Matrix-Multiply-Recursive (\mc{A_{21}, B_{12}, C_{22}, n/2})
    	\State Matrix-Multiply-Recursive (\mc{A_{12}, B_{21}, C_{11}, n/2})
    	\State Matrix-Multiply-Recursive (\mc{A_{12}, B_{22}, C_{12}, n/2})
    	\State Matrix-Multiply-Recursive (\mc{A_{22}, B_{21}, C_{21}, n/2})
    	\State Matrix-Multiply-Recursive (\mc{A_{22}, B_{22}, C_{22}, n/2})
  \end{algorithmic}
  \label{alg:merge}
\end{algorithm}
\end{itemize}
\end{frame}


\begin{frame}{‌ضرب ماتریس‌ها}
\begin{itemize}\itemr
\item[-]
یک ضرب ماتریسی با اندازهٔ 
\m{n}
 را به ۸ ضرب ماتریسی با اندازهٔ 
\m{n/2}
تبدیل کردیم.
\item[-]
فرض می‌کنیم که در تقسیم ماتریس به ماتریس‌های کوچک‌تر، تنها اندیس‌ها را نامگذاری مجدد می‌کنیم و بنابراین عملیات در زمان ثابت می‌تواند انجام شود.
\item[-]
بنابراین برای تحلیل این الگوریتم، می‌توانیم از رابطه بازگشتی زیر استفاده کنیم.
\begin{center}
\m{T(n) = 8 T(n/2) +  \ath{1}}
\end{center}
\end{itemize}
\end{frame}


\begin{frame}{‌ضرب ماتریس‌ها}
\begin{itemize}\itemr
\item[-]
با حل کردن این معادله به دست می‌آوریم
\m{T(n) = \ath{n^3}}.
\item[-]
بنابراین روش تقسیم و حل زمان محاسبات را در ضرب ماتریسی کاهش نمی‌دهد.
\item[-]
با استفاده از الگوریتم استراسن
\fn{1}{Strassen}
که از یک الگوریتم تقسیم و حل بهینه‌تر استفاده می‌کند، زمان محاسبات کاهش پیدا خواهد کرد.
\end{itemize}
\end{frame}