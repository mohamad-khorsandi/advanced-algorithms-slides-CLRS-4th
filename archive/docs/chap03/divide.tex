\begin{frame}{‌الگوریتم‌های تقسیم و حل}
\begin{itemize}\itemr
\item[-]
برای حل یک مسئله به روش‌های متنوعی می‌توان الگوریتم طراحی کرد.
\item[-]
الگوریتم مرتب‌سازی درجی یک الگوریتم ساده است که به روش افزایشی با مرتب‌سازی زیر آرایه‌های کوچک‌تر آرایه آغاز می‌شود و در نهایت کل آرایه را مرتب می‌کند. در واقع به ازای هر عنصر
\code{A[i]}
، این عنصر در مکان مناسب خود در زیر آرایه مرتب شدهٔ
\code{A[1 : i-1]}
قرار می‌گیرد.
\end{itemize}
\end{frame}


\begin{frame}{‌الگوریتم‌های تقسیم و حل}
\begin{itemize}\itemr
\item[-]
در این قسمت با روشی دیگر برای حل مسئله‌های محاسباتی آشنا می‌شویم، که به آن روش تقسیم و حل
\fn{1}{divide and conquer method}
(تقسیم و غلبه)
گفته می‌شود و الگوریتم‌هایی که از این روش استفاده می‌کنند، در دستهٔ الگوریتم‌های تقسیم و حل قرار می‌گیرند.
\item[-]
از روش تقسیم و حل برای حل مسئلهٔ مرتب‌سازی استفاده می‌کنیم و زمان اجرای آن را محاسبه می‌کنیم.
\item[-]
خواهیم دید که با استفاده از این روش، مسئلهٔ مرتب‌سازی در زمان کمتری نسبت به الگوریتم مرتب‌سازی درجی حل می‌شود.
\end{itemize}
\end{frame}


\begin{frame}{‌الگوریتم‌های تقسیم و حل}
\begin{itemize}\itemr
\item[-]
بسیاری از  الگوریتم‌های کامپیوتری بازگشتی
\fn{1}{recursive}
هستند. در یک الگوریتم بازگشتی، برای حل یک مسئله با یک ورودی معین ، خود الگوریتم با ورودی‌های کوچکتر فراخوانی می‌شود.
\item[-]
برای مثال، برای به دست آوردن فاکتوریل عدد n کافی است فاکتوریل عدد n-1 را فراخوانی کنیم.
\item[-]
به الگوریتم‌هایی که ورودی مسئله را تقسیم می‌کنند و به طور بازگشتی الگوریتم را برای قسمت‌های تقسیم شده فراخوانی می‌‌کنند، الگوریتم‌های تقسیم و حل گفته می‌شود.
\end{itemize}
\end{frame}


\begin{frame}{‌الگوریتم‌های تقسیم و حل}
\begin{itemize}\itemr
\item[-]
به عبارت دیگر یک الگوریتم تقسیم و حل یک مسئله را به چند زیر مسئله تقسیم می‌کند که مشابه مسئلهٔ اصلی هستند و الگوریتم را برای زیر مسئله‌ها فراخوانی می‌کند و سپس نتایج به دست آمده از زیر مسئله‌ها را با هم ترکیب می‌کند تا نتیجهٔ نهایی برای مسئلهٔ اصلی به دست آید.
\item[-]
معمولاً‌ پس از شکسته شدن یک مسئله به زیر مسئله‌ها، زیر مسئله‌هایی به دست می‌آیند که می‌توانند دوباره شکسته شوند و این روند تا جایی ادامه پیدا می‌کند که مسئله امکان شکسته شدن نداشته باشد. وقتی مسئله امکان شکسته شدن نداشته باشد، حالت پایه
\fn{1}{base case}
به دست می‌آید که حل مسئله در حالت پایه به سادگی امکان پذیر است.
\end{itemize}
\end{frame}


\begin{frame}{‌الگوریتم‌های تقسیم و حل}
\begin{itemize}\itemr
\item[-]
یک الگوریتم تقسیم و حل از سه مرحلهٔ زیر تشکیل شده‌است.\\
1. تقسیم
\fn{1}{divide}
: مسئله به چند زیر مسئله که نمونه‌های کوچکتر مسئلهٔ اصلی هستند تقسیم می‌شود.\\
۲. حل یا غلبه
\fn{2}{conquer}
: زیر مسئله‌ها به صورت بازگشتی حل می‌شوند.\\
۳. ترکیب
\fn{3}{combine}
: زیر مسئله‌های حل شده با یکدیگر ترکیب می‌شوند تا جواب مسئلهٔ اصلی به دست بیاید.
\end{itemize}
\end{frame}
