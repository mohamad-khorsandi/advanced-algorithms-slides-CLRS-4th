
\begin{frame}{‌الگوریتم استراسن}
\begin{itemize}\itemr
\item[-]
ضرب دو ماتریس
\m{n \times n}
را در زمان کمتر از
\m{n^3}
نیز می‌توان انجام داد. از آنجایی که برای ضرب دو ماتریس مربعی با اندازهٔ n دقیقا به
\m{n^3}
گام محاسباتی نیاز است،‌ بسیاری بر این باور بودند که ضرب ماتریسی نمی‌تواند در زمانی کمتر صورت بگیرد تا اینکه در سال ۱۹۶۹ ولکر استراسن
\fn{1}{Volker Strassen}
 ریاضیدان آلمانی، الگوریتمی با زمان اجرای
\ath{n^{\lg 7}}
ابداع کرد. از آنجایی که
\m{\lg 7 = 2.8073 ...}
، بنابراین می‌توان گفت الگوریتم استراسن در زمان
\abo{n^{2.81}}
ضرب دو ماتریس را محاسبه می‌کند.
\item[-]
الگوریتم استراسن یک الگوریتم از نوع و تقسیم و حل است.
\item[-]
استراسن مجددا در سال ۱۹۸۶ الگوریتمی از مرتبه
\abo{n^{2.48}}
ارائه داد.
در سال ۱۹۹۰ الگوریتمی از مرتبه
\abo{n^{2.38}}
و در سال ۲۰۲۳ یک الگوریتم بهبود یافته ارائه شد.
 جستجو برای پیدا کردن سریع‌ترین الگوریتم ضرب ماتریسی همچنان ادامه دارد.
\end{itemize}
\end{frame}


\begin{frame}{‌الگوریتم استراسن}
\begin{itemize}\itemr
\item[-]
ایدهٔ الگوریتم استراسن این است که در مراحل تقسیم و ترکیب از عملیات بیشتری استفاده می‌کند و بنابراین مراحل تقسیم و ترکیب در این الگوریتم نسبت به مراحل تقسیم و ترکیب در الگوریتم تقسیم و حل عادی زمان بیشتری صرف می‌کند ولی در ازای این افزایش زمان، در مرحله حل بازگشتی زمان کمتری مصرف می‌شود. در واقع در مرحله بازگشتی به جای فراخوانی ۸ تابع بازگشتی ۷ تابع بازگشتی فراخوانی می‌شوند.
\item[-]
به عبارت دیگر عملیات مورد نیاز برای یکی از فراخوانی‌های بازگشتی توسط تعدادی عملیات جمع در مراحل تقسیم و ترکیب انجام می‌شود.
\end{itemize}
\end{frame}


\begin{frame}{‌الگوریتم استراسن}
\begin{itemize}\itemr
\item[-]
به عنوان مثال فرض کنید می‌خواهیم به ازای دو عدد دلخواه 
\m{x}
 و 
\m{y}
  ، مقدار
\m{x^2 - y^2}
را محاسبه کنیم. اگر بخواهیم این محاسبات را به صورت معمولی انجام دهیم، باید ابتدا x و y را به توان ۲ برسانیم و سپس دو مقدار به دست آمده را از هم کم کنیم. اما یک روش دیگر برای این محاسبات وجود دارد.
\item[-]
می‌دانیم
\m{x^2 - y^2 = (x+y)(x-y)}
، بنابراین می‌توانیم این محاسبات را با یک عمل ضرب و دو عمل جمع و تفریق انجام دهبم. اگر x و y دو عدد باشند،‌ زمان انجام محاسبات تفاوت چندانی نخواهد کرد، اما اگر x و y دو ماتریس بزرگ باشند، یک عمل ضرب کمتر بهبود زیادی در زمان اجرا ایجاد می‌کند.
\item[-]
توجه کنید که جمع دو ماتریس مربعی با اندازهٔ n در زمان 
\m{O(n^2)}
انجام می‌شود، و ضرب دو ماتریس در زمان
\m{O(n^3)} .
\end{itemize}
\end{frame}


\begin{frame}{‌الگوریتم استراسن}
\begin{itemize}\itemr
\item[-]
حال که با ایدهٔ الگوریتم استراسن آشنا شدیم، الگوریتم را بررسی می‌کنیم.\\
۱. اگر
\m{n=1}
، آنگاه هر ماتریس تنها یک درایه دارد. در این صورت باید یک عملیات ضرب ساده انجام داد که در زمان
\ath{1}
امکان پذیر است. اگر
\m{n \neq 1}
، آنگاه هر یک از ماتریس‌های ورودی
\m{A}
و
\m{B}
 را به چهار ماتریس
\m{n/2 \times n/2}
تقسیم می‌کنیم. این عملیات نیز در
\ath{1}
امکان پذیر است.\\
۲. با استفاده از زیر ماتریس‌های به دست آمده از مرحله قبل تعداد ۱۰ ماتریس
\m{S_1, S_2, ..., S_{10}}
محاسبه می‌شوند. این عملیات در زمان
\ath{n^2}
انجام می‌شود.\\
۳. تابع ضرب ماتریسی به تعداد ۷ بار بر روی ماتریس‌های
\m{S_i}, \m{A_{ij}}, \m{B_{ij}}
که ابعاد هر کدام 
\m{n/2 \times n/2}
است، به طور بازگشتی انجام می‌شود. نتیجهٔ این محاسبات در ۷ ماتریس
\m{P_1, P_2, ..., P_7}
ذخیره می‌شود.
عملیات این مرحله در زمان
\m{7 T(n/2)}
انجام می‌شود.\\
۴. با استفاده از ماتریس‌های
\m{P_1, P_2, ..., P_7}
، ماتریس‌های
\m{C_{11}, C_{12}, C_{21}, C_{22}}
محاسبه می‌شود. این عملیات نیز در زمان
\ath{n^2}
انجام می‌شود.
\end{itemize}
\end{frame}


\begin{frame}{‌الگوریتم استراسن}
\begin{itemize}\itemr
\item[-]
بنابراین  زمان کل مورد نیاز برای الگوریتم استراسن از رابطه بازگشتی زیر به دست می‌آید.
\begin{flushleft}
\m{T(n) = 7 T(n/2) + \ath{n^2}}
\end{flushleft}
\item[-]
با حل این رابطه بازگشتی به دست می‌آوریم :
\begin{flushleft}
\m{T(n) = \ath{n^{\lg 7}} = O(n^{2.81})}
\end{flushleft}
\end{itemize}
\end{frame}


\begin{frame}{‌الگوریتم استراسن}
\begin{itemize}\itemr
\item[-]
حال ببینیم ماتریس‌های
\m{P_k}
چگونه با استفاده از ماتریس‌های
\m{A_{ij}}
و
\m{B_{ij}}
محاسبه می‌شوند.
\end{itemize}
\end{frame}


\begin{frame}{‌الگوریتم استراسن}
\begin{itemize}\itemr
\item[-]
در مرحلهٔ اول تعداد ۱۰ ماتریس 
\m{S_i}
 به صورت زیر محاسبه می‌شوند.
\begin{align*}
&\m{S_1 = B_{12} - B_{22}}\\
&\m{S_2 = A_{11} + A_{12}}\\
&\m{S_3 = A_{21} + A_{22}}\\
&\m{S_4 = B_{21} - B_{11}}\\
&\m{S_5 = A_{11} + A_{22}}\\
&\m{S_6 = B_{11} + B_{22}}\\
&\m{S_7 = A_{12} - A_{22}}\\
&\m{S_8 = B_{21} + B_{22}}\\
&\m{S_9 = A_{11} - A_{21}}\\
&\m{S_{10} = B_{11} + B_{12}}\\
\end{align*}
\end{itemize}
\end{frame}


\begin{frame}{‌الگوریتم استراسن}
\begin{itemize}\itemr
\item[-]
در محاسبات فوق ۱۰ بار ماتریس‌های
\m{n/2 \times n/2}
را با هم جمع کردیم که این عملیات در زمان
\ath{n^2}
امکان پذیر است.
\end{itemize}
\end{frame}


\begin{frame}{‌الگوریتم استراسن}
\begin{itemize}\itemr
\item[-]
در مرحلهٔ بعد ۷ ماتریس 
\m{P_i}
 را با استفاده از زیر ماتریس‌های 
\m{A_{ij}}
  و 
\m{B_{ij}}
   و ماتریس‌های 
\m{S_i}
    بدست می‌آوریم.
\begin{align*}
&\m{P_1 = A_{11} \cdot S_1 (= A_{11} \cdot B_{12} - A_{11} \cdot B_{22})}\\
&\m{P_2 = S_2 \cdot B_{22} (= A_{11} \cdot B_{22} + A_{12} \cdot B_{22})} \\
&\m{P_3 = S_3 \cdot B_{11} (= A_{21} \cdot B_{11} + A_{22} \cdot B_{11})}\\
&\m{P_4 = A_{22} \cdot S_4 (= A_{22} \cdot B_{21} - A_{22} \cdot B_{11})}\\
&\m{P_5 = S_5 \cdot S_6 (= A_{11} \cdot B_{11} + A_{11} \cdot B_{22} + A_{22} \cdot B_{11} + A_{22} \cdot B_{22})}\\
&\m{P_6 = S_7 \cdot S_8 (= A_{12} \cdot B_{21} + A_{12} \cdot B_{22} - A_{22} \cdot B_{21} - A_{22} \cdot B_{22})}\\
&\m{P_7 = S_9 \cdot S_{10} (= A_{11} \cdot B_{11} + A_{11} \cdot B_{12} - A_{21} \cdot B_{11} - A_{21} \cdot B_{12})}
\end{align*}
\end{itemize}
\end{frame}


\begin{frame}{‌الگوریتم استراسن}
\begin{itemize}\itemr
\item[-]
بنابراین در اینجا به ۷ عملیات ضرب نیاز داریم که به صورت بازگشتی انجام می‌شوند.
\item[-]
در مرحلهٔ آخر باید زیر ماتریس‌های 
\m{C_{ij}}
 را با استفاده از ماتریس‌های 
\m{P_i}
  به دست آوریم.
\item[-]
این محاسبات به صورت زیر انجام می‌شوند.
\begin{align*}
&\m{C_{11} = P_5 + P_4 - P_2 + P_6}\\
&\m{C_{12} = P_1 + P_2}\\
&\m{C_{21} = P_3 + P_4}\\
&\m{C_{22} = P_5 + P_1 - P_3 - P_7}
\end{align*}
\end{itemize}
\end{frame}


\begin{frame}{‌الگوریتم استراسن}
\begin{itemize}\itemr
\item[-]
با بسط دادن این روابط می‌توانیم
\m{C_{ij}}
ها را بر اساس
\m{A_{ij}}
و
\m{B_{ij}}
ها به دست آوریم و نشان دهیم که عملیات ضرب به درستی انجام می‌شود.
\begin{align*}
&\m{C_{11} = P_5 + P_4 - P_2 + P_6 (= A_{11} \cdot B_{11} + A_{12} \cdot B_{21})}\\
&\m{C_{12} = P_1 + P_2 (= A_{11} \cdot B_{12} + A_{12} \cdot B_{22})}\\
&\m{C_{21} = P_3 + P_4 (= A_{21} \cdot B_{11} + A_{22} \cdot B_{21})}\\
&\m{C_{22} = P_5 + P_1 - P_3 - P_7 (= A_{21} \cdot B_{12} + A_{22} \cdot B_{22})}
\end{align*}
\item[-]
در مرحله آخر تنها از عملیات جمع استفاده می‌کنیم بنابراین محاسبهٔ
\m{C_{ij}}
ها در زمان
\ath{n^2}
انجام می‌پذیرد.
\end{itemize}
\end{frame}


\begin{frame}{‌الگوریتم استراسن}
\begin{itemize}\itemr
\item[-]
دقت کنید که در این روش
در صورتی که تعداد سطرها یا ستون‌های یک ماتریس فرد باشند، می‌توان یک سطر با مقادیر صفر و یا یک ستون با مقادیر صفر به ماتریس اضافه کرد و پس از انجام عملیات ضرب سطر و ستون با مقادیر صفر را حذف کرد.
\end{itemize}
\end{frame}