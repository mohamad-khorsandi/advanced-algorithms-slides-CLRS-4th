\begin{frame}{‌تحلیل سرشکنی}
\begin{itemize}\itemr
\item[-]
فرض کنید عضو یک باشگاه می‌شوید. باشگاه از شما مبلغی به ازای حق عضویت دریافت می‌کند که باید ماهیانه بپردازید. اما در هر بار استفاده از باشگاه نیز باید مبلغی پرداخت کنید. فرض کنید حق عضویت ۶۰۰ تومان است و در هربار استفاده باید ۳۰ تومان بپردازد. اگر بخواهید هر روز از باشگاه استفاده کنید در واقع باید ماهیانه ۱۵۰۰ تومان یا به عبارت دیگر روزی ۵۰ تومان بپردازید.
\item[-]
وقتی هزینه را به طور میانگین به ازای واحدهای کوچک‌تر محاسبه می‌کنیم می‌گوییم هزینه‌ها را سرشکن
\fn{1}{amortize}
می‌کنیم.
\end{itemize}
\end{frame}


\begin{frame}{‌تحلیل سرشکنی}
\begin{itemize}\itemr
\item[-]
همچنین هنگامی که زمان اجرای یک الگوریتم را محاسبه می‌کنیم، می‌توانیم میانگین لازم را برای انجام عملیات محاسبه کنیم. چنین تحلیلی، تحلیل سرشکن
\fn{1}{amortize analysis}
گفته می‌شود. در تحلیل سرشکن الگوریتم، زمان کل اجرا بر تعداد عملیات تقسیم می‌شود و زمان لازم برای اجرای یک عمل به دست می‌آید.
\item[-]
تحلیل سرشکن، کارایی هر یک از عملیات را به طور متوسط مشخص می‌کند.
به عبارت دیگر، اگر تعدادی از عملیات به زمان اجرای زیادی لازم داشته باشند و بقیه عملیات زمان زیادی را صرف نکنند، با تقسیم زمان اجرای کل بر تعداد عملیات نشان می‌دهیم به طور متوسط هریک از عملیات در چه زمانی اجرا می‌شوند.
\end{itemize}
\end{frame}