%%%%%%%%%%%%%%%%%%%%%%%%%
\section{الگوریتم‌های حریصانه}
%%%%%%%%%%%%%%%%%%%%%%%%%

\begin{frame}{‌الگوریتم‌های حریصانه}
\begin{itemize}\itemr
\item[-]
مسئله‌های بهینه‌سازی به دنبال جواب بهینه در مجموعه‌ای از جواب‌ها برای یک مسئله می‌گردند. یک جواب بهینه جوابی است که در یک معیار اندازه‌گیری بهترین باشد. برای مثال یک جواب بهینه می‌تواند کوچک‌ترین، بزرگ‌ترین، کوتاهترین، بلندترین و غیره باشد.
\item[-]
برنامه‌ریزی پویا یکی از روش‌ها برای حل مسئله‌های بهینه‌سازی است.
\item[-]
الگوریتم‌های حریصانه
\fn{1}{greedy algorithms}
دسته‌ای دیگر از الگوریتم‌ها برای حل مسئله‌های بهینه‌سازی هستند.
\end{itemize}
\end{frame}


\begin{frame}{‌الگوریتم‌های حریصانه}
\begin{itemize}\itemr
\item[-]
فرض کنید می‌خواهیم در مدت
\m{n}
روز به بیشترین دارایی ممکن دست پیدا کنیم.
برای این کار کافی است که در هر روز بیشترین دارایی ممکن را کسب کنیم.
بنابراین برای به دست آوردن بیشترین دارایی در مدت
\m{n}
روز باید در روز اول بیشترین دارایی را کسب کنیم و زیر مسئله‌ای که باید در گام بعد حل شود این است که چگونه در مدت
\m{n-1}
روز بیشترین دارایی را کسب کنیم.
پس در هر روز یک انتخاب حریصانه انجام می‌دهیم و آن انتخاب، کسب بیشترین دارایی در همان روز است. می‌دانیم که با این انتخاب حریصانه در مدت
\m{n}
روز بیشترین دارایی را کسب خواهیم کرد.
\item[-]
یک الگوریتم حریصانه مسئله را از بالا به پایین حل می‌کند. برای حل یک مسئله به روش حریصانه در هر گام یک انتخاب بهینه انجام می‌شود و از یک مسئله یک زیرمسئله به دست می‌آید.
این فرایند ادامه پیدا می‌کند تا زیرمسئله‌ای باقی نماند. در پایان جواب مسئله مجموعهٔ همهٔ انتخاب‌های بهینه است.
به عبارت دیگر در هر گام یک انتخاب حریصانه برای به دست آوردن جواب بهینه صورت می‌گیرد و در پایان مجموعهٔ همهٔ انتخاب‌های حریصانه جواب مسئله است.
% و در هر گام الگوریتم انتخاب‌هایی می‌کند تا به پاسخ بهینه برسد. در الگوریتم‌های حریصانه 
% در هر گام، الگوریتم بهینه‌ترین انتخاب را انجام می‌دهد به این امید که در پایان جواب بهینه به دست آید.
%\item[-]
%الگوریتم‌های حریصانه تنها در برخی از مسائل با ویژگی‌های خاص به پاسخ بهینه دست می‌یابند.
\end{itemize}
\end{frame}


\begin{frame}{‌الگوریتم‌های حریصانه}
\begin{itemize}\itemr
\item[-]
تنها برخی از مسئله‌های بهینه‌سازی را می‌توان به روش حریصانه حل کرد.
\item[-]
برای مثال مسئله درخت جستجوی دودویی بهینه را در نظر بگیرید. اگر در هر گام برای ساختن درخت جستجوی دودویی بهینه از بین همهٔ کلیدها، کلیدی را به عنوان ریشه در نظر بگیریم که بیشترین احتمال وقوع را داشته باشد، درخت به دست آمده الزاما بهینه نیست.
\item[-]
همانطور که مشاهده کردیم ریشهٔ یک درخت جستجوی دودویی بهینه ممکن است کلیدی باشد که بیشترین احتمال وقوع را نداشته باشد.
%\item[-]
%در صورتی یک مسئله را می‌توان به روش حریصانه حل کرد که پس از نوشتن رابطه بازگشتی، تنها یک زیرمسئله جواب مسئله باشد.
\end{itemize}
\end{frame}
