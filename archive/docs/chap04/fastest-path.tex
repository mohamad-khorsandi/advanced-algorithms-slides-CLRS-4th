
\begin{frame}{‌سریع‌ترین مسیر در جدول}
\begin{itemize}\itemr
\item[-]
جدول
\m{A}
با
\m{m}
سطر و
\m{n}
ستون را در نظر بگیرید به طوری‌که مقدار هرکدام از درایه‌های جدول یک عدد صحیح است.
\item[-]
می‌خواهیم مهره‌ای را با شروع از درایه
\m{(1,1)}
حرکت داده، به درایهٔ
\m{(m,n)}
منتقل کنیم. فرض کنید درایهٔ
\m{(1,1)}
در شمال غرب و درایهٔ
\m{(m,n)}
در جنوب شرق جدول قرار دارد.
\item[-]
وقتی مهره وارد درایهٔ
\m{(i,j)}
می‌شود، باید
\m{A[i,j]}
ثانیه در آن درایه صبر کند و پس از آن به حرکت ادامه دهد.
\item[-]
با فرض اینکه مهره تنها می‌تواند به سمت جنوب یا شرق یا مورب به جنوب شرق حرکت کند، می‌خواهیم کمترین زمان ممکن برای انتقال مهره از درایهٔ
\m{(1,1)}
به
\m{(m,n)}
را محاسبه کنیم.
\end{itemize}
\end{frame}


\begin{frame}{‌سریع‌ترین مسیر در جدول}
\begin{itemize}\itemr
\item[-]
در گام اول بررسی می‌کنیم مسئله دارای زیر ساختار بهینه است یا اصل بهینگی در آن برقرار است.
\item[-]
اگر از درایهٔ
\m{(1,1)}
مهره را حرکت داده در کمترین زمان ممکن به درایهٔ
\m{(i,j)}
برسیم حتما از یکی از سه درایهٔ
\m{(i-1,j)}
یا
\m{(i,j-1)}
یا
\m{(i-1,j-1)}
عبور کرده‌ایم.
\item[-]
در صورتی که از
\m{(i-1,j)}
عبور کرده باشیم الزاما برای حرکت از درایه
\m{(1,1)}
به
\m{(i-1,j)}
کمترین زمان ممکن را صرف کرده‌ایم. این گزاره را می‌توانیم با برهان خلف اثبات کنیم.
\item[-]
به همین ترتیب ممکن است از درایه‌های
\m{(i,j-1)}
یا
\m{(i-1,j-1)}
عبور کرده باشیم که به طور مشابه می‌توانیم اثبات کنیم الزاما در کمترین زمان ممکن به این درایه‌ها رسیده‌ایم.
\end{itemize}
\end{frame}


\begin{frame}{‌سریع‌ترین مسیر در جدول}
\begin{itemize}\itemr
\item[-]
در گام دوم رابطه‌ای برای توصیف جواب مسئله براساس جواب زیر مسئله‌ها به صورت زیر می‌نویسیم.
\begin{align*}
\m{T[i,j]} = \left\{\begin{array}{lr}
          \m{\min(T[i-1,j],T[i,j-1],T[i-1,j-1]) + A[i,j]}&\m{j > 1 , i > 1}~\text{اگر}\\
          \m{T[i-1,1] + A[i,1]}&\m{j = 1 , i > 1}~\text{اگر}\\
          \m{T[1,j-1] + A[1,j]}&\m{i = 1 , j > 1}~\text{اگر}\\
          \m{A[1,1]}&\m{j = 1 , i = 1}~\text{اگر}
\end{array}\right.
\end{align*}
\end{itemize}
\end{frame}


\begin{frame}{‌سریع‌ترین مسیر در جدول}
\begin{itemize}\itemr
\item[-]
سپس جدول
\m{T}
را در زمان
\ath{mn}
تکمیل می‌کنیم و
\m{T[m,n]}
جواب مسئله است.
\end{itemize}
\end{frame}