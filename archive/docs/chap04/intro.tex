%%%%%%%%%%%%%%%%%%%%%%%%%
\section{برنامه‌ریزی پویا}
%%%%%%%%%%%%%%%%%%%%%%%%%

\begin{frame}{‌برنامه‌ریزی پویا}
\begin{itemize}\itemr
\item[-]
برنامه‌ریزی پویا
\fn{1}{dynamic programming}
روشی دیگر برای حل مسائل محاسباتی است که توسط آن همانند روش تقسیم و حل جواب یک مسئله از جواب زیر مسئله‌های آن به دست می‌آید.
{\footnotesize{
(در اینجا واژه
programming
به معنی برنامه‌ریزی و طراحی یک جدول برای پیش‌بینی آینده است و نه به معنی برنامه نویسی.)
}}
\item[-]
در برنامه‌ریزی پویا هر یک از زیر مسئله‌ها تنها یک بار حل می‌شوند. جواب یک زیرمسئله در یک جدول ذخیره می‌شود و از آن جواب برای حل زیر مسئله‌های دیگر با اندازه‌های بزرگتر استفاده می‌شود. در واقع در هر مرحله یک زیر مسئلهٔ بزرگ‌تر با استفاده از جواب یک زیر مسئله کوچک‌تر حل می‌شود و این روند ادامه پیدا می‌کند تا اینکه مسئلهٔ اصلی با استفاده از بزرگترین زیرمسئلهٔ به دست آمده حل می‌شود.
\item[-]
برنامه‌ریزی پویا در بسیاری از مسائل بهینه سازی
\fn{2}{optimization problem}
کاربرد دارد. چنین مسئله‌هایی معمولا چند جواب دارند که ما به دنبال جوابی می‌گردیم که مقدار بهینه (کوچکترین یا بزرگترین) داشته باشد.
\end{itemize}
\end{frame}



\begin{frame}{‌برنامه‌ریزی پویا}
\begin{itemize}\itemr
\item[-]
برای مثال برای پیدا کردن عدد فیبوناچی n ام، باید عدد فیبوناچی n-1 ام و عدد فیبوناچی n-2 ام را محاسبه کنیم.
\item[-]
می‌توانیم یک رابطه بازگشتی برای محاسبه عدد فیبوناچی بنویسیم و با استفاده از یک الگوریتم بازگشتی آن را حل کنیم. مشکل الگوریتم بازگشتی این است که برخی از زیرمسئله‌ها بیش از یک بار حل می‌شوند. برای مثال برای محاسبه عدد فیبوناچی پنجم عدد فیبوناچی چهارم و سوم محاسبه می‌شوند. اما هنگام محاسبه عدد فیبوناچی چهارم عدد سوم برای بار دوم باید محاسبه شود.
\end{itemize}
\end{frame}

\begin{frame}{‌برنامه‌ریزی پویا}
\begin{itemize}\itemr
\item[-]
یک روش برای حل این مشکل این است که به جای حل مسئله از بالا به پایین،یعنی با شروع از مسئله بزرگ‌تر و محاسبه زیرمسئله‌های کوچکتر، مسئله را از پایین به بالا حل کنیم، بدین معنی که ابتدا زیرمسئله را حل کنیم و نتایج را ذخیره کرده و از نتایج در مسئله‌های بزرگتر استفاده کنیم.
\item[-]
برای مثال در مسئله محاسبه عدد فیبوناچی n ام، ابتدا عدد فیبوناچی اول، سپس دوم، سوم، .. را محاسبه کرده تا به عدد فیبوناچی n ام برسیم.
\item[-]
این روش حل مسئله از پایین به بالا با شروع به محاسبه زیرمسئله‌های کوچکتر و استفاده از جواب زیرمسئله‌ها در مسئله‌های بزرگتر برنامه‌ریزی پویا نامیده می‌شود.
\end{itemize}
\end{frame}