
\begin{frame}{نتیجه‌گیری}
\begin{itemize}\itemr
\item[-]
یک الگوریتم روشی است گام‌به‌گام برای حل یک مسئلهٔ محاسباتی.
\item[-]
روش‌های گوناگون برای طراحی یک الگوریتم برای یک مسئله محاسباتی وجود دارد.
\item[-]
برای یک مسئله ممکن است الگوریتم‌های متفاوت با رویکردهای متفاوت وجود داشته باشد. دو معیار مهم سنجش الگوریتم‌ها زمان اجرا و میزان حافظه استفاده شده توسط آنها است. ممکن است یک الگوریتم زمان اجرای بسیار بالایی داشته باشد ولی میزان حافظهٔ مورد نیاز آن نیز بسیار بالا باشد. چنین الگوریتمی در موقعیت‌هایی کاربرد دارد که زمان اجرا مهم‌ترین معیار سنجش است.
برای مثال در سیستم‌های بلادرنگ نیاز به زمان پاسخ پایین وجود دارد.
 الگوریتم دیگری می‌تواند زمان حافظهٔ بسیار کمی استفاده کند ولی زمان اجرای آن نیز نسبتا پایین باشد چنین الگوریتمی در موقعیتی کاربرد دارد که میزان حافظه بسیار حائز اهمیت است. برای مثال در سیستم‌های نهفته حافظه محدود است. همچنین زمان اجرا و میزان حافظه مورد نیاز یک الگوریتم ممکن است حد وسط باشد. چنین الگوریتمی وقتی استفاده می‌شود که هر دو معیار زمان و حافظه به یک اندازه اهمیت داشته باشند.
\end{itemize}
\end{frame}


\begin{frame}{نتیجه‌گیری}
\begin{itemize}\itemr
\item[-]
روش‌ها و رویکردهای متفاوتی را برای حل مسائل محاسباتی از جمله روش تقسیم و حل، برنامه‌ریزی پویا، حریصانه و جستجوی فضای حالت را بررسی کردیم.
\item[-]
روش‌های تقسیم و حل و برنامه‌ریزی پویا و حریصانه برای حل مسئله در زمان چندجمله‌ای به کار می‌روند. اما مسئله‌های زیادی وجود دارند که هنوز الگوریتم چندجمله‌ای برای آنها دریافت نشده است و بنابراین برای حل اینگونه مسائل باید همهٔ جواب‌های احتمالی بررسی شوند تا جواب مورد نظر پیدا شود.
\item[-]
چنین الگوریتم‌هایی در دستهٔ الگوریتم‌های جستجوی فضای حالت یا جستجوی ترکیبیاتی قرار می‌گیرند.
\end{itemize}
\end{frame}


\begin{frame}{نتیجه‌گیری}
\begin{itemize}\itemr
\item[-]
 اگر فضای حالت بسیار بزرگ باشد ممکن است حتی برای مسئله‌های نسبتاً کوچک رویکرد جستجوی ترکیبیاتی در زمان معقول پاسخگو نباشد.
 روش پسگرد روشی است که برای بررسی فضای حالت به طور منظم استفاده می‌شود و تعدادی از حالات حذف می‌شوند.
همچنین در مسائل بهینه‌سازی به منظور کاهش تعداد حالات از روش شاخه و کران برای کوچک‌کردن فضای حالت استفاده می‌کنیم.
\item[-]
مسائلی وجود دارند که برای آنها الگوریتم چند جمله‌ای یافته نشده است، اما الگوریتم‌های تقریبی وجود دارد که در زمان چند جمله‌ای جواب نزدیک به جواب دقیق یا به عبارت دیگر جوابی تقریبی برای مسئله پیدا می‌کنند. در الگوریتم‌های تقریبی اثبات می‌شود جواب تقریبی یافته شده توسط الگوریتم چه میزان انحراف از جواب دقیق مسئله خواهد داشت.
\end{itemize}
\end{frame}