
\begin{frame}{تابع پیچیدگی زمانی}

\begin{itemize}\itmsep{5mm}
\item[-]
تابع پیچیدگی زمانی برای یک الگویتم، زمان (و یا تعداد گام‌هایی) را مشخص می‌کند که الگوریتم برای پیدا کردن جواب مسئله نیاز دارد،
به طوری که این زمان تابع اندازهٔ ورودی مسئله است.
\item[-]
بنابراین اگر ورودی یک مسئله 
\m{n}
 باشد و تعداد گام‌های لازم برای محاسبهٔ جواب مسئله توسط یک الگوریتم 
\m{f(n)}
باشد، می‌گوییم زمان محاسبه
\fn{1}{computation time}
یا زمان اجرا
\fn{2}{running time}
 یا پیچیدگی زمانی
\fn{3}{time complexity}
  الگوریتم
\m{f(n)}
است.
\item[-]
هدف طراحی الگوریتم‌ها، پیدا کردن الگوریتم‌هایی برای مسائل است که زمان اجرای آنها تا حد امکان کوچک باشد.
هر تابع با یک نرخ معین رشد می‌کند.
می‌توانیم نرخ رشد توابع
\fn{4}{growth rate of functions}
 مختلف را با یکدیگر مقایسه کنیم.
\end{itemize}

\end{frame}

\begin{frame}{مقایسه رشد توابع}

\begin{figure}[!ht]
\centering
\includestandalone[width=0.5\textwidth]{figs/chap01/complexity}
\label{fig:complexity}
\end{figure}

\end{frame}

\begin{frame}{مقایسه رشد توابع پیچیدگی}

\begin{itemize}\itmsep{2mm}
\item[-]
اگر هر گام در یک الگوریتم فقط یک میکروثانیه زمان ببرد، می‌توانیم زمان تقریبی محاسبه به ازای توابع رشد متفاوت را به صورت زیر با یکدیگر مقایسه کنیم.
\end{itemize}

\begin{center}
\begin{tabular}{c | c c c}
اندازهٔ $n$ 
 & 20 & 40 & 60 \\ \hline
تابع پیچیدگی $f(n)$ &  & &
\\
$n$ &
۰/‌۰۰۰۰۲ ثانیه &
۰/‌۰۰۰۰۴ ثانیه &
۰/‌۰۰۰۰۶ ثانیه \\
$n^2$ &
۰/‌۰۰۰۴ ثانیه &
۰/‌۰۰۱۶ ثانیه &
۰/‌۰۰۳۶ ثانیه \\
$n^3$ &
۰/‌۰۰۸ ثانیه &
۰/‌۰۶۴ ثانیه &
۰/‌۲۱۶ ثانیه \\
$n^5$ &
۳/۲ ثانیه &
۱/۷ دقیقه &
۱۳ دقیقه \\
$2^n$ &
۱ ثانیه &
۱۲/۷ روز &
۳۶۶ قرن \\
$3^n$ &
۵۸ دقیقه &
۳۸۵۵ قرن &
$\text{۱/۳} \times \text{10}^\text{13}$ قرن 
\\
\end{tabular}
\end{center}
\end{frame}


\begin{frame}{مسئله کوتاهترین مسیر}
مسئله کوتاهترین مسیر بین دو شهر
\m{c_x}
و
\m{c_y}
تشکیل شده است از شرحی از پارامتر های مسئله یعنی:
\begin{itemize}\itmsep{2mm}
\item[-]
مجموعه ای از شهرها،
\m{C = \{ c_1, c_2, ..., c_m \}}
\item[-]
مجموعه ای از جاده‌ها، به طوری که هر جاده دو شهر را به هم متصل می‌کند،
\m{R \subseteq C \times C}
\item[-]
تابعی که به ازای هر دو شهر به هم متصل شده توسط یک جاده، طول جاده را مشخص می‌کند،
\m{len : R \rightarrow \NN}
\end{itemize}

و همچنین شرحی از جواب مسئله یعنی:
\begin{itemize}\itmsep{5mm}
\item[-]
مسیر
\m{P = \langle c_{f(1)}, c_{f(2)}, ..., c_{f(n)} \rangle}
با
\m{f(1) = x}
،
\m{f(n) = y}
،
\m{1 < n \leq m}
وجود داشته باشد، به طوری که
مقدار پارامتر
\m{L}
 کمینه باشد.

\begin{center}
\m{L = \sum_{i=1}^{n-1} len(c_{f(i)}, c_{f(i+1)})}
\end{center}

\end{itemize}
\end{frame}

\begin{frame}{نمونهٔ مسئله کوتاهترین مسیر}
کوتاهترین مسیر بین دو شهر
\m{c_1}
و
\m{c_5}
را پیدا کنید به طوری که
\begin{align*}
\m{C} =~& \m{ \{ c_1, c_2, c_3, c_4, c_5 \}} \\
\m{R} =~& \m{\{ (c_1, c_2), (c_1,c_3), (c_2, c_3), (c_2, c_4), (c_3, c_4), (c_3, c_5), (c_4, c_5) \}} \\
\m{len} =~& \m{\{ ((c_1, c_2),2), ((c_1,c_3),5), ((c_2, c_3),1), ((c_2, c_4),6),}  \\
~& \m{((c_3, c_4),4), ((c_3, c_5),8), ((c_4, c_5),3) \}}
\end{align*}

\begin{figure}[!ht]
  \centering
  \includestandalone{figs/chap01/g1}
  \label{fig:g1}
\end{figure}

\end{frame}

\begin{frame}{یک الگوریتم ساده برای مسئله کوتاهترین مسیر}
\begin{itemize}\itmsep{2mm}
\item[1.] همهٔ مسیرها از 
\m{c_x}
به
\m{c_y}
را پیدا می‌کنیم.
\item[2.]
طول همهٔ آن مسیرها را محاسبه می‌کنیم.
\item[3.]
کوتاهترین مسیر را به دست می‌آوریم.
\end{itemize}
\end{frame}


\begin{frame}{پیچیدگی زمانی الگوریتم سادهٔ کوتاهترین مسیر}
\begin{itemize}\itmsep{5mm}
\item[-]
در بدترین حالت همهٔ شهرها با یک مسیر به هم متصل شده‌اند.
\item[-]
در این صورت تعداد همهٔ مسیرها از شهر
\m{c_x}
به شهر
\m{c_y}
شامل
m
شهر
برابر است با
\m{(m-2)!}.
\item[-]
بنابراین تنها برای شمردن همهٔ مسیرها شامل m شهر و مقایسهٔ طول آنها به
\m{(m-2)!}
گام زمانی نیاز داریم.
\item[-]
رشد تابع
\m{n!}
از رشد تابع
\m{2^n}
نیز سریع تر است.
\item[-]
اگر بررسی هر مسیر فقط یک میکروثانیه زمان ببرد، برای شمارش و بررسی همهٔ مسیرها در مجموعه‌ای با تنها ۶۰ شهر به 
$\text{۲/۶} \times \text{10}^\text{66}$
قرن زمان نیاز داریم.
($\text{60} ! = \text{۸/۳} \times \text{10}^\text{81}$)
\end{itemize}
\end{frame}

\begin{frame}{الگوریتم کوتاهترین مسیر دایکسترا}
\begin{itemize}\itmsep{5mm}
\item[-] 
الگوریتم کوتاهترین مسیر دایکسترا توسط ادسخر دایکسترا
\fn{1}{Edsger Dijkstra}
 در سال ۱۹۵۶ ابداع شد.
\item[-]
این الگوریتم
کوتاهترین مسیر از
\m{c_x}
به
\m{c_y}
را محاسبه می‌کند.
\item[-]
 پیچیدگی زمانی این الگوریتم از مرتبه
\m{n^2}
است.
\item[-]
با استفاده از الگوریتم دایکسترا برای مجموعه‌ای از ۶۰ شهر به تنها ۰/‌۰۰۳۶ ثانیه زمان نیاز داریم.
\end{itemize}
\end{frame}