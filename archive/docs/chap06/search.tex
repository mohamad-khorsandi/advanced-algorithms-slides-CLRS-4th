%%%%%%%%%%%%%%%%%%%%%%%%%
\section{جستجوی ترکیبیاتی}
%%%%%%%%%%%%%%%%%%%%%%%%%

\begin{frame}{‌جستجوی ترکیبیاتی}
\begin{itemize}\itemr
\item[-]
روش‌های تقسیم و حل، برنامه‌ریزی پویا و حریصانه را برای حل دسته‌ای از مسئله‌های محاسباتی بررسی کردیم. در مسئله‌های بررسی شده، یک راه‌حل ساده برای پیدا کردن جواب مسئله جستجوی همهٔ حالت‌ها است ولی چنین جستجویی در زمان معقول (زمان چند جمله‌ای) امکان پذیر نیست چرا که تعداد همهٔ حالت‌های ممکن از مرتبه نمایی است و بنابراین جستجوی همهٔ حالت‌ها در زمان نمایی صورت می‌گیرد. توسط روش‌های تقسیم و حل، برنامه‌ریزی پویا و حریصانه روش‌هایی برای حل مسئله‌ها در زمان معقول ارائه کردیم.
\item[-]
پیدا کردن جواب در زمان چند جمله‌ای برای همهٔ مسئله‌ها همیشه امکان پذیر نیست و گاهی تنها راه حل ، جستجوی فضای حالت برای آن مسئله است. جستجوی فضای حالت به معنی تولید همهٔ جواب‌های مسئله و انتخاب حالت بهینه یا حالت مورد نظر از بین همهٔ حالت‌های ممکن است. تعداد این حالت‌ها از مرتبهٔ نمایی است و بنابراین چنین الگوریتم‌هایی به زمان نمایی برای محاسبه نیاز دارند و در نتیجه برای مسائلی با اندازهٔ بزرگ در عمل قابل استفاده نیستند.
\end{itemize}
\end{frame}


\begin{frame}{‌جستجوی ترکیبیاتی}
\begin{itemize}\itemr
\item[-]
در مبحث جستجوی ترکیبیاتی
\fn{1}{combinatorial search}
، روش‌های جستجوی همه فضای حالت برای یک مسئله بررسی می‌شود. در این روش‌ها مطالعه می‌کنیم چگونه به طور منظم همهٔ حالت‌ها را بررسی کنیم و یا اینکه چگونه با استفاده از روش‌هایی فضای حالت را محدود کنیم.
\item[-]
ترکیبیات
\fn{2}{combinatorics}
شاخه‌ای از ریاضیات است که در آن به بررسی ساختارهای متناهی و شمارش این ساختارها می‌پردازیم.
\item[-]
برای مثال گراف یک ساختار متناهی است و تعداد مسیرها در یک گراف، متناهی است. برای پیدا کردن بلندترین مسیر در یک گراف می‌توانیم همهٔ مسیرها را بررسی و بلندترین آنها را انتخاب کنیم.
\item[-]
بهینه سازی ترکیبیاتی
\fn{3}{combinatorial optimization}
شاخه‌ای از بهینه‌سازی است که در آن مجموعهٔ جواب‌های امکان پذیر گسسته است و هدف پیدا کردن جواب بهینه از بین همهٔ جواب‌های ممکن است.
\end{itemize}
\end{frame}